% the abstract

%\addcontentsline{toc}{chapter}{\numberline{}Abstract} ===>NOTE that adding \numberline{} alines the word "Abstract" with the numbered chapter titles
Nanopores can be used as extremely sensitive instruments to directly measure properties of individual molecules.  Progress in the nanopore field has been motivated largely by a desire to sequence DNA quickly and accurately.  Recent developments, notably the engineering of the nanopore protein \textit{Mycobacterium smegmatis} porin A [MspA] for use in experiments with DNA, have enabled the identification of nucleotide sequence based on measurements of ionic current.

This work is focused on attempting to understand and control the motion of a single DNA molecule in an MspA nanopore.  The behavior of individual molecules is dominated by thermal motion.  Experiments are carried out to determine the biophysical properties of ssDNA that govern its thermal motion when trapped in MspA: the effective charge and the diffusion constant.  These quantities differ depending on the identity of the nucleotides in the nanopore, as do the measured current blockage and the current fluctuations.  Interesting correlations exist between the effective charge of a given nucleotide and the amount of current it blocks, as well as between the diffusion constant and the measured current fluctuations.  At large electrophoretic forces, the existence of a noise source independent of the ssDNA itself is hypothesized: noise having to do with the presence of a protein attached to the ssDNA.

Further work makes use of a DNA helicase enzyme as a biochemical ratchet to step the DNA through the nanopore one base at a time under the stochastic action of enzymatic ATP hydrolysis.  Having quantified the electrophoretic force on ssDNA in an MspA nanopore, we demonstrate that mechanical force can be substituted for the action of ATP to induce a helicase to unwind dsDNA.  The kinetics of this force-induced stepping of a helicase along DNA are reported at various temperatures, applied electrophoretic forces, and buffer conditions.  It is hoped that future work in this direction will lead to the development of a method for stepping a helicase along DNA deterministically.
