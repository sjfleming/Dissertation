\chapter{Pulse generator design}
\label{pulse_generator}

This pulse generator is designed to produce simultaneous positive and negative voltage pulses of variable voltage up to \SI{5}{\V}, with variable duration from \SI{50}{\ns} to \SI{10}{\ms}.  The requirements were that the noise be as low as possible, and ideally zero when the outputs are grounded.  This resulted in a battery-powered design using FET switches for the outputs.  The generator accepts control TTL input signals.  Ground of the input is isolated from the ground of the generator itself using an optocoupler.  Parts were through-hole mounted and soldered on a breadboard, contained inside an aluminum box, with knobs and connections on the outside.  Special thanks to Jim MacArthur, who helped with the design.

\begin{figure}[h]
\begin{centering}
\includegraphics[width=0.7\textwidth]{figures/pulse_gen_diagram.png}
\caption[Pulse generator: block diagram]{Block diagram of the basic idea for the low-noise pulse generator.}
\label{fig:pulse_gen_diagram}
\end{centering}
\end{figure}

\begin{figure}[h]
\begin{centering}
\includegraphics[width=\textwidth]{figures/pulse_gen_schematic.jpg}
\caption[Pulse generator: schematic]{Initial design schematic for the low-noise pulse generator, as drawn by Jim MacArthur, the Electronics Shop Manager at Harvard.}
\label{fig:pulse_gen_schematic}
\end{centering}
\end{figure}

\begin{figure}[h]
\begin{centering}
\includegraphics[width=0.95\textwidth]{figures/pulse_gen_board.pdf}
\caption[Pulse generator: circuit wiring diagram]{Wiring diagram for the implementation of the low-noise pulse generator.}
\label{fig:pulse_gen}
\end{centering}
\end{figure}

\begin{figure}[h]
\begin{centering}
\includegraphics[width=0.95\textwidth]{figures/pulse_gen_switch_debounce.pdf}
\caption[Pulse generator: switch debouncer]{Switch debouncer for the pulse generator, when used in manual push-button mode.  Not shown: TTL inputs go through an HCPL2211 optocoupler, which is used to decouple the input ground of a TTL signal from the pulse generator ground.  The connection of the HCPL2211 is shown in Figure \ref{fig:pulse_gen_schematic}.}
\label{fig:pulse_gen_debounce}
\end{centering}
\end{figure}