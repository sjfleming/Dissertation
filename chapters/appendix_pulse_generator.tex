\chapter{Pulse generator design}
\label{pulse_generator}

Specifications (V, duration, low noise).  How it will be connected to experiment.

\section{Design}

\begin{figure}[h]
\begin{centering}
\includegraphics[width=0.7\textwidth]{figures/pulse_gen_diagram.png}
\caption[Pulse generator: block diagram]{Block diagram of the basic idea for the low-noise pulse generator.}
\label{fig:pulse_gen_diagram}
\end{centering}
\end{figure}

\begin{figure}[h]
\begin{centering}
\includegraphics[width=\textwidth]{figures/pulse_gen_schematic.jpg}
\caption[Pulse generator: schematic]{Initial design schematic for the low-noise pulse generator, as drawn by Jim MacArthur, the Electronics Shop Manager at Harvard.}
\label{fig:pulse_gen_schematic}
\end{centering}
\end{figure}

\section{Schematic}

\begin{figure}[h]
\begin{centering}
\includegraphics[width=0.95\textwidth]{figures/pulse_gen_board.pdf}
\caption[Pulse generator: circuit wiring diagram]{Wiring diagram for the implementation of the low-noise pulse generator.}
\label{fig:pulse_gen}
\end{centering}
\end{figure}

\begin{figure}[h]
\begin{centering}
\includegraphics[width=0.95\textwidth]{figures/pulse_gen_switch_debounce.pdf}
\caption[Pulse generator: switch debouncer]{Switch debouncer and ground isolator for the pulse generator.}
\label{fig:pulse_gen_debounce}
\end{centering}
\end{figure}

\section{Spike compensation}

Compensation of capacitive current spikes.