\chapter{The first-passage problem}
\label{first_passage}

\section{Justification}

Why one dimension?  Why the (homogeneous (time-independent drift and diffusion) Fokker-Planck / Smoluchowski equation?  What is the potential or quasipotential?  Justifications for drag coefficient being Einstein relation and for the entropic considerations.

\section{First passage time}

The mathematical discussion here follows the treatment of first passage times and the one-dimensional Fokker-Planck equation by Gardiner \citep{Gardiner1985}.

Probability a diffusing particle is still in the interval $(0,L)$ is the survival probability, $S(x,t)$

\begin{equation}
S(x_0,t) = \int_{0}^{L} p(x,t \lvert x_0,0) \,dx
\label{eqn:survival}
\end{equation}

Smoluchowski equation:

\begin{equation}
\pard{S(x_0,t)}{t} = A(x) \pard{S(x_0,t)}{x} + \frac{1}{2} B(x) \pardd{S(x_0,t)}{x}
\label{eqn:smoluchowski}
\end{equation}

Particle starts at position $x_0$, and so $S(x_0,0) = 1$ on the interval $(0,L)$, and is zero elsewhere at $t=0$.  $x=L$ is the absorbing boundary, so $S(L,t)=0$, while $x=0$ is the reflecting boundary, so $\frac{\partial}{\partial x} S(0,t)=0$.

First passage time is given by

\begin{equation}
\tau \, (x_0) = \int_{0}^{\infty} S(x_0,t) \,dt
\label{eqn:first_passage}
\end{equation}

Integrating Equation \ref{eqn:smoluchowski} in time over $(0,\infty)$ gives

\begin{equation}
A(x) \pard{\tau (x_0)}{x} + \frac{1}{2} B(x) \pardd{\tau (x_0)}{x} = -1
\label{eqn:first_passage}
\end{equation}

and the boundary conditions $\tau(0)=\tau(L)=0$, and $\pard{\tau}{x} \Bigr|_{x=0} =0$ .

Following Gardiner \citep{Gardiner1985}, the solution for $\tau (x_0)$ is written in terms of an integrating factor, $\psi$:

\begin{equation}
\psi \, (y) = \exp \left [ \int_0^y \frac{2A(x')}{B(x')} \,dx' \right ]
\label{eqn:psi}
\end{equation}

as

\begin{equation}
\tau \, (x_0) = 2 \int_{x_0}^{L} \left ( \frac{1}{\psi(y)} \int_{0}^{y} \frac{\psi(z)}{B(z)} \,dz \right ) \,dy
\label{eqn:first_passage_sol}
\end{equation}

In this case, for a drift-diffusion process,

\begin{equation}
\pard{S(x_0,t)}{t} = \frac{F(x)}{\gamma} \pard{S(x_0,t)}{x} + D \pardd{S(x_0,t)}{x}
\label{eqn:smoluchowski_final}
\end{equation}

so $B(x) = 2D$, and $A(x) = \frac{F(x)}{\gamma}$, and by the Einstein relation, $\gamma = \frac{k_B T}{D}$, so $A(x) = \frac{D}{k_B T} F(x)$.  Therefore,

\begin{equation}
\psi \, (y) = \exp \left [ \int_0^y  \frac{F(x')}{k_B T} \,dx' \right ]
\label{eqn:psi2}
\end{equation}

and also

\begin{equation}
\tau \, (x_0) = \frac{1}{D} \int_{x_0}^{L} \left ( \frac{1}{\psi(y)} \int_{0}^{y} \psi(z) \,dz \right ) \,dy
\label{eqn:first_passage_sol}
\end{equation}

Given a force profile, $F(x)$, this equation can be integrated numerically to solve for the first passage time as a function of initial location.  Given a probability distribution of starting locations, $w(x_0)$, the mean first passage time for escape can be computed as

\begin{equation}
\langle \tau \, \rangle = \int_{0}^{L} \tau(x_0) \, w(x_0) \,dx_0
\label{eqn:mean_first_passage_sol}
\end{equation}