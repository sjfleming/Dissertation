\begin{savequote}[75mm]
All the physical and chemical laws that are known to play an important part in the life of organisms are of [a] statistical kind; any other kind of lawfulness and orderliness that one might think of is being perpetually disturbed and made inoperative by the unceasing heat motion of the atoms.
\qauthor{Erwin Schr\"odinger, \textit{What Is Life?}}
\end{savequote}

\chapter{Controlling the movement of a helicase on DNA}
\label{helicase_motion_control}

The general goal and its motivation (talk about homopolymers here).

The broader question about mechanical force in biology, and using mechanical force to get an enzyme to perform its function.

\section{Experiment}

\section{Mechanical force as a sufficient substitute for ATP}

Sliding mode, agreement of current levels.  All sorts of results about sliding distributions, average rates with voltage and temperature.

\begin{figure}[h]
\begin{centering}
\includegraphics[width=0.9\textwidth]{figures/helicase_force_stepping.pdf}
\caption[Mechanical force substitutes for ATP and moves helicase]{...}
\label{fig:helicase_force_stepping}
\end{centering}
\end{figure}

\begin{figure}[h]
\begin{centering}
\includegraphics[width=\textwidth]{figures/helicase_levels_sliding.pdf}
\caption[Mechanical force results in the same current levels]{... several molecules aligned ... M3-MspA, \SI{1}{\Molar} KCl, \SI{2}{\m\Molar} ADPNP, \SI{37}{\celsius}, \SI{160}{\mV}.}
\label{fig:helicase_force_stepping_2}
\end{centering}
\end{figure}

\subsection{Effect of a non-hydrolyzable ATP analog}

\begin{figure}[h]
\begin{centering}
\includegraphics[width=0.8\textwidth]{figures/helicase_sliding_analog.pdf}
\caption[Unzipping time due to force depends on ATP analog]{... several molecules aligned ... M3-MspA, \SI{1}{\Molar} KCl,, \SI{37}{\celsius}.}
\label{fig:helicase_sliding_analog}
\end{centering}
\end{figure}

\begin{figure}[h]
\begin{centering}
\includegraphics[width=0.5\textwidth]{figures/helicase_sliding_step_dist.pdf}
\caption[Distribution of step durations using mechanical force]{... several molecules aligned ... M3-MspA, \SI{1}{\Molar} KCl, \SI{2}{\m\Molar} ADPNP, \SI{37}{\celsius}.}
\label{fig:helicase_sliding_time_dist}
\end{centering}
\end{figure}

\section{The effects of voltage and temperature}

\begin{figure}[h]
\begin{centering}
\includegraphics[width=0.7\textwidth]{figures/helicase_sliding_voltage.pdf}
\caption[Forced helicase stepping depends on voltage]{...  M3-MspA, \SI{1}{\Molar} KCl, \SI{2}{\m\Molar} ADPNP.}
\label{fig:helicase_stepping_voltage}
\end{centering}
\end{figure}

\begin{figure}[h]
\begin{centering}
\includegraphics[width=0.7\textwidth]{figures/helicase_sliding_temp.pdf}
\caption[Forced helicase stepping depends on temperature]{...  M3-MspA, \SI{1}{\Molar} KCl, \SI{2}{\m\Molar} ADPNP.}
\label{fig:helicase_stepping_temp}
\end{centering}
\end{figure}


\section{Attempts at temporal control using voltage pulses}

\subsection{The physical picture}
The idea and a description of the physics.

Limits in terms of membrane charging.

\subsection{Experimental setup and data}

\begin{figure}[h]
\begin{centering}
\includegraphics[width=0.9\textwidth]{figures/pulse_electronics.pdf}
\caption[Electronic setup for low-noise pulses across a nanopore]{...}
\label{fig:helicase_pulse_setup}
\end{centering}
\end{figure}

Experiments and data

\begin{figure}[h]
\begin{centering}
\includegraphics[width=\textwidth]{figures/helicase_pulse_level_change.pdf}
\caption[Voltage pulse can induce a helicase step]{M3-MspA, applied constant bias voltage of \SI{160}{\mV} in \SI{1}{\Molar} KCl, \SI{2}{\m\Molar} ADPNP, at \SI{37}{\celsius}. Data filtered to \SI{200}{\Hz}}
\label{fig:helicase_pulse}
\end{centering}
\end{figure}

\begin{figure}[h]
\begin{centering}
\includegraphics[width=\textwidth]{figures/helicase_data_pulses.pdf}
\caption[A fraction of pulses move the helicase]{M3-MspA, applied constant bias voltage of \SI{160}{\mV} in \SI{1}{\Molar} KCl, \SI{2}{\m\Molar} ADPNP, at \SI{37}{\celsius}.  Gray is raw data, hardware filtered at \SI{10}{\kHz}, black is \SI{1}{\kHz} filtered data.}
\label{fig:helicase_pulse_traces}
\end{centering}
\end{figure}

\subsection{Analysis and discussion}

discussion.

\begin{figure}[h]
\begin{centering}
\includegraphics[width=0.7\textwidth]{figures/helicase_pulse_correlation.pdf}
\caption[Quantifying efficacy of pulses]{M3-MspA, applied constant bias voltage of \SI{160}{\mV} in \SI{1}{\Molar} KCl, \SI{2}{\m\Molar} ADPNP, at \SI{37}{\celsius}.}
\label{fig:helicase_pulse_efficacy}
\end{centering}
\end{figure}

Many pulse voltages and durations were tried, from 200mV to 2V, and 200ns to 2ms, in both 1M KCl and 0.5M KCl, with and without ATP and ADPNP.

What really remains is to try other enzymes.
