\begin{savequote}[75mm]
All the physical and chemical laws that are known to play an important part in the life of organisms are of [a] statistical kind; any other kind of lawfulness and orderliness that one might think of is being perpetually disturbed and made inoperative by the unceasing heat motion of the atoms.
\qauthor{Erwin Schr\"odinger, \textit{What Is Life?}}
\end{savequote}

\chapter{Controlling the movement of a helicase on DNA}
\label{helicase_motion_control}

\section{Background}

\subsection{Motivation for controlling helicase motion}

One of the main factors that limits the accuracy of a DNA sequence obtained from a single read in a nanopore is the presence of homopolymer regions, or repeated nucleotides, in the DNA.  Repeated nucleotides in stretches longer than the five bases are common in biology \citep{Dechering1998}, but the constriction of MspA reads only about \num{5} nucleotides simultaneously.  This means that as the DNA is unwound by a helicase or other enzyme and fed through the pore base-by-base, the \num{5} nucleotides in the pore remain the same, and the current does not change.  Thus the nanopore misses information about stretches of homopolymers longer than \num{5} nucleotides.

Reads from several identical copies of a DNA molecule can be obtained and the data combined using a variety of algorithms (for a discussion see Reference \citenum{Szalay2015}).  Combining information from \num{100} molecules can yield approximately \num{99}\% accuracy in the DNA sequence.  However, the information about the length of long homopolymer repeats is missing, and this missing information accounts for approximately half of the remaining error \citep{Szalay2015}.  As data quality improve and as the data analysis increasingly takes place using recurrent neural networks \citep{Boza2017}, it is possible that the major source of error that will remain is these long stretches of homopolymers.

If the length of these homopolymer regions of DNA could be determined, it would boost the overall accuracy of DNA sequences obtained with a nanopore.  One strategy is to use the time information obtained from nanopore current recordings.  Since current levels last different amounts of time, it is possible to make inferences about repeats \citep{Sarkozy2017}.  Yet the most desirable solution would be to determine the number of repeats experimentally.  The strategy of the work presented in this chapter is to develop such an approach, based on exerting temporal control over the motion of the helicase enzyme.  If the motion of the enzyme could be precisely timed, and the enzyme made to step at regular intervals, then the number of nucleotides in a homopolymer repeat could be read directly.

\subsection{General considerations about mechanical force in biology}

Much work has been carried out in studying the role of mechanical force in the functioning of motor proteins, ever since the development of single-molecule experimental techniques such as optical tweezers, magnetic tweezers, and atomic force microscopy \citep{Neuman2008}.  The nanopore-DNA experimental setup described here has also been proposed as a sensitive instrument capable of measuring small forces and displacements \citep{Derrington2015}.

Mechanical forces are important in biochemistry, particularly in the functioning of motor proteins.  For an overview, see References \citenum{Bustamante2004} and \citep{Oster2003}.  In general, motor proteins convert chemical energy into mechanical work.  Motor proteins, including helicases, polymerases, and also mysosins (involved in muscle contraction) and kinesins (involved in intracellular cargo trafficking) walk along directional tracks, hydrolyzing nucleotide triphosphates in order to do so.

% Mechanical force is also important in force-sensitive ion channels such as those involved in mechanotransduction, \textit{i.e.} biochemical signaling within a cell in response to mechanical forces \citep{Jaalouk2009}.

An optical tweezer setup was used to measure the force applied by an \textit{E. coli} RNA polymerase during trascription in 1995 (approximately \SI{14}{\pico\N}) \citep{Yin1995}.  Myosin-V was found to act as a mechanical ratchet, rectifying mechanical forces applied to it \citep{Gebhardt2006}.  External forces were applied in the forward direction, assisting the enzyme's motion along actin, as well as in the reverse direction, using an optical tweezer.  The rate of stepping by myosin-V was found to depend on this applied force \citep{Clemen2005}.  Kinesin is capable of producing forces of about \SI{7}{\pico\N} \citep{Higuchi1997}.  Studies on kinesin have demonstrated that force-induced backward steps of kinesin might in fact synthesize ATP, in the reverse reaction of ATP hydrolysis involved in forward stepping \citep{Hyeon2009}.  Synthesis of ATP was demonstrated by Itoh \textit{et al.} by applying a torque to rotate the F$_1$-ATPase in the reverse of its ATP-hydrolysis direction \citep{Itoh2004}.

\subsection{This work}

Mechanical force has been shown by Cherf \textit{et al.} to mediate unzipping of a dsDNA duplex by phi29 DNA polymerase in $\alpha$-hemolysin nanopore experiments \citep{Cherf2012}.  Here, this same idea of force-mediated unzipping of a duplex is explored, using the Dda helicase enzyme.  The strategy is to advance the helicase along DNA using applied mechanical force.  Experiments demonstrate that mechanical force can be used as a substitute for the hydrolysis of ATP in order to force the enzyme to take single-nucleotide steps.  Measurements are made to quantify helicase step times as a function of voltage and temperature.  Attempts are made at temporal control over helicase stepping using brief pulses of applied voltage.


\section{Experiment}

The experimental setup is the same as that shown in Figure \ref{fig:helicase_scheme}b, with M3-MspA in a PC lipid membrane.  A Dda helicase - DNA complex is captured in the nanopore, and a constant voltage bias is applied.  The buffer is typically \SI{1}{\Molar} KCl with \SI{25}{\m\Molar} K-phosphate buffer at pH \num{8.00}.  \SI{2}{\m\Molar} MgCl$_2$ is added to the cis side with the helicase, but no ATP is added.  The only force applied to advance the helicase along the DNA is the electrophoretic force on the DNA induced by the voltage across the membrane.

\section{Mechanical force as a sufficient substitute for ATP}

Figure \ref{fig:helicase_force_stepping}a shows a short segment of current versus time data for a single DNA molecule, taken as Dda helicase is forced to step along the DNA due to a constant \SI{160}{\mV} applied bias at \SI{37}{\celsius}, unwinding dsDNA duplex ahead of it.  The DNA used in this experiment is sequence \# 0002, along with its complement and tether, listed in Appendix \ref{sequences}.  The same sort of discrete steps in the current are observed as in the case where helicase is driven by ATP (see Figure \ref{fig:helicase_data}, for comparison).  The current level means can be aligned to known model levels for the same sequence obtained while running the helicase forward using ATP, when it steps one base at a time.  The comparison is shown in Figure \ref{fig:helicase_force_stepping}b, and this demonstrates that the helicase is still stepping base-by-base along the DNA.  The DNA sequence used here was intentionally designed to produce a large current variation, which is achieved by transitioning from \texttt{TTTTT}, one of the lowest current levels, to \texttt{GGAAA}, the highest current level.

\begin{figure}[H]
\begin{centering}
\includegraphics[width=0.9\textwidth]{figures/helicase_force_stepping.pdf}
\caption[Mechanical force substitutes for ATP and moves helicase]{(a) Selected segment of a current versus time trace obtained at \SI{37}{\celsius} as a constant \SI{160}{\mV} bias is applied across the membrane.  The Dda helicase is apparently forced to move forward along the DNA strand, unzipping the dsDNA duplex ahead of it.  Note the relatively long time scale on the x-axis.  Gray is the raw data obtained at \SI{10}{\kHz}, while blue is filtered to \SI{1}{\kHz}.  (b) The current level means, with the time information removed, showing the same data.  Current levels measured when the helicase advances due to applied force (red) are aligned to known model levels, measured with ATP (black).}
\label{fig:helicase_force_stepping}
\end{centering}
\end{figure}

The data from several molecules can be aligned to current levels from a known model, as is shown in Figure \ref{fig:helicase_force_stepping_2}.  Shown are the first \num{64} levels from a longer sequence made by ligating Appendix \ref{sequences} \# 0002 and its complement to \# 0003\_ hairpin.  Measured current levels due to forced unzipping (red) align well with model levels (black), again indicating that the helicase is taking single-base steps.

\begin{figure}[h]
\begin{centering}
\includegraphics[width=\textwidth]{figures/helicase_levels_sliding.pdf}
\caption[Mechanical force results in the same current levels]{Aligned data from several molecules of identical DNA sequence, showing the current levels measured in M3-MspA (red) as helicase steps along the strand.  The helicase is advanced by the mechanical force applied due to a \SI{160}{\mV} transmembrane potential difference.  Current levels are aligned to known model levels from measurements using ATP.  Error bars are the standard deviation of the mean currents of levels from each molecule measured.  \SI{1}{\Molar} KCl, \SI{2}{\m\Molar} ADPNP, \SI{2}{\m\Molar} MgCl$_2$, \SI{37}{\celsius}.}
\label{fig:helicase_force_stepping_2}
\end{centering}
\end{figure}

\subsection{Effect of a non-hydrolyzable ATP analog}

Experiments were carried out to ascertain whether or not the presence of a non-hydrolyzable ATP analog made any difference in the stepping behavior of the helicase.  Since helicase has an ATP binding pocket, and undergoes a conformational change upon ATP binding, it is natural to assume that the change in conformation might cause a change in the stepping behavior under applied force.  The ATP analog used in these experiments is adenylyl-imidodiphosphate [ADPNP] (alternatively abbreviated AMP-PNP).  It is similar to ATP in all but the final phosphate, which is linked to the second phosphate via a nitrogen instead of an oxygen, and so it cannot be hydrolyzed.

Figure \ref{fig:helicase_sliding_analog} shows histograms of the times it takes a full DNA molecule to be traversed by Dda helicase using force instead of ATP.  The DNA used here is Appendix \ref{sequences} \# 0002 with complement and its tether, having a dsDNA duplex region of \num{50} base pairs in front of the helicase.  Panel (a) compares the effects of \SI{160}{\mV} applied bias to \SI{180}{\mV} applied bias.  It cam be seen that the mean unzipping time is shorter at the higher bias, but both distributions are quite broad.  A log-normal fit is plotted as a dashed line to guide the eye.

Figure \ref{fig:helicase_sliding_analog}b shows the same histograms when \SI{2}{\m\Molar} ADPNP is added.  The distributions tighten up somewhat compared to no ADPNP; however, they are still quite broad, spanning a range of over \SI{100}{\s} in the time it takes to unzip \num{50} base pairs.  Because the presence of ADPNP seems to result in a tighter distribution of transit times, \SI{2}{\m\Molar} ADPNP is included in the experiments presented in the rest of this work.

\begin{figure}[H]
\begin{centering}
\includegraphics[width=0.8\textwidth]{figures/helicase_sliding_analog.pdf}
\caption[Forced unzipping time depends on the presence of an ATP analog]{(a) Without ATP analog, the distribution of the time it takes to force a helicase to unzip a \num{50}bp duplex has a long tail.  The average of the \SI{180}{\mV} distribution (red) is a shorter time than the \SI{160}{\mV} distribution (blue), but both are quite broad.  (b) With \SI{2}{\m\Molar} ADPNP present, the same distributions become slightly less broad.  Data taken with M3-MspA, \SI{1}{\Molar} KCl, \SI{25}{\m\Molar} K-phosphate, \SI{2}{\m\Molar} MgCl$_2$, pH \num{8.00}, \SI{37}{\celsius}.  Log-normal fits are shown as dotted lines to guide the eye.}
\label{fig:helicase_sliding_analog}
\end{centering}
\end{figure}

Due to the single-base resolution obtained from the current versus time traces recorded in these nanopore experiments, it is possible to look at the durations of individual steps of the helicase.  Shown in Figure \ref{fig:helicase_sliding_time_dist} is the probability of a step lasting for a given duration, plotted at three applied bias voltages: \SI{120}{\mV}, \SI{160}{\mV}, and \SI{180}{\mV}.  Note that the data are plotted on a log-log scale.  The differences between the distributions at the different biases are enough to shift the mean lower at higher applied bias.

\begin{figure}[h]
\begin{centering}
\includegraphics[width=0.5\textwidth]{figures/helicase_sliding_step_dist.pdf}
\caption[Distribution of step durations using mechanical force]{Probability distributions for helicase step durations measured without ATP, when Dda is advanced by mechanical force through a duplex.  The three voltages shown all display quite broad distributions, but step durations are generally shorter at higher applied bias voltages.  M3-MspA, \SI{1}{\Molar} KCl, \SI{25}{\m\Molar} K-phosphate, \SI{2}{\m\Molar} MgCl$_2$, pH \num{8.00}, \SI{2}{\m\Molar} ADPNP, \SI{37}{\celsius}.}
\label{fig:helicase_sliding_time_dist}
\end{centering}
\end{figure}

\section{The effects of voltage and temperature}

In an effort to control the motion of the helicase enzyme on DNA, it is important to understand the dependence of the stepping rate on applied voltage and temperature.  The electrophoretic force applied to the ssDNA is directly proportional to the applied voltage, according to $F = \sigma_{\text{eff}} V$.  The quantity $\sigma_{\text{eff}}$ for ssDNA in the MspA nanopore was measured in Chapter \ref{dna_thermal_motion_mspa}, and its average value (over the four nucleotides) can be expressed as \SI{0.177}{\pico\N / \mV} (see Table \ref{table:fit_values}).  This means that the applied force at a bias of \SI{160}{\mV} is approximately \SI{28}{\pico\N}, and that applied biases of \SI{120}{\mV} and \SI{200}{\mV} differ in applied force by \SI{14}{\pico\N}.

Figure \ref{fig:helicase_stepping_voltage} shows the measured step durations as a function of applied constant voltage at various temperatures.  Step durations are characterized by median step duration.  The distribution of the step durations that gives rise to each data point is quite broad, as mentioned in connection with Figure \ref{fig:helicase_sliding_time_dist}.  The error bars correspond only to $median(t)/\sqrt{N}$, and do not reflect the width of the underlying distributions.  In enzymology, this sort of data is typically plotted as a ``force - velocity curve" \citep{Bustamante2004}.

\begin{figure}[h]
\begin{centering}
\includegraphics[width=0.7\textwidth]{figures/helicase_sliding_voltage.pdf}
\caption[Forced helicase stepping depends on voltage]{The median step duration of Dda helicase forced to unwind a duplex is plotted as a function of voltage, at several temperatures.  Spline curves are drawn to guide the eye along measurements at constant temperature.  Error bars are $median(t)/\sqrt{N}$, where $N$ was typically many hundreds of individual steps.  M3-MspA, \SI{1}{\Molar} KCl, \SI{25}{\m\Molar} K-phosphate, \SI{2}{\m\Molar} MgCl$_2$, pH \num{8.00}, \SI{2}{\m\Molar} ADPNP.  The voltage biases on the x-axis span the range of approximately \SI{21}{\pico\N} to \SI{35}{\pico\N}.}
\label{fig:helicase_stepping_voltage}
\end{centering}
\end{figure}

The same data are re-plotted in Figure \ref{fig:helicase_stepping_temp}, this time as an explicit function of temperature.  It is clear from this plot that temperature has a very large impact on the stepping of the helicase, even changes as small as \SI{10}{\celsius} or less.  At some point, the temperature will become too high for the helicase to function, and the protein could even denature.  \SI{50}{\celsius} was the highest temperature studied here, and the helicase still seemed to function as normal.  Higher temperatures were not easily accessible with this experimental setup due to the formation of bubbles which would often destroy the fragile lipid membrane.

\begin{figure}[h]
\begin{centering}
\includegraphics[width=0.7\textwidth]{figures/helicase_sliding_temp.pdf}
\caption[Forced helicase stepping depends on temperature]{Same data shown in Figure \ref{fig:helicase_stepping_voltage}.  The median step duration of Dda helicase forced to unwind a duplex is plotted here as a function of temperature, at several applied bias voltages.  Spline curves are drawn to guide the eye along measurements at constant voltage.  Error bars are $median(t)/\sqrt{N}$, where $N$ was typically many hundreds of individual steps.  M3-MspA, \SI{1}{\Molar} KCl, \SI{25}{\m\Molar} K-phosphate, \SI{2}{\m\Molar} MgCl$_2$, pH \num{8.00}, \SI{2}{\m\Molar} ADPNP.}
\label{fig:helicase_stepping_temp}
\end{centering}
\end{figure}


\section{Helicase on dsDNA and ssDNA}

Experiments were also carried out to assess the differences in helicase movement when it is pushed against dsDNA versus ssDNA.  This can be achieved using a DNA hairpin construction, as shown in Figure \ref{fig:helicase_stepping_full_ss_ds}.  The DNA sequence is provided in Appendix \ref{sequences} as \# 0002 with its complement and tether, ligated to \# 0003\_ hairpin.

\begin{figure}[h]
\begin{centering}
\includegraphics[width=\textwidth]{figures/helicase_sliding_trace_full.pdf}
\caption[Forced helicase stepping depends on unzipping]{The two diagrams show the sequential unzipping of a DNA hairpin by a helicase.  At first, the helicase is up against a dsDNA duplex, but after the entire duplex is unzipped, the helicase is on ssDNA.  Data from a single molecule are plotted below.  The red arrow denotes 18-carbon abasic spacers that are put in the sequence to mark the transition from dsDNA to ssDNA.  M3-MspA, \SI{1}{\Molar} KCl, \SI{25}{\m\Molar} K-phosphate, \SI{2}{\m\Molar} MgCl$_2$, pH \num{8.00}.}
\label{fig:helicase_stepping_full_ss_ds}
\end{centering}
\end{figure}

When the helicase is unzipping the duplex, it is pushing up against dsDNA, but after the full duplex has been unzipped, the helicase turns the hairpin loop and continues to move on ssDNA (see the two diagrams in Figure \ref{fig:helicase_stepping_full_ss_ds}).  The sequence \# 0003\_ hairpin (Appendix \ref{sequences}) contains three 18-carbon abasic spacers at a position where they will be in the constriction of MspA when the duplex has just been fully unzipped.  The large increase in current that accompanies these spacers provides a marker in the sequence to denote this transition, and is shown with a red arrow in the current trace in Figure \ref{fig:helicase_stepping_full_ss_ds}.  The current trace before and after this marker reflect traversal of the same number of bases.  It is clear from the current trace that the movement along ssDNA is much faster than along dsDNA.  Figure \ref{fig:helicase_stepping_ss_ds} shows two segments of current trace, each \num{6} seconds, from before and after this abasic marker.

\begin{figure}[h]
\begin{centering}
\includegraphics[width=0.85\textwidth]{figures/helicase_sliding_trace_ss_and_ds.pdf}
\caption[Forced helicase stepping on dsDNA versus ssDNA]{Current traces showing \SI{6}{\s} of movement of helicase along (a) dsDNA, and (b) ssDNA, from the same individual molecule.  Same experimental conditions as in Figure \ref{fig:helicase_stepping_full_ss_ds}.}
\label{fig:helicase_stepping_ss_ds}
\end{centering}
\end{figure}

The data in Figure \ref{fig:helicase_stepping_ss_ds} panel (a) reflects the movement of Dda helicase against dsDNA, while panel (b) shows data generated by Dda movement along ssDNA.  The difference in the rate of helicase stepping is clearly visible, with the helicase stepping much faster along ssDNA.  This data is taken without ATP, when the helicase is pushed along the strand by mechanical force using an applied bias of \SI{160}{\mV}.

A histogram of the stepping times in the two cases is plotted in Figure \ref{fig:helicase_stepping_hist_ss_ds}.  Here the fraction of the total number of measured current levels is plotted against their duration.  Both distributions appear to be roughly exponential but with a fat tail.  The distributions share a large amount of overlap, but the difference in time to traverse a long strand is pronounced (see for example, Figure \ref{fig:helicase_stepping_full_ss_ds}), mainly due to contributions from levels with very long durations in the dsDNA distribution, and the high probability of very short duration levels in the ssDNA distribution.

\begin{figure}[h]
\begin{centering}
\includegraphics[width=0.55\textwidth]{figures/helicase_sliding_durations_ss_ds.pdf}
\caption[Forced helicase step times on dsDNA and ssDNA]{Distribution of Dda helicase step times, \textit{i.e.} current level durations, on ssDNA and dsDNA.  Median level duration is \SI{0.1}{\s} on ssDNA and \SI{0.6}{\s} on dsDNA.  M3-MspA, \SI{1}{\Molar} KCl, \SI{25}{\m\Molar} K-phosphate, \SI{2}{\m\Molar} MgCl$_2$, pH \num{8.00}.}
\label{fig:helicase_stepping_hist_ss_ds}
\end{centering}
\end{figure}


\section{Attempts at temporal control using voltage pulses}

\subsection{The physical picture}

The movement of helicase along DNA is a stochastic process, with or without ATP present.  ATP of course facilitates the stepping of helicase, but stepping can still occur without ATP on much longer time scales when force is applied.  One picture consistent with these observations and the crystal structure of Dda and other similar helicases is that the Dda helicase has to wait for a large fluctuation in order to undergo the conformational change required in order to take a step.  The binding of ATP and its subsequent hydrolysis facilitate these conformational changes and increase their rates.  In the absence of ATP, the helicase has some low probability of undergoing a large enough fluctuation to undergo the necessary conformational change.  However, the temperature apparently increases the stepping rate, probably by increasing the size of fluctuations.  The applied force also apparently increases the stepping rate, to a lesser extent, possibly by increasing the frequency with which attempts are made at taking a step.  It is also possible that the helicase must wait for the final base pair of the dsDNA duplex to ``breathe" open \citep{Jose2009}, and that this could also limit the stepping rate.

In any event, the probability of helicase stepping increases at increased applied force.  The aim of the work in this section is to increase the applied force by a large amount, but for only a short duration.  The goal is to greatly increase the stepping probability while a short-duration force is applied, and thereby control when the helicase steps.

\subsection{Experimental setup}

This brief increase in electrophoretic force is applied using a low-noise voltage pulse generator (see Figure \ref{fig:helicase_pulse_setup}).  Due to the unacceptable amount of current noise introduced into the experiment by commercial pulse generators that plug into a wall socket, it was necessary to design and build a low-noise pulse generator for this purpose.  Details of the design are included in Appendix \ref{pulse_generator}.

\begin{figure}[h]
\begin{centering}
\includegraphics[width=0.9\textwidth]{figures/pulse_electronics.pdf}
\caption[Electronic setup for low-noise pulses across a nanopore]{Schematic of the electronic setup used to apply short voltage pulses across the nanopore.  The nanopore equivalent circuit is shown on the bottom left, with access resistance $R_{\text{a}}$, pore resistance $R_{\text{pore}}$, and membrane capacitance $C_{\text{mem}}$.  The voltage difference across the membrane is $\Delta V = V_{\text{bias}} + V_{\text{pulse}}$.  Voltage pulses of duration \SI{50}{\ns} to \SI{10}{\ms} and up to \SI{5}{\V} are applied using a custom-built low-noise pulse generator, inserted between the cis side of the experiment and the ground of the Axopatch's headstage.  In order to cancel out capacitive current spikes before they reach the current amplifier, the pulse generator simultaneously outputs a pulse of opposite polarity, meant to compensate.}
\label{fig:helicase_pulse_setup}
\end{centering}
\end{figure}

The pulse generator is battery powered and has its ground isolated from the ground of external control signals.  It can generate pulses of up to \SI{5}{\V} of variable duration between \SI{50}{\ns} and \SI{10}{\ms}.  It can be controlled by TTL signals, either from a manual push-button, or from automated signals sent by a digital output of the NI-DAQ USB 6003 (see Figure \ref{fig:setup_capture} for reference).  The digital output of the NI-DAQ USB 6003 is also digitized by the Digidata 1440A and recorded on the computer, as a third channel (in addition to measured current and applied voltage bias) that records information about the timing of pulses.

The connection of the pulse generator to the experimental setup is shown in Figure \ref{fig:helicase_pulse_setup}.  As shown, there is a simple circuit meant to cancel the transient capacitive current spikes that are caused by applying a voltage pulse across the membrane capacitance.  These current spikes would otherwise by amplified by the Axopatch and obscure a portion of the signal.  In the configuration shown in Figure \ref{fig:helicase_pulse_setup}, the variable resistor $R_{\text{comp}}$ and capacitor $C_{\text{comp}}$ are meant to match the access resistance $R_{\text{a}}$ and the membrane capacitance $C_{\text{mem}}$ in the nanopore setup.  The pulse generator applies a simultaneous pulse of opposite polarity across the compensation resistor and capacitor, and the resulting current spike is summed with its opposite polarity spike, from the nanopore, before the Axopatch headstage.  This setup works well, though compensating the capacitance exactly requires patience, as the capacitances involved are quite small (on the order of \SI{1}{\pico\F}).

One further consideration is that of limitations on the speed of voltage pulses.  Pulses arrive at the membrane after traveling through the access resistance and charging the membrane capacitance.  However, here the access resistance of relevance here is the access resistance to the entire membrane, not to the nanopore.  In this case, $R_{\text{a}} \approx $ \SI{10}{\kilo\ohm}, and $\tau = R_{\text{a}} C_{\text{mem}} \approx $ \SI{10}{\ns}.  In this case, pulses on the order of \SI{10}{\ns} or longer would be feasible.

\subsection{Data obtained using voltage pulses}

For an idea of what the intended results look like, refer to Figure \ref{fig:helicase_pulse}.  Two separate instances are plotted where a \SI{300}{\mV} pulse is applied, and the pulse causes the current level to change, implying that the helicase has taken a step.

\begin{figure}[h]
\begin{centering}
\includegraphics[width=\textwidth]{figures/helicase_pulse_level_change.pdf}
\caption[Voltage pulse can induce a helicase step]{Plots of current and voltage as a brief \SI{300}{\mV} pulse is applied to a Dda - DNA complex in  M3-MspA.  (a-b) Helicase takes a step, and the current level changes, as a pulse is applied.  There is a constant bias voltage of \SI{160}{\mV} in \SI{1}{\Molar} KCl, \SI{25}{\m\Molar} K-phosphate buffer, pH \num{8.00}, \SI{2}{\m\Molar} MgCl$_2$ \SI{2}{\m\Molar} ADPNP, at \SI{37}{\celsius}.  Current data are filtered to \SI{200}{\Hz}.  These two plots were selected from the data, but they are relatively rare.}
\label{fig:helicase_pulse}
\end{centering}
\end{figure}

Panel (a) shows a pulse of \SI{0.5}{\ms} duration, while the pulse in panel (b) is \SI{1}{\ms} in duration.  Current (filtered to \SI{200}{\Hz}) is plotted in black.  (It is worth pointing out the difference between adequate current spike compensation in panel (a), and inadequate compensation in panel (b), where the current spike is visible in the data.)  The red line shows the voltage.  There is a constant voltage of \SI{160}{\mV} applied the entire time, to hold the helicase-DNA complex in the pore and to measure the ionic current.  At \SI{0.1}{\s} on the time axis, a brief pulse of \SI{300}{\mV} is applied, which adds on top of the constant voltage.  These two instances actually represent rare occurrences, and were carefully selected from the data to demonstrate the desired scenario.

A slightly more accurate idea of what a full dataset looks like is depicted in Figure \ref{fig:helicase_pulse_traces}, although these segments of data are still selected to show some current level changes that coincide with voltage pulses.  The vertical red lines indicate the timing of voltage pulses, and the green arrows indicate current level changes that coincide with voltage pulses.  Other current level changes are due to the constant applied bias of \SI{160}{\mV}.

\begin{figure}[h]
\begin{centering}
\includegraphics[width=\textwidth]{figures/helicase_data_pulses.pdf}
\caption[A fraction of pulses move the helicase]{Some helicase steps coincide with voltage pulses, but others do not.  Gray is raw current data, hardware filtered at \SI{10}{\kHz}, black is \SI{1}{\kHz} filtered data.  Red vertical lines indicate the timing of pulses, approximately \SI{104}{\ms} apart, \SI{1}{\ms} in duration, and \SI{300}{\mV} in magnitude, in addition to a constant bias voltage of \SI{160}{\mV}.  Green arrows indicate pulses that coincide with changes in current level. M3-MspA in \SI{1}{\Molar} KCl, \SI{25}{\m\Molar} K-phosphate buffer, pH \num{8.00}, \SI{2}{\m\Molar} MgCl$_2$ \SI{2}{\m\Molar} ADPNP, at \SI{37}{\celsius}.}
\label{fig:helicase_pulse_traces}
\end{centering}
\end{figure}

\section{Analysis and discussion}

\subsection{Parameter space}

This experiment has a lot of parameter space, including holding voltages, pulse voltages, pulse durations, temperatures, salt concentrations, and buffer conditions.  Many sets of parameters have been tested, though experiments have not yet completely explored the parameter space.  Pulse voltages from \SI{200}{\mV} to \SI{2}{\V} have been tried, as well as pulse durations from \SI{200}{\ns} to \SI{2}{\ms}.  Both \SI{1}{\Molar} KCl and \SI{0.5}{\Molar} KCl have been tested in some subset of conditions, with and without ADPNP or a low concentration of ATP.  These experiments have indicated that pulses of duration less than about \SI{100}{\micro\s} seem to be largely ineffective, at least at the voltages used.

From a practical standpoint, with a PC lipid bilayer membrane, the tolerable voltage range is limited.  A PC lipid membrane will usually break at sustained voltages above about \SI{300}{\mV}.  The membrane will certainly break at \SI{400}{\mV}.  However, it is possible to exceed these voltage tolerances for a brief duration.  For example, using \SI{0.5}{\ms} pulses where the total applied voltage reaches \SI{460}{\mV} is perfectly tolerable.  However, as the duration gets pushed to about \SI{2}{\ms}, the membrane can become unstable and break, ending the experiment.

\subsection{Indications of limited success with Dda}

A more quantitative look at the outcome of experiments using voltage pulses to advance Dda helicase through a dsDNA duplex is shown in Figure \ref{fig:helicase_pulse_efficacy}.  The data show that, even under the most favorable conditions tested here, which were \SI{1}{\ms} pulses of \num{300} - \SI{400}{\mV}, less than \num{1}\% of the applied pulses were effective.

\begin{figure}[h]
\begin{centering}
\includegraphics[width=0.75\textwidth]{figures/helicase_pulse_efficacy.pdf}
\caption[Quantifying efficacy of pulses]{Under the most favorable conditions tested, only about 1\% of voltage pulses are effective in advancing the helicase.  Error bars are the standard error of the mean for several individual DNA-helicase complexes.  The black dashed line shows the mean value corresponding to random coincidence of a pulse with a level change, as measured in the \SI{0}{\ms} control experiment.  M3-MspA nanopore, with applied constant bias voltage of \SI{160}{\mV} in \SI{1}{\Molar} KCl, \SI{25}{\m\Molar} K-phosphate, pH \num{8.00}, \SI{2}{\m\Molar} MgCl$_2$, \SI{2}{\m\Molar} ADPNP, at \SI{37}{\celsius}.}
\label{fig:helicase_pulse_efficacy}
\end{centering}
\end{figure}

``Effective" is defined here as the voltage pulse coinciding with a current level change.  In order to account for the probability of this occurring by random chance, a control experiment was carried out under the same conditions, and is shown in Figure \ref{fig:helicase_pulse_efficacy} as the dashed black line.

\subsection{Future directions}

Further experiments could be carried out to explore more of the parameter space, including lower salt conditions and increased temperatures.  Experiments could also be carried out using other membrane materials with higher voltage tolerance, such as lipids like mycolic acid \citep{Langford2011}, or triblock copolymer membranes \citep{Nardin2000a, Gonzalez-Perez2009}.  This would allow larger amplitude voltage pulses to be explored.

A likely route to success in controlling the motion of DNA through the nanopore using voltage pulses could be to try other DNA enzymes.  There is no reason \textit{a priori} that Dda helicase should work better than any other enzyme.  Work has already shown that phi29 DNA polymerase undergoes voltage-mediated unzipping \citep{Cherf2012}, and so it is possible that these same experiments could be attempted using phi29 DNA polymerase.  Other processive DNA enzymes could be tried as well.  It seems that it takes quite a bit of force to push Dda through a DNA duplex.  There may be enzymes for which this force threshold is lower, and these may respond more readily to the strategy outlined in this chapter.
