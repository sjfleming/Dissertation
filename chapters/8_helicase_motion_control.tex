\begin{savequote}[75mm]
All the physical and chemical laws that are known to play an important part in the life of organisms are of [a] statistical kind; any other kind of lawfulness and orderliness that one might think of is being perpetually disturbed and made inoperative by the unceasing heat motion of the atoms.
\qauthor{Erwin Schr\"odinger, \textit{What Is Life?}}
\end{savequote}

\chapter{Controlling the movement of a helicase on DNA}
\label{helicase_motion_control}

The general goal and its motivation (talk about homopolymers here).

The broader question about mechanical force in biology, and using mechanical force to get an enzyme to perform its function.

\section{Experiment}

\section{Mechanical force as a sufficient substitute for ATP}

Sliding mode, agreement of current levels.  All sorts of results about sliding distributions, average rates with voltage and temperature.

\begin{figure}[h]
\begin{centering}
\includegraphics[width=0.9\textwidth]{figures/helicase_force_stepping.pdf}
\caption[Mechanical force substitutes for ATP and moves helicase]{...}
\label{fig:helicase_force_stepping}
\end{centering}
\end{figure}

\section{The effects of voltage and temperature}

\subsection{Helicase stepping at various applied voltages}



\subsection{Helicase stepping at various temperatures}



\subsection{Effect of a non-hydrolyzable ATP analog}



\section{Attempts at temporal control using voltage pulses}

\subsection{The physical picture}
The idea and a description of the physics.

Limits in terms of membrane charging.

\subsection{Experimental setup and data}

Experiments and data

\subsection{Analysis and discussion}

discussion.