\chapter{Data Analysis}
\label{data_analysis}

\section{General}

Data are recorded using the ClampEx 10.3 software provided by the Molecular Devices Corporation, maker of the Axopatch 200B patch clamp current amplifier.  Analog measurements from the Axopatch 200B are digitized using the Axon CNS Digidata 1440A, a \num{16}-bit analog-to-digital converter capable of acquisition at \SI{250}{\kHz}.  This process results in a \texttt{.ABF} file, the proprietary Axon Binary File format.  In general, for the purposes of further analysis, the data are imported into Matlab (The MathWorks, Inc.) using a suite of files, several of which are part of the PoreView package written by Tamas Szalay, and ultimately these make use of the Matlab file \texttt{ABFLOAD.m}, written by Harald Hentschke and Forrest Collman.

PoreView, run in Matlab, is capable of displaying arbitrarily large data files using downsampling and caching.  Most of the rest of the data analysis was written in Matlab, using methods available in the PoreView package to load data as needed.


\section{Event-finding and analysis}

An ``event" in this context refers to a contiguous segment of data where the measured current has dropped below the open pore current.  This analysis is performed using the class \texttt{analysis.m}, which utilizes parts of PoreView for data handling, and has many methods which enable the user to easily perform common tasks, such as creating current histograms, finding events, plotting scatter plots, plotting individual events, and finding discrete current levels within an event.  Event data are aggregated, along with statistics about each event, and are saved in a \texttt{.MAT} file for future analysis.

Event-finding is achieved by means of setting a threshold for the type of blockages that are considered events, finding the open pore current by identifying the center of the highest-current gaussian peak in the current histogram of the entire file (that totals at least \num{5}\% of the data), filtering the data and identifying rough regions in a downsampled version of the data where the current is below the threshold, identifying contiguous such segments, and then taking a detailed look at the raw data to identify the beginning and end of each event exactly.  This use of downsampled data to identify candidate events speeds up the algorithm considerably.  After each event is found, a variety of event statistics are computed.


\section{Level-finding algorithm}
\label{level_finding}

\begin{figure}[H]
\begin{centering}
\includegraphics[width=0.95\textwidth]{figures/data_analysis_level_finding.pdf}
\caption[Data analysis: level-finding]{(a) Current versus time data recorded for a helicase stepping along DNA held in the MspA nanopore.  Stable current levels are marked by abrupt transitions, where the helicase slides one base forward (or backward).  Raw data are hardware filtered at 10kHz (gray).  Data filtered at 1kHz in software are shown in black.  (b) After application of a level-finding algorithm, the discrete mean current levels are shown in red.}
\label{fig:data_analysis_levels}
\end{centering}
\end{figure}

In an MspA experiment where a helicase is used to ratchet DNA through the nanopore one base at a time, recordings of current as a function of time exhibit discrete steps between stable levels (see Figure \ref{fig:data_analysis_levels}a).  Identifying steps in time-series data, and stable levels, is sometimes referred to as ``change-point detection."  I have tested several methods, including hypothesis testing (the null hypothesis being that two segments of data are drawn from the same underlying distribution).  Hypothesis testing using a p-value threshold with the Student t-distribution produces reasonable results, but is not excellent.  Testing using the Kolmogorov-Smirnov test does slightly better, since it does not assume the distributions are gaussian, however the computational intensity makes this approach unfeasible.

A better algorithm has been developed by Schreiber and Karplus \citep{Schreiber2015}, and was used in other work on nanopore data \citep{Laszlo2014}.  This algorithm looks at a segment of data (Figure \ref{fig:data_analysis_level_alg}a), and moves the location of an index within that segment.  It computes the variance of the entire segment, as well as of the two segments that result when the data is divided into two at the given index.  The algorithm finds the index which minimizes the ratio of within-segment variance to between-segment variance (the ``metric" in Figure \ref{fig:data_analysis_level_alg}b).  If this ratio is below a specified threshold (which is set by probabilities calculated using a prior), then the segment is divided into two at that index (Figure \ref{fig:data_analysis_level_alg}c).  This approach can be run on the data recursively, starting with the entire event, and splitting it into two (or not) over and over until the threshold for splitting is no longer met.  The computation of variance can also be sped up considerably by pre-computing cumulative sums of $I$ and $I^2$.  The results of the algorithm are shown in Figure \ref{fig:data_analysis_levels}b.

\begin{figure}[H]
\begin{centering}
\includegraphics[width=0.6\textwidth]{figures/data_analysis_level_finding_algorithm.pdf}
\caption[Data analysis: level-finding algorithm]{(a) Segment of nanopore current versus time data.  (b) The metric calculated to determine a level change, as defined by Schreiber and Karplus \citep{Schreiber2015}.  This metric has a minimum at a level transition.  The extreme cusp is characteristic of this particular metric, and is extremely advantageous.  (c) The current versus time data, with level 1 in red and level 2 in blue.  This procedure can be performed recursively on each new level until some threshold is satisfied.}
\label{fig:data_analysis_level_alg}
\end{centering}
\end{figure}

\section{Alignment of data to a known model}
\label{level_alignment}

Expected current levels for a given 5-mer sequence in MspA are known.  This is the ``known model."  The ssDNA strand translocating through the nanopore can be modeled as a hidden Markov model (see Figure \ref{fig:data_analysis_hmm}), where the hidden states, $S_i$ are the actual location along the ssDNA strand.  There are certain probabilities of transitions between those states, and the measured observable is the current, which is drawn from a distribution that depends on the hidden state (which bases are in the pore).

\begin{figure}[h]
\begin{centering}
\includegraphics[width=0.9\textwidth]{figures/hmm_model.pdf}
\caption[Data analysis: hidden Markov model]{A hidden Markov model can be used to model the process of DNA moving through a nanopore.  The hidden states $S$ of the hidden Markov model correspond to positions of the DNA strand, moved by one base.  There are certain probabilities for transitions between states of the model, and these correspond to the skip, stay, forward, and backward stepping probabilities.}
\label{fig:data_analysis_hmm}
\end{centering}
\end{figure}

The emission probability is the probability of observing a certain current, given that the Markov model is in a certain state.  The emission probability can be calculated as shown in Figure \ref{fig:data_analysis_emission}, where the area of overlap between the model current distribution and the measured current distribution is the emission probability $p(I|S_i)$ for that model state $S_i$.

\begin{figure}[h]
\begin{centering}
\includegraphics[width=\textwidth]{figures/hmm_emission.pdf}
\caption[Data analysis: hidden Markov model emission probability]{Emission probabilities, $p(I|S_i)$, correspond to the probability that hidden state $S_i$ gives rise to measured current $I$.  The emission probabilities here are calculated by computing the area of overlap between the model current distribution and the measured current distribution.}
\label{fig:data_analysis_emission}
\end{centering}
\end{figure}

The probabilities of transitions between states correspond to probabilities of: a back-step, a stay (still in same place), a forward step (the expected situation), and a skip.  If these probabilities are known, then an efficient algorithm for finding the most likely sequence of states that gave rise to the observed measurements is the Viterbi algorithm \citep{Viterbi1967, Forney1973}, which has been used in the literature for nanopore data \citep{Timp2012, Szalay2015}.  This most likely sequence of states is here referred to as an ``alignment" of the data to the known model levels.  Such an alignment is shown in Figure \ref{fig:data_analysis_alignment}.

In Figure \ref{fig:data_analysis_alignment}, panel (a) shows the aligned levels, panel (b) shows the model states assigned to each measured current level, and panel (c) shows the inner workings of the Viterbi algorithm, which traces a maximum likelihood path through all possible alignments of measured levels to model states.

\begin{figure}[H]
\begin{centering}
\includegraphics[width=0.8\textwidth]{figures/helicase_level_alignment.pdf}
\caption[Data analysis: Viterbi level alignment to model]{(a) Most likely alignment of measured current levels to known model levels.  (b) Each measured level can be labeled according to which model level it aligns with.  Regions of stays are visible, as well as skips and backsteps.  Some levels are identified as noise (red \texttt{X}).  (c) The Viterbi algorithm involves tracing the most likely path through a matrix of alignment probabilities.  The matrix is shown here, as well as the optimal path that corresponds to the most likely alignment (red line).}
\label{fig:data_analysis_alignment}
\end{centering}
\end{figure}

\section{Alignment of data for model parameter estimation}

A more difficult problem is the problem of estimating the parameters of the model (the skip, stay, forward, and backward stepping probabilities) if they are unknown.  An excellent article which explains this problem has been written by Rabiner \citep{Rabiner1989}.  There are several possible methods of solution, but they involve approximations.  The approach used here is to sample from the posteriors for the parameter values using a Markov chain Monte Carlo approach.  A proposed set of parameters is generated, and then the proposed parameters are either accepted or rejected probabilistically based on a calculation of how likely the measured current levels would be given these proposed parameters.  This likelihood computation is carried out using the forward algorithm \citep{Rabiner1989}, and is only performed on a random subset of alignments, in the case of a dataset with multiple alignments, due to computational complexity.

In Markov chain Monte Carlo, it is often the case that generating proposed sets of parameters in a clever way is critical to the success of the approach.  Here, proposals are generated according to a sub-optimal probabilistic traceback through the matrix of likelihoods generated by the forward algorithm.  This is shown in Figure \ref{fig:data_analysis_traceback}.

\begin{figure}[h]
\begin{centering}
\includegraphics[width=0.7\textwidth]{figures/hmm_traceback.pdf}
\caption[Data analysis: parameter estimation MCMC proposals]{Proposals for new sets of parameters can be generated by calculating their most likely values for a sub-optimal alignment, as in Eddy \citep{Eddy1995} or Szalay \citep{Szalay2015}.  The sub-optimal alignment is obtained by a probabilistic traceback through the matrix containing an exponentiated version of the forward variable, calculated using the forward algorithm \citep{Rabiner1989}.}
\label{fig:data_analysis_traceback}
\end{centering}
\end{figure}

The traceback is probabilistic, and depends on a parameter which functions like a temperature that increases or decreases the likelihood of alignments further and further away from optimal.  This method for sampling of sub-optimal alignments from a hidden Markov model was discussed by Eddy \citep{Eddy1995}.  A similar approach to proposal generation was used by Szalay for correcting nanopore sequence data \citep{Szalay2015}.  Here a sub-optimal alignment is used to compute the parameters which would give rise to that sub-optimal alignment, and these parameters are proposed to the Markov chain Monte Carlo algorithm.

Without this (admittedly involved) proposal generation technique, the Monte Carlo sampler does not perform well.  If the parameters are tweaked at random, the resulting alignments are usually quite poor, and the sampler is likely to reject most proposals.  If the parameters are tweaked very slowly around good starting values (say from local expectation maximization), then the alignments can be good, but the parameters are likely to get stuck in local minima, and not sample the true posteriors.

\begin{figure}[h]
\begin{centering}
\includegraphics[width=\textwidth]{figures/hmm_samples.pdf}
\caption[Data analysis: parameter estimation MCMC sampling]{Markov chain Monte Carlo provides a means to sample from the posteriors for the distributions of parameter values.  Here the alignments are performed using a simulated dataset.}
\label{fig:data_analysis_hmm_samples}
\end{centering}
\end{figure}

Samples generated using this approach from the posteriors for the forward, backward, skip, and stay probabilities are shown in Figures \ref{fig:data_analysis_hmm_samples} and \ref{fig:data_analysis_hmm_sample_hist}.  Figure \ref{fig:data_analysis_hmm_samples} plots all the samples generated by the Markov chain Monte Carlo technique.  Note that certain parameter values sometimes change together, which is a benefit of the sub-optimal alignment proposal approach.  Histograms of these values are shown in Figure \ref{fig:data_analysis_hmm_sample_hist}.  This data is simulated data, generated for the purposes of validating this approach, so the true parameter values are known.  The true parameter values are plotted as the colored vertical lines in Figure \ref{fig:data_analysis_hmm_sample_hist}.

\begin{figure}[h]
\begin{centering}
\includegraphics[width=\textwidth]{figures/hmm_sample_hist.pdf}
\caption[Data analysis: parameter estimation posteriors]{Validation fo the approach using a simulated dataset.  Histograms of the sampled posteriors are plotted for the back, stay, skip, and forward probabilities.  The known values are the vertical lines.}
\label{fig:data_analysis_hmm_sample_hist}
\end{centering}
\end{figure}
