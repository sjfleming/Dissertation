\begin{savequote}[75mm]
He shall see, that nature is the opposite of the soul, answering to it part for part. ... Its beauty is the beauty of his own mind. Its laws are the laws of his own mind. ... So much of nature as he is ignorant of, so much of his own mind does he not yet possess.
\qauthor{Ralph Waldo Emerson, ``The American Scholar"}
\end{savequote}

\chapter{Conclusion}
\label{conclusion}

\section{Summary}

The work described here has endeavored to understand and control the motion of a single DNA molecule through an MspA nanopore.  Chapter \ref{dna_thermal_motion_mspa} focused on quantifying the thermal motion of ssDNA in MspA.  The effective charges and diffusion constants of different homopolymer strands of ssDNA were measured.  The thermal motion of the ssDNA that is consistent with these measured values is also consistent with data from nanopore sequencing experiments using MspA.  It seems that the thermal motion of the ssDNA is responsible for some of the observed signal averaging over neighboring nucleotides.  The measured effective charges were found to be correlated with the current blockage measured for different homopolymer ssDNA.  Additionally, the diffusion constants seem to be correlated with the measured current fluctuations for each homopolymer.  Finally, a source of additional current noise was identified at large bias voltages, and it is proposed that it could be caused by the NeutrAvidin protein tethered to the ssDNA.

Chapter \ref{helicase_motion_control} provided details of experiments meant to control the motion of DNA through MspA by controlling the motion of a Dda helicase enzyme along the DNA strand.  It was demonstrated that applied electrophoretic force is a sufficient substitute for ATP to get the helicase to perform its enzymatic function.  When pushed by an applied force, the helicase can still step by one nucleotide at a time.  The effects of applied force and temperature on stepping rates were explored, among other parameters.  A pulse generator was constructed in an effort to control the motion of the helicase enzyme using brief voltage pulses.

\section{Future directions}

While voltage pulses only infrequently advanced the Dda helicase enzyme along DNA under the conditions explored here, it is feasible that a similar approach could work to control the motion of other DNA enzymes, including helicases and polymerases.  In addition to pulses of voltage, pulses of locally increased temperature could be utilized, since temperature has a pronounced effect on the stepping rate of Dda helicase.

\section{Outlook}

The promise of using nanopores for DNA sequencing is already being realized.  Its main advantages are in the long reads (many tens to hundreds of kilobases at a stretch) that can be obtained, as well as in speed, electrical readout, portability, and integration into large parallel arrays, as has been achieved in the MinION device made by Oxford Nanopore, Inc.

The main drawback of the nanopore approach to DNA sequencing remains the relatively lower accuracy compared to other approaches.  However, there is still much room for improvement, including in terms of new algorithms for data analysis, such as recurrent neural networks, as well as in terms of the experimental approach itself, including the use of new nanopores such as CsgG \citep{Goyal2014, Deamer2016}.  One of the main sources of error remains long homopolymer stretches.  If the stated goals of the research described here could be realized, this source of error could be eliminated.

From a broader perspective, the nanopore approach holds the promise of obtaining sequence information not only from aggregated reads from many identical DNA molecules, but from many reads of a single DNA molecule.  This, coupled with the demonstrated ability to measure epigenetic modifications with a nanopore \citep{Schreiber2013, Laszlo2013, Wescoe2014, Simpson2017}, undoubtedly means that new kinds of information about DNA will soon be made available by nanopore sequencing.
