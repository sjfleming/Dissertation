\begin{savequote}[75mm]
... the movement ascends from the first-beginnings and by successive degrees emerges upon our senses, so that those bodies also are moved which we are able to perceive in the sun's light, yet it does not openly appear by what blows they are made to do so.
\qauthor{Lucretius, \textit{On the Nature of Things}}
\end{savequote}

\chapter{Single-molecule measurements and noise}
\label{thermal_motion}

Measurements of single molecules inevitably involve small signals.  Any attempt to measure a small signal will immediately reveal that small signals are often obscured by noise.  The elimination of noise sources is the constant preoccupation of the experimentalist, and it is a necessary precursor to measuring current through a nanopore.

First and foremost, external noise sources must be eliminated to whatever extent possible.  In the experiments reported here, a Faraday cage is used to screen out external electric fields, and a vibration isolation table is used to prevent mechanical disturbances.

\section{Current noise in the open MspA nanopore}

A measurement of current in an M2-MspA nanopore in a PC lipid membrane at zero applied bias is shown in Figure \ref{fig:mspa_noise_intro}a.  The current is filtered at \SI{10}{\kHz} in hardware with the 4-pole Bessel filter built into the Axopatch 200B.  A histogram of the current is shown alongside the current-versus-time trace.  The histogram reveals that the current noise is Gaussian distributed, as demonstrated by the gray line Gaussian fit.  In panel (b), the current noise power spectral density is plotted for the same data.

The current noise power spectral density is computed by first computing a discrete Fourier transform, $\tilde{I}_k(\omega)$, of the recording of current versus time, $I_j(t)$, as:

\begin{equation}
\tilde{I}_k(f) = \sum_{j=0}^{N-1} I_j(t) \, e^{-i \, 2\pi j k / n}
\label{eqn:dft}
\end{equation}

\noindent
where $N-1$ is the number of data points recorded, and the indices $j, k \in [0,N-1]$.  ($i$ is the imaginary unit.)  The current noise power spectral density, $S_I(f)$ is then computed as

\begin{equation}
S_I(f) = \frac{2}{N} \left| \tilde{I}_k(f) \right| ^2
\label{eqn:power_spectrum}
\end{equation}

In practice, long segments of data are broken up into sections of $2^{22}$ data points.  The Fourier transforms are computed, and their magnitudes are averaged.  Some of the plots in Chapter \ref{dna_thermal_motion_mspa} further smooth the power spectral density by binning into equal bins in log frequency, and taking the median of each bin.

\begin{figure}[h]
\begin{centering}
\includegraphics[width=\textwidth]{figures/mspa_noise_intro.pdf}
\caption[Current noise in the MspA nanopore]{On the left is plotted a \SI{10}{\s} trace of ionic current through M2-MspA with no applied voltage bias.  The current is \SI{0}{\pA} on average, but there is some spread to the values, shown in the current histogram.  The thin gray line is a Gaussian fit, showing that the current noise is Gaussian distributed.  On the right, the current noise power spectral density is plotted as a function of frequency.  The current noise here is white at frequencies below \SI{1}{\kHz}, where there is a gradual increase due to capacitance, followed by a falloff due to the \SI{10}{\kHz} 4-pole Bessel hardware filter.  For reference, the calculated Johnson noise is shown as a dashed gray line.}
\label{fig:mspa_noise_intro}
\end{centering}
\end{figure}

Noise in a nanopore experiment can be caused by several different underlying physical mechanisms, which will be described here.  Central to an understanding of noise is the phenomenon of thermal motion of individual molecules, also called Brownian motion.

\section{Thermal motion of ions in an electrolyte}

Particles at a temperature $T$ have an associated kinetic energy of $\frac{1}{2} k_B T$ per degree of freedom, where $k_B$ is Boltzmann's constant.  Water molecules and ions in solution therefore have a translational kinetic energy of $\frac{3}{2} k_B T$, and by setting $\frac{1}{2} mv^2 = \frac{3}{2} k_B T$, we find that a molecule or ion with mass $m$ has a root mean square velocity $v=\sqrt{\frac{3 k_B T}{m}}$.  For a potassium ion at \SI{20}{\celsius}, this is \SI{433}{\m/\s}, an incredibly high speed \citep{Hille2001}.  However, in solution, ions and water molecules are constantly colliding with one another and undergoing diffusive random walks.

\section{Contributions to noise in nanopore experiments}

The random thermal motions of ions in solution lead to local variations in electrical potential.  The same thermal motion of electrons in a carbon resistor also leads to fluctuations in potential.  These fluctuations are the physical basis of several contributions to measured noise.  Voltage fluctuations across resistors and capacitors both cause fluctuating currents.

\subsection{Thermal fluctuations of charge in resistors}

Voltage fluctuations can be measured across a resistor, and were first quantified in 1928 by John Johnson \citep{Johnson1928} and derived from first principles by Harry Nyquist \citep{Nyquist1928}.  The noise power spectral density, $S_I (f)$, with units of current squared per frequency, is given for a resistor with resistance $R$ by

\begin{equation}
S_I(f) = 4 k_B T / R
\label{eqn:johnson}
\end{equation}

For an arbitrary circuit with complex conductance $Y(f)$, this generalizes via the fluctuation-dissipation theorem to \citep{Kogan1996, Hoogerheide2010, Sakmann1995}

\begin{equation}
S_I(f) = 4 k_B T \, \operatorname{Re}(Y(f))
\label{eqn:general}
\end{equation}

\noindent
where Re denotes the real part of the complex conductance.  This expression reduces to Equation \ref{eqn:johnson} for the simple case of a resistor.

\begin{figure}[h]
\begin{centering}
\includegraphics[width=\textwidth]{figures/noise_electronics.pdf}
\caption[Simple electrical model of the nanopore setup]{(a) Cartoon of an MspA nanopore in a PC lipid bilayer membrane, with ions in solution.  When a bias is applied, the membrane charges up as a capacitor, with opposite ions lining up on opposite sides of the membrane.  (b) A simple circuit model for the nanopore shown in (a).  $R_a$ is the access resistance, $C_m$ the membrane capacitance, and $R_p$ the resistance of the nanopore.  (c) A circuit model for the nanopore along with simplified measurement electronics.  The feedback resistor $R_f$ in the current amplifier also contributes to measurement noise.}
\label{fig:circuit_equiv}
\end{centering}
\end{figure}

\subsection{Thermal fluctuations of charge across capacitors}

Thermal fluctuations of charge carriers also cause current fluctuations when a capacitance is in series with some resistance.  In the case of a nanopore experiment, a simple model electrical circuit can be drawn as in Figure \ref{fig:circuit_equiv}.  $R_p$ is the resistance of the nanopore, and $R_f$ is the resistance of the amplifier's feedback resistor, \SI{500}{\mega\ohm} for the Axopatch 200B in whole cell mode with $\beta = 1$.  $C_m$ is the capacitance of the membrane, typically a few picofarads, and $R_a$ is the access resistance, which is given by $R_a = \rho/4r$, where $\rho$ is the resistivity of the electrolyte and $r$ is the radius of the nanopore \citep{Hall1975}.  The expression for the complex conductance is

\begin{equation}
Y(f) = i \, 2 \pi \, f \, C_{in} + \left[ 2 R_a + \frac{1}{ 1/R_p + i \, 2 \pi \, f \, C_m } \right]^{-1} + \frac{1}{R_f}
\label{eqn:admittance_model}
\end{equation}

\noindent
where $C_{in}$ is the input capacitance of the amplifier's headstage, about \SI{4}{\pF} for the Axopatch 200B.

\subsection{The case of a non-ideal capacitor}

Another contribution to current noise comes from the fact that no capacitor is completely ideal, and there is some element of dissipation.  This is referred to as dielectric loss, and is described by \citep{Sakmann1995}

\begin{equation}
S_{I,headstage}(f) = 8 \pi k_B T \, f \, C_m \tan \delta
\label{eqn:dielectric}
\end{equation}

\noindent
where for water $\tan \delta \approx 1$ \citep{Gaiduk2006,Hoogerheide2010}.  Setting $\tan \delta = 1$ appears to fit the data for this nanopore setup rather well (Section \ref{sec:full_noise_model}).

\subsection{Conductance fluctuations with an applied bias}
\label{sec:ion_number_fluctuations}

One further source of noise presents itself in nanopore experiments when a voltage bias is applied across the membrane and current begins to flow.  The conductance of the nanopore itself, $1/R_p$, at any given moment depends on the number of ions in the nanopore.  This number is not fixed, and fluctuates due to diffusion.  These conductance fluctuations lead to extra current noise when current is flowing due to an applied bias.  Sometimes referred to as ``ion number fluctuation noise" \citep{Hoogerheide2010}, it can be written

\begin{equation}
S_{I,ion}(f) = a_{ion} I^2
\label{eqn:ion_number}
\end{equation}

This current noise is proportional to the square of the current, and is shown for M3-MspA in Figure \ref{fig:mspa_noise}.  A fit to these measurements yields $a_{ion} =$ \SI{1.12e-9}{\Hz^{-1}}.  The excess white noise in the open pore is likewise proportional to the square of the current for OmpF protein nanopores \citep{Queralt-Martin2015} and solid state nanopores \citep{Hoogerheide2009}.

\begin{figure}[h]
\begin{centering}
\includegraphics[width=0.8\textwidth]{figures/mspa_noise_with_current.pdf}
\caption[Current noise in MspA depends on current]{Current noise measured in open M3-MspA, as a function of current.  Data points are the mean current noise from \num{80} to \SI{280}{\Hz}.  The red line is a fit to Equation \ref{eqn:ion_number}, with $a_{ion} = $ \SI{1.12e-9}{\Hz^{-1}}.  Evidently, the current noise is proportional to the square of the current.  For reference, the Johnson noise is shown as a dashed gray line.  The conductance of the nanopore is \SI{2.25}{\nano\siemens} in \SI{1}{\Molar} KCl, \SI{10}{\m\Molar} HEPES, pH \num{8.00}.}
\label{fig:mspa_noise}
\end{centering}
\end{figure}

\subsection{Noise due to the amplifier}

There is also a small amount of input voltage noise at the amplifier's headstage.  This is approximately \SI{3}{\nV/\sqrt{\Hz}} for the Axopatch 200B \citep{Sakmann1995}.  This voltage noise couples to the headstage's input capacitance according to

\begin{equation}
S_{I,headstage}(f) = (2 \pi \, f \, C_{in})^2 S_{V,headstage}
\label{eqn:headstage}
\end{equation}

\noindent
where $S_{V,headstage} = ($ \SI{3}{\nV/\sqrt{\Hz}} $)^2$.

\section{Full noise model}
\label{sec:full_noise_model}

A complete model of the current noise in the experiment is given by a combination of the above factors, and can be written:

\begin{multline}
S_I(f) = 4 k_B T \operatorname{Re} \left( \left[ 2 R_a + \frac{1}{ 1/R_p + i 2 \pi f C_m } \right]^{-1} + \frac{1}{R_f} \right) + 8 \pi k_B T \, f \, C_m \tan \delta + a_{ion} I^2 + (2 \pi \, f \, C_{in})^2 S_{V,headstage}
\label{eqn:full_nosie_model}
\end{multline}

The effects of a filter can be accounted for by modeling the filter's complex transfer function, $h(f)$, in MATLAB.  The filtered noise then obeys $S_{I,filered}(f) = S_I(f) |h(f)|^2$.  The hardware filter used in these experiments is a 4-pole Bessel filter at \SI{10}{\kHz}.

This noise model does a good job approximating the noise measured in experiments.  Figure \ref{fig:noise_model_fit} shows a fit of Equation \ref{eqn:full_nosie_model} to noise measured in M2-MspA.  Three noise plots are shown for applied biases of \SI{0}{\mV}, \SI{100}{\mV}, and \SI{150}{\mV}.  The parameters which were allowed to vary were $R_a$ and $C_m$.  The best fit to all three curves yields values of $R_a = $ \SI{1}{\mega\ohm} and $C_m = $ \SI{0.3}{\pF}.

\begin{figure}[h]
\begin{centering}
\includegraphics[width=0.8\textwidth]{figures/noise_model_fit.pdf}
\caption[Current noise in MspA fit to a model]{Current noise measured in M2-MspA, at three applied voltage biases.  Voltages are \SI{0}{\mV} (red), \SI{100}{\mV} (green), and \SI{150}{\mV} (blue).  The best fit of the noise model described here is used to plot the solid black lines, which correspond to $C_m = $ \SI{0.3}{\pF}, and $R_a = $ \SI{1}{\mega\ohm}.  For reference, the Johnson noise is shown as a dashed gray line.  The conductance of the nanopore is \SI{2.27}{\nano\siemens} in \SI{1}{\Molar} KCl, \SI{25}{\m\Molar} phosphate buffer, pH \num{8.00}.}
\label{fig:noise_model_fit}
\end{centering}
\end{figure}

\section{Further considerations}

\subsection{Limits on temporal resolution}

Current measurements from the Axopatch 200B are digitized by the Digidata 1440A (Molecular Devices, Inc.) at \SI{100}{\kHz}, which is one sample every \SI{10}{\us}.  Faster sampling would seem to be better, but as discussed previously, capacitance noise proportional to frequency squared begins to dominate near \SI{10}{\kHz} and above.  A \SI{10}{\kHz} filter is employed to reduce this noise.

Even if the capacitance could be greatly reduced (which would be difficult due to the need for a lipid membrane), and if the bandwidth of the current amplifier were increased, MspA conducts very little current.  Nanopores with diameters of approximately \SI{1}{\nm} have a conductance on the order of \SI{1}{\nano\siemens} in \SI{1}{\Molar} KCl in the open state, and so they conduct a current of about \SI{100}{\pA} at \SI{100}{\mV} bias.  \SI{100}{\pA} is approximately \num{600} ions per microsecond flowing through the nanopore.  When the nanopore is blocked by an analyte molecule, relevant differences in current can be on the order of \SI{1}{\pA}.  This corresponds to differences of only \num{6} ions per microsecond.  Statistical variations in this number due to counting become more significant the faster the current is sampled.

\subsection{Motion of analyte molecules}

The solution would seem to be that analyte molecules should spend as much time as possible in the nanopore.  Free translocation of ssDNA, shown in Figure \ref{fig:mspa_100mer_scatter}, reveals that ssDNA translocates through the pore very quickly, spending on the order of \SI{1}{\us} on each base at \SI{180}{\mV}.  The preceding discussion reveals that this is not nearly enough time to obtain meaningful signal to make any sort of base identification.  Reducing the applied bias voltage does not help obtain more information, since it slows the translocation of the DNA while at the same time slowing the motion of ions proportionally.  Biochemical strategies for slowing down ssDNA translocation in a nanopore are discussed in Chapter \ref{dna_enzymes}.

Analyte DNA molecules themselves undergo thermal motion.  In a nanopore, this thermal motion leads to back-and-forth motion of the DNA in the nanopore's narrowest constriction, and causes the signal to be averaged over several bases at a time.  This sort of thermal motion of DNA is its own source of noise.