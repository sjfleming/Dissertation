\begin{savequote}[75mm]
... the movement ascends from the first-beginnings and by successive degrees emerges upon our senses, so that those bodies also are moved which we are able to perceive in the sun's light, yet it does not openly appear by what blows they are made to do so.
\qauthor{Lucretius, \textit{On the Nature of Things}}
\end{savequote}

\chapter{Single-molecule measurements and noise}
\label{thermal_motion}

Brief historical background.

\section{Thermal motion of ions in an electrolyte}

\section{Noise measurement in the MspA nanopore}

Nanopores with diameters of approximately \SI{1}{\nm} have a conductance on the order of \SI{1}{\nano\siemens} in the open state, and so they conduct a current of about \SI{100}{\pA} at \SI{100}{\mV} bias.  A current of \SI{100}{\pA} is approximately \num{600} ions per microsecond flowing through the nanopore.

End with a mention that thermal motion of an analyte DNA molecule would be its own source of noise.