\begin{savequote}[75mm]
At first sight, the essentially statistical character of atomic mechanics might even seem difficult to reconcile with an explanation of the marvellously refined organisation, which every living thing possesses.
\qauthor{Niels Bohr, ``Light and Life" \cite{Bohr1933a}}
\end{savequote}

\chapter{Lipid membranes and the MspA nanopore}
\label{lipids_mspa}

The fundamental unit of life in biology is the cell.  Though they are incredibly sophisticated and there is unbelievable diversity among species, cells all have in common that they must have a barrier between the ``inside" of the cell and the ``outside" world, and that they must have a means by which to transfer materials from the inside to the outside of the cell and vice versa.

\begin{figure}[h]
\begin{centering}
\includegraphics[width=0.8\textwidth]{figures/cell.pdf}
\caption[Ion channels in cells]{Cartoon of a cell.  The cell membrane is studded with proteins of all kinds.  Some of these proteins span both leaflets of the lipid bilayer membrane.  Ion channels allow certain ions in and out of the cell in a controlled manner.  Porins are open holes in the cell membrane that allow small molecules in and out of the cell.}
\label{fig:ion_channels}
\end{centering}
\end{figure}

\section{Membranes and ion channels in biology}

Cells have solved the problem of setting up a barrier that defines ``inside" and ``outside" with the \textit{cell membrane} (see Figure \ref{fig:ion_channels}).  The detailed structure of cell membranes was the subject of much debate up until 1966, when Dan Branton showed experimentally that cell membranes are composed of a lipid bilayer, studded by proteins, some of which span both leaflets of the membrane \citep{Branton1966,Branton2016}.

\begin{figure}[h]
\begin{centering}
\includegraphics[width=0.8\textwidth]{figures/membrane.pdf}
\caption[Lipids and the lipid bilayer membrane]{The lipid referred to in this work as ``PC" is 1,2-diphytanoyl-\textit{sn}-glycero-3-phosphocholine.  The chemical structure of the lipid is shown, along with a cartoon version showing its amphiphilic character.  Self-assembly of this lipid can result in many three-dimensional structures, one of which is the lipid bilayer that comprises cell membranes.  Water occupies the space above and below the membrane, next to the hydrophilic head groups, and the hydrophobic tails hide in a central layer, through which ions cannot pass.  The lipid bilayer is a fluid structure.}
\label{fig:lipids}
\end{centering}
\end{figure}

The structure of lipids and a lipid bilayer membrane are shown in Figure \ref{fig:lipids}.  The lipid used in this work is 1,2-diphytanoyl-\textit{sn}-glycero-3-phosphocholine, and will be referred to as ``PC."  All lipids are composed of a hydrophilic head group and hydrophobic tails, as shown.  Under the appropriate conditions, lipids can self-assemble into a fluid bilayer structure.  This is the structure of lipid membranes.  The hydrophilic heads face aqueous solution on both sides, while the interior is extremely hydrophobic, and prevents water (to some extent) and ions (almost completely) from passing through.

Of the proteins which span both leaflets of the bilayer membrane, some allow for the controlled passage of ions from one side of the membrane to the other.  These proteins are known as ``ion channels."  For a thorough treatment of ion channels, see Reference \citenum{Hille2001}.  Other membrane-spanning proteins have larger openings, on the order of nanometers.  These proteins allow for the uncontrolled passage of anything small enough to fit through them, and are known as ``porins."  \textit{Mycobacterium smegmatis} porin A [MspA] is one of these.

\begin{figure}[h]
\begin{centering}
\includegraphics[width=\textwidth]{figures/mspa.pdf}
\caption[The protein nanopore MspA]{The three-dimensional structure of the protein nanopore MspA, viewed as a solvent-accessible surface.  MspA is a porin comprised of eight identical subunits.  On the left and right, one subunit is shown in blue.  The middle image depicts a cross-section of the nanopore, with the cut part in gold.  MspA has a funnel shape, and the narrowest constriction is approximately \SI{1.2}{\nm} in diameter.  This image was generated in PyMol using the crystal structure accessed from the Protein Data Bank (1UUN), obtained by Faller \textit{et al.} at \SI{2.5}{\angstrom} resolution using X-ray diffraction \citep{Faller2004}.}
\label{fig:mspa}
\end{centering}
\end{figure}

\section{The engineering of MspA for DNA experiments}

The three-dimensional structure of the MspA protein is shown in Figure \ref{fig:mspa}.  In its biological context, MspA forms a hydrophilic, water-filled pore through the otherwise thick hydrophobic membrane of mycobacterium.  This pore allows the passage of hydrophilic molecules of appropriate size and charge into and out of the cell \citep{Niederweis1999}, and is the main pathway for hydrophilic molecules to enter and exit \citep{Stahl2001}.  MspA is an octameric protein composed of \SI{19406}{\dalton} subunits, forming a \SI{150}{\kilo\dalton} porin \citep{Niederweis1999}.

Initial work cloning the \textit{mspA} gene was done in 1999 by Niederweis \textit{et al.} \citep{Niederweis1999}.  That work  included single-channel recordings of ionic conductance through wild-type MspA channels reconstituted in an artificial lipid bilayer.  The structure of the MspA protein was solved by Faller \textit{et al.} in 2004 \citep{Faller2004} (Figure \ref{fig:mspa}).  The size of the narrowest constriction of MspA, at approximately \SI{1.2}{\nm}, is just right to allow the passage of ssDNA, but too narrow to accommodate dsDNA.

\begin{figure}[h]
\begin{centering}
\includegraphics[width=\textwidth]{figures/mspa_mutations.pdf}
\caption[Genetic engineering of MspA]{The wild-type MspA protein has negatively-charged aspartic acids lining its narrowest constriction.  Much work was done by Butler \textit{et al.} \citep{Butler2008} in engineering MspA to allow negatively-charged ssDNA to translocate through it.  This work resulted in ``M2" MspA, shown on the left.  M2-MspA is the pore used in most of the work reported here, except for the work reported in Chapter \ref{dna_thermal_motion_mspa}, and where otherwise noted.  The MspA mutant used in Chapter \ref{dna_thermal_motion_mspa} has a few extra mutations, and is shown on the right.  In this work, this MspA mutant will be referred to as ``M3."}
\label{fig:mspa_engineering}
\end{centering}
\end{figure}

The first use of MspA in a nanopore experiment with the aim of characterizing DNA was by Butler \textit{et al.} in 2008 \citep{Butler2008}.  In that work, wild-type MspA was found to have a conductance of around \SI{5}{\nano\siemens} in \SI{1}{\Molar} KCl at room temperature.  However, the wild-type pore would spontaneously gate at applied bias voltages above \SI{60}{\mV}.  ssDNA was not detectable.  Butler \textit{et al.} surmised that the negative charges lining the narrowest constriction of wild-type MspA were preventing ssDNA from entering and translocating through the pore (since the phosphate backbone of DNA is also negatively charged in solution).  They made the mutations D90N/D91N/D93N, changing the negatively charged aspartic acid residues lining the constriction to neutral asparagine residues.  This allowed ssDNA to translocate through the pore, and also prevented spontaneous gating.  Three further mutations, E139K/D134R/D188R, added positively charged amino acids in the vestibule and around the top of MspA.  Butler \textit{et al.} found that this enhanced the capture rate of DNA at lower voltage biases.  This pore is called "M2-MspA," and is shown in Figure \ref{fig:mspa_engineering}.

\begin{figure}[h]
\begin{centering}
\includegraphics[width=\textwidth]{figures/mspa_surface_charge.png}
\caption[Hydrophobicity and surface charge of MspA]{Surface of MspA, colored by surface charge and hydrophobicity.  Hydrophobic regions are yellow, while negative regions are red, positive are blue, and neutral hydrophilic areas are white.  The outer surface shows a clear hydrophobic transmembrane region.  Cutaways of wild-type MspA and M3-MspA show the nature of the mutations made to accommodate negatively-charged DNA.  Coloring scheme is according to Hagemans \textit{et al.} \citep{Hagemans2015}.}
\label{fig:mspa_engineering_charge}
\end{centering}
\end{figure}

A closer look at the surface of MspA reveals why it is a tranmembrane porin.  Figure \ref{fig:mspa_engineering_charge} shows surface charges and hydrophobicity, colored according to the scheme developed by Hagemans \textit{et al.} \citep{Hagemans2015}.  Hydrophobic amino acids are in yellow, while negative charges are red and positive charges are blue.  White is neutral.  The figure shows that the outer surface of MspA has a $\sim$ \SI{5}{\nm} barrel of hydrophobic amino acids on the surface.  This sits in the lipid bilayer, which is also hydrophobic.  When MspA inserts into a lipid bilayer, this bottom half will insert, ensuring that pores introduced on a particular side of the membrane always have the same orientation.  The middle and right panels of Figure \ref{fig:mspa_engineering_charge} show a cutaway view of the interior of MspA.  Wild-type MspA has a negatively-charged constriction, while M3-MspA has a neutral constriction, and several added positive charges in the vestibule.

It is worth noting that ``M2" is the standard nomenclature used for a commonly used MspA mutant which can be found in the literature \citep{Butler2008,Manrao2012,Schreiber2013,Derrington2015}.  The designation of the second MspA mutant (Figure \ref{fig:mspa_engineering}, right) as ``M3" is so far unique to this dissertation, as this mutant does not appear in the literature, except in the work published by our group \citep{Lu2015,Fleming2017} which is described in Chapter \ref{dna_thermal_motion_mspa}.


\section{Measurements obtained with the MspA nanopore}

Nanopore experiments are measurements of a single molecule, and this is ensured by the experimental setup.  A lipid bilayer of PC is formed across a $\sim$ \SI{30}{\um} aperture in a Teflon partition that separates two volumes of buffered KCl solution, each contacted by an electrode.  A single nanopore is allowed to insert into this lipid bilayer, and forms the only conductive path through the membrane.  Details of the protocol are contained in Appendix \ref{protocol}.

\subsection{Experimental setup}
\label{experimental_setup}

A schematic of the experimental setup is shown in Figure \ref{fig:setup}.  Silver / silver chloride [Ag/AgCl] electrodes make contact with an aqueous electrolyte on either side of a lipid bilayer membrane.  The lipid is typically diphytanoyl-phosphatidylcholine [PC].  A bilayer is formed over a small ($\sim$ \SI{30}{\um}) aperture in a Teflon partition by dissolving PC in a solvent oil such as 1-hexadecene and spreading the lipid over the aperture.  A single MspA nanopore is inserted into this bilayer membrane.  The electrodes are connected to a sensitive current amplifier: in this work, the Axopatch 200B (Molecular Devices, Inc.).  A simplified version of the current amplifier is also shown schematically in Figure \ref{fig:setup}, where the current amplification comes from the fact that whatever current flows through the nanopore is the same current that flows through the feedback resistor, $R_f$.  Conceptually, this means that even a small current (\SI{1}{\pA}) can cause a measurable voltage (\SI{1}{\mV}) at the output of the operational amplifier if the value of $R_f$ is large enough (\SI{1}{\giga\ohm}).

\begin{figure}[h]
\begin{centering}
\includegraphics[width=0.9\textwidth]{figures/experiment_setup.pdf}
\caption[Experimental setup]{Schematic of the experimental setup.  Ag/AgCl electrodes make contact with a buffered solution of potassium chloride [KCl].  A lipid bilayer of PC blocks the passage of ions from one electrode to the other.  The only current which can cross the membrane is the ionic current through the single MspA nanopore.  The current is measured using a current amplifier.  The simplified concept of current amplification is sketched on the right.  A bias is applied across the membrane, and the picoampere current which flows through the nanopore also flows through the large feedback resistor, $R_f$.  The output voltage is then proportional to the current through the nanopore, but larger by a factor of $R_f$.}
\label{fig:setup}
\end{centering}
\end{figure}

Note that the electrodes are connected to the experiment in such a way that a positive applied voltage, $V_{bias}$, means that the ``trans" side of the membrane (Figure \ref{fig:setup}) is positive with respect to the ``cis" side, where analyte molecules are typically introduced.  This means that at positive voltage bias, negatively-charged DNA will be drawn through the nanopore, from the cis side to the trans side.

\subsection{Electronics considerations}

Using the Axopatch 200B in ``whole cell" mode employs a feedback resistor, $R_f$, of \SI{500}{\mega\ohm}, which means that a \SI{1}{\pA} current shows up as an output voltage of \SI{0.5}{\mV}.  Following this amplification by another 20-fold amplification (the headstage accounts for a factor of two, and setting the output gain $\alpha=10$ on the Axopatch 200B amplifies by a further factor of ten) results in a signal of \SI{10}{\mV} per \SI{1}{\pico\ampere} of ionic current.  The analog-to-digital converter is a Digidata 1440a from Molecular Devices, Inc., with 16-bit conversion over a range of \SI{-10}{\volt} to $+$\SI{10}{\volt}, meaning that the signal is quantized in \SI{0.3}{\mV} steps.  With a digitization noise of less than \SI{1}{\mV}, this means that a \SI{10}{\milli\volt} signal, corresponding to \SI{1}{\pico\ampere} of ionic current, is readily measurable.  This is quite a feat, considering that \SI{1}{\pA} is only about \num{60} ions every \SI{10}{\us}, and the current can be sampled as fast as once every \SI{4}{\us}.

Typically, ionic current is recorded at \SI{100}{\kHz} after hardware filtering with a 4-pole Bessel filter at \SI{10}{\kHz}.  The Axopatch 200B applies a voltage bias across the membrane by setting the value of $V_{bias}$ in Figure \ref{fig:setup}, and this voltage is also recorded at \SI{100}{\kHz} during all experiments.

\subsection{Current as a function of applied voltage}

The simplest experiment that can be performed is a measurement of ionic current through the open nanopore, without any analyte molecules present.  Open pore measurements of M2-MspA are shown in Figure \ref{fig:iv_mspa}.  At room temperature ($\sim$ \SI{23}{\degreeCelsius}) in \SI{1}{\Molar} KCl, \SI{25}{\milli\Molar} phosphate buffer, at pH \num{8.00}, the conductance of M2-MspA is roughly \SI{2}{\nano\siemens}.  However, as shown in Figure \ref{fig:iv_mspa}a, the current-versus-voltage curve is not linear.  The characteristic larger conductance at negative applied biases, as well as the slight nonlinear dip in conductance between $\sim$ \SI{100}{\mV} and $\sim$ \SI{180}{\mV}, reveals the pore's orientation in the membrane.  This current-versus-voltage curve is recorded at the beginning of each new experiment, in order to ensure that the pore orientation is as expected.

\begin{figure}[h]
\begin{centering}
\includegraphics[width=\textwidth]{figures/mspa_iv.pdf}
\caption[MspA current versus voltage]{Measurements of a single, upright, open M2-MspA nanpore without any analyte molecules.  (a) Current versus voltage relation measured in \SI{1}{\Molar} KCl, \SI{25}{\milli\Molar} phosphate buffer, pH \num{8.00}, at room temperature.  (b) Current is plotted as a function of time.  At \num{3} seconds, the applied voltage is switched from $+$\SI{100}{\milli\volt} to \SI{-100}{\milli\volt}.  It can be seen that the nanopore spontaneously closes periodically at \SI{-100}{\milli\volt} bias.  This is referred to as ``gating."}
\label{fig:iv_mspa}
\end{centering}
\end{figure}

Figure \ref{fig:iv_mspa}b shows \num{6} seconds of a current versus time trace.  The first \SI{3}{\s} are at $+$\SI{100}{\mV} applied bias across the membrane, while the last \SI{3}{\s} are at $-$\SI{100}{\mV}.  It can be seen that the current is very stable and quiet at $+$\SI{100}{\mV}.  At $-$\SI{100}{\mV}, the pore occasionally closes spontaneously.  This is referred to as ``gating," and it was also observed by Butler \textit{et al.} for M2-MspA \citep{Butler2008}.  Gating at large negative applied bias is another confirmation that the nanopore is inserted in the correct orientation; however, gating is not as reliable a factor as the current-versus-voltage curve, due to occasional variability between gating in individual pores.


\subsection{Effect of temperature}

The amount of ionic current passing through an MspA nanopore at constant voltage depends on temperature.  Measurements of current versus voltage at constant temperature are shown in Figure \ref{fig:mspa_current_temp}a, at temperatures from \SI{5}{\degreeCelsius} to \SI{60}{\degreeCelsius} in steps of \SI{5}{\degreeCelsius}.  Figure \ref{fig:mspa_current_temp}b shows the same data plotted as conductance, $G=I/V$, versus voltage.  The non-linear relationship between current and voltage is clearly visible, and this makes the nanopore different from a simple resistor.

\begin{figure}[h]
\begin{centering}
\includegraphics[width=\textwidth]{figures/mspa_temp.pdf}
\caption[MspA current versus temperature]{Measurements of the effect of temperature on the open pore current of M3-MspA.  (a) The current is plotted as a function of voltage at temperatures from \SI{5}{\degreeCelsius} to \SI{60}{\degreeCelsius} in steps of \SI{5}{\degreeCelsius}.  (b) The same data are plotted as conductance versus voltage.  Note that for an ideal resistor, the conductance would be constant as a function of voltage.}
\label{fig:mspa_current_temp}
\end{centering}
\end{figure}

What exactly is responsible for the dependence of open pore conductance on temperature?  An initial hypothesis might be that the change in the viscosity of the electrolyte solution with temperature should contribute.  The electrical mobility, $\mu$, of an ion is related to its diffusion constant, $D$, its charge, $q$, and the surrounding temperature, $T$, in the following way:

\begin{equation}
\frac{D}{\mu} = \frac{k_B T}{q}
\label{eqn:einstein_mobility}
\end{equation}

\noindent
where $k_B$ is Boltzmann's constant.  At the small length scales of an ion, a small Reynolds number means that viscous forces dominate, and the Stokes-Einstein relation is expected to hold, giving

\begin{equation}
D = \frac{k_B T}{6\pi\eta r}
\label{eqn:stokes_einstein}
\end{equation}

\noindent
for the diffusion constant, $D$, of an ion in free solution, where $\eta$ is the dynamic viscosity of the solution and $r$ is the ion's radius.  Combining Equations \ref{eqn:einstein_mobility} and \ref{eqn:stokes_einstein} gives the ionic mobility in terms of solution viscosity:

\begin{equation}
\mu = \frac{e}{6 \pi \eta r}
\label{eqn:mobility}
\end{equation}

\noindent
where for ions of valence $1$, including potassium and chloride ions, the charge $q=e$, the charge of one electron.

The ionic mobility can be used to compute the total solution conductivity, $\lambda$, for the sum of two species of ions, here potassium and chloride:

\begin{equation}
\lambda = F(\mu_1 + \mu_2)
\label{eqn:conductivity}
\end{equation}

\noindent
where $F$ is Faraday's constant, the electric charge per mole of electrons.  Written explicitly in terms of solvent viscosity:

\begin{equation}
\lambda = \frac{1}{\eta} \frac{Fe}{6\pi} \left( \frac{1}{r_1}+\frac{1}{r_2} \right)
\label{eqn:conductivity_eta}
\end{equation}

The only temperature-dependent term in the above equation is the viscosity, and so the equation can be rewritten

\begin{equation}
\lambda(T) = \lambda(T_0) \left[ \frac{\eta(T_0)}{\eta(T)} \right]
\label{eqn:conductivity_temp}
\end{equation}

\noindent
where $\lambda(T)$ is the conductivity at some temperature $T$, and likewise $\eta(T)$ is the viscosity of the solution at temperature $T$.

The total conductance measured in a nanopore experiment, $G$, is directly proportional to this solution conductivity, $\lambda$.  Assuming that the the temperature-dependence of $G$ is due solely to solution viscosity, then $G$ has the same temperature dependence as $\lambda$:

\begin{equation}
G(T) = G(T_0) \left[ \frac{\eta(T_0)}{\eta(T)} \right]
\label{eqn:conductance_temp}
\end{equation}

The viscosity of liquid water as a function of temperature was measured by Kestin \textit{et al.} \cite{Kestin1978}, and is plotted in Figure \ref{fig:mspa_current_temp_viscosity}a along with a fit (solid black line).  Using this fit for viscosity, Figure \ref{fig:mspa_current_temp_viscosity}b then plots Equation \ref{eqn:conductance_temp} as a solid black line, with $T_0=$ \SI{25}{\degreeCelsius}.  The data points (circles) are the \SI{50}{\mV} data points shown in Figure \ref{fig:mspa_current_temp}b.  Scaling Equation \ref{eqn:conductance_temp} to match the data point at \SI{25}{\degreeCelsius} results in an excellent fit to the data.  This suggests that solution viscosity is the major contributor to the  change in conductance with temperature in an open nanopore.  A similar conclusion was reached by Meller and Branton \citep{Meller2002}, who found that the temperature-dependence of the open pore conductance of $\alpha$-hemolysin was proportional to the bulk mobility of the electrolyte as a function of temperature.  As noted by Meller and Branton, this suggests that these pores do not undergo large conformational changes in this temperature range.  At temperatures from about \SI{50}{\degreeCelsius} to \SI{90}{\degreeCelsius}, deviations on the order of $10-20\%$ are observed in $\alpha$-hemolysin, with the pore conducting more current than would be predicted due to viscosity alone \citep{Kang2005}.

\begin{figure}[h]
\begin{centering}
\includegraphics[width=\textwidth]{figures/mspa_temp_viscosity.pdf}
\caption[MspA current versus temperature explained by viscosity]{The viscosity of water is responsible for the dependence of M3-MspA open pore current on temperature.  On the left, the black circles are measurements of the viscosity of water taken from Kestin \textit{et al.} \cite{Kestin1978}.  The black line is an exponential fit, the equation of which is shown.  The right panel shows data measured in this work (black circles).  Each circle is the average conductance measured for open pore M3-MspA at 50mV bias.  The black line is Equation \ref{eqn:conductance_temp}, and it is scaled to be equal to the conductance measured at \SI{25}{\degreeCelsius}.  Evidently, the temperature dependence of the open pore conductance is captured by accounting for the change in the viscosity of water with temperature.}
\label{fig:mspa_current_temp_viscosity}
\end{centering}
\end{figure}

\subsection{Molecules block current through the MspA nanopore}

Using the nanopore as a sensing instrument involves introducing other analyte molecules into the experiment.  Analyte molecules are typically introduced on the cis side of the membrane (see Figure \ref{fig:setup}), and so a positive applied voltage results in an electric field which pulls negatively-charged analyte molecules from the cis side through the nanopore to the trans side of the membrane.  In these experiments, DNA is typically the anlayte of choice, and the phosphate backbone of DNA acquires one negative charge per phosphate in aqueous solutions with pH above $\sim 1.5$.  The nucleobases themselves (A, T, C, and G) are uncharged in the pH range of \num{4.4} to \num{9.2} \citep{Verdolino2008, Acharya2004}.

\begin{figure}[h]
\begin{centering}
\includegraphics[width=\textwidth]{figures/ATP_event_basics.pdf}
\caption[Current blockage events in the MspA nanopore]{Ionic current through M2-MspA is plotted as a function of time.  Electrolyte solution is \SI{1}{\Molar} KCl, \SI{25}{\milli\Molar} phosphate, pH \num{8.00}.  Measurements are taken at \SI{23}{\degreeCelsius}, and \SI{160}{\mV} bias is applied across the membrane.  Data are low-pass filtered at 10kHz.  Blockages of the open pore current signify single molecules in the nanopore.  Here the blockages are caused by ATP molecules at 2mM concentration.  Current blockage ``events" can be characterized by a mean current blockage and a duration.}
\label{fig:mspa_event_basics}
\end{centering}
\end{figure}

Simpler even than a strand of ssDNA is a single nucleotide triphosphate, such as ATP.  ATP alone, though it is a small molecule, causes enough of a current blockage in the M2-MspA nanopore to be detected as an individual blockage event.  (It is interesting to note that in the VDAC nanopore, which has a diameter of approximately \SI{3}{\nm}, ATP cannot be detected as individual current blockage events, but rather manifests as a drop in average current and an increase in current noise \citep{Rostovtseva1998, Rostovtseva2002}.)  Figure \ref{fig:mspa_event_basics} shows the ionic current trace when ATP is introduced on the cis side of the membrane at a concentration of \SI{2}{\milli\Molar}.  The open nanopore alone, consistent with the discussion above, shows a steady current at \SI{160}{\mV}.  After the addition of ATP (center panel), individual current blockage events are visible.  One such individual blockage event is shown in detail (right panel).  Certain characteristic quantities can be computed for each event, including the local open pore current just before the event ($I_0$), the mean current while the molecule is in the pore ($I$), and the duration of the current blockage.

\begin{figure}[h]
\begin{centering}
\includegraphics[width=0.6\textwidth]{figures/100mer_ssDNA_scatter.pdf}
\caption[Scatter plot of 100mer ssDNA in MspA]{Scatter plot of just over 5700 events of 100mer ssDNA translocating through M3-MspA in \SI{1}{\Molar} KCl, \SI{10}{\milli\Molar} HEPES, pH \num{8.00}.  The applied voltage bias is \SI{180}{\mV}.  There is an edge at around \SI{0.1}{ms} to the left of which no events occur.  This limit corresponds to the shortest events detectable when using a \SI{10}{\kHz} low-pass filter in hardware.}
\label{fig:mspa_100mer_scatter}
\end{centering}
\end{figure}

The mean current blockage and duration can be computed for all the events observed, and a scatter plot can be created.  For example, Figure \ref{fig:mspa_100mer_scatter} shows a scatter plot of all the events recorded as \num{100}mer ssDNA (sequence \# 0080 in Appendix \ref{sequences}) translocates through the M3-MspA nanopore.  The vertical axis shows the ratio of blocked-pore current to open-pore current ($I/I_0$), while the horizontal axis shows the event duration on a logarithmic scale.  Scatter plots are a convenient way to visualize all events together, and are helpful for observing clusters of similar types of events.

For example, the scatter plot in the center of Figure \ref{fig:mspa_5kb_scatter} shows groupings of events in different regions of the scatter plot of \num{5300} base (\SI{5.3}{\kilo\base}) ssDNA from bacteriophage lambda.  The four panels in the figure show overlays of events from four regions of the scatter plot.  Shown in this way, it becomes apparent that the events from the region of the scatter plot denoted by the red box all have a similar structure.  The current drops to near $0.6 I_0$ and remains there until, after a variable waiting period, the current drops to $<0.1 I_0$ and the event ends, presumably as the strand translocates through the nanopore.  This is in contrast with the events in the region of the scatter plot denoted by the yellow box, where the current drops to near $0.1 I_0$ immediately, and the events are quite short in duration.  It could be hypothesized that the yellow box contains events for which the free end of the ssDNA is captured immediately, whereas the red box contains events where the tangled ball of ssDNA is captured by the nanopore, but it takes many attempts to find a free ssDNA end, at which point the ssDNA can translocate.

\begin{figure}[h]
\begin{centering}
\includegraphics[width=0.9\textwidth]{figures/5kb_ssDNA_scatter.pdf}
\caption[Scatter plot of 5.3kb ssDNA in MspA]{A scatter plot of several hundred events of 5.3kb ssDNA translocating through M3-MspA is shown in the center.  Small regions from the scatter plot have been selected, and the events contained in those regions are plotted, one on top of the other, in the four corner plots.  In the event overlays, the ends of the events are synchronized at $t=0$.  The applied voltage bias is \SI{180}{\mV}, and the electrolyte is \SI{1}{\Molar} KCl, \SI{10}{\milli\Molar} HEPES, pH \num{8.00}.}
\label{fig:mspa_5kb_scatter}
\end{centering}
\end{figure}

