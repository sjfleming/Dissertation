\begin{savequote}[75mm]
At first sight, the essentially statistical character of atomic mechanics might even seem difficult to reconcile with an explanation of the marvellously refined organisation, which every living thing possesses.
\qauthor{Niels Bohr, ``Light and Life" \cite{Bohr1933a}}
\end{savequote}

\chapter{Lipid membranes and the MspA nanopore}
\label{lipids_mspa}

The fundamental unit of life in biology is the cell.  Though they are incredibly sophisticated and there is unbelievable diversity among species, cells all have in common that they must have a barrier between the ``inside" of the cell and the ``outside" world, and that they must have a means by which to transfer materials from the inside to the outside of the cell and vice versa.

\section{Membranes and ion channels in biology}

Cells have solved the problem of setting up a barrier that defines ``inside" and ``outside" with the \textit{cell membrane}.  The detailed structure of cell membranes was the subject of much debate up until 1966, when Dan Branton showed experimentally that cell membranes are composed of a lipid bilayer, studded by proteins, some of which span both leaflets of the membrane \citep{Branton1966}.

Many proteins span the entire membrane, but some also allow for passage of ions from one side of the membrane to the other.  These proteins are known as ``ion channels."  For a thorough treatment of ion channels, see Reference \citenum{Hille1978}.  Other membrane-spanning proteins have larger openings, on the order of nanometers, and are known as ``porins."  \textit{Mycobacterium smegmatis} porin A [MspA] is one of these.

\begin{figure}[h]
\begin{centering}
\includegraphics[width=0.6\textwidth]{figures/nanopore.pdf}
\caption[Ion channels in cells]{...}
\label{fig:ion_channels}
\end{centering}
\end{figure}

\begin{figure}[h]
\begin{centering}
\includegraphics[width=\textwidth]{figures/mspa.pdf}
\caption[The protein nanopore MspA]{The three dimensional structure of the protein nanopore MspA.  MspA is a porin comprised of eight identical subunits.  On the left and right, one subunit is shown in blue.  The middle image depicts a cross-section of the nanopore, with the cut part in gold.  MspA has a funnel shape, and the narrowest constriction is approximately 1.2nm in diameter.  This image was generated in PyMol using the crystal structure accessed from the Protein Data Bank (1UUN), obtained by Faller \textit{et al.} at 2.5\AA $ $ resolution using X-ray diffraction \citep{Faller2004a}.}
\label{fig:mspa}
\end{centering}
\end{figure}

\begin{figure}[h]
\begin{centering}
\includegraphics[width=0.6\textwidth]{figures/nanopore.pdf}
\caption[Lipids and the lipid bilayer membrane]{...}
\label{fig:lipids}
\end{centering}
\end{figure}

\section{The engineering of MspA for DNA experiments}

\begin{figure}[h]
\begin{centering}
\includegraphics[width=0.6\textwidth]{figures/nanopore.pdf}
\caption[Genetic engineering of MspA]{...}
\label{fig:mspa_engineering}
\end{centering}
\end{figure}

\section{Electrical measurements of the MspA nanopore}

\subsection{Experimental setup}

\begin{figure}[h]
\begin{centering}
\includegraphics[width=0.6\textwidth]{figures/nanopore.pdf}
\caption[Experimental setup]{...}
\label{fig:setup}
\end{centering}
\end{figure}

\subsection{Current as a function of applied voltage}

Mention gating at reverse bais.

\begin{figure}[h]
\begin{centering}
\includegraphics[width=0.6\textwidth]{figures/nanopore.pdf}
\caption[MspA current versus voltage]{...}
\label{fig:iv_mspa}
\end{centering}
\end{figure}

\subsection{Effect of temperature}

Its relationship to water's viscosity as a function of temperature

\begin{figure}[h]
\begin{centering}
\includegraphics[width=0.6\textwidth]{figures/nanopore.pdf}
\caption[MspA current versus temperature]{...}
\label{fig:mspa_current_temp}
\end{centering}
\end{figure}

\subsection{Molecules block current through the MspA nanopore}

ssDNA 100mer and 5kb, along with maybe ATP alone

\begin{figure}[h]
\begin{centering}
\includegraphics[width=0.6\textwidth]{figures/nanopore.pdf}
\caption[100mer ssDNA in the MspA nanopore]{...}
\label{fig:mspa_100mer}
\end{centering}
\end{figure}

\begin{figure}[h]
\begin{centering}
\includegraphics[width=0.6\textwidth]{figures/nanopore.pdf}
\caption[5.3kb ssDNA in the MspA nanopore]{...}
\label{fig:mspa_5kb}
\end{centering}
\end{figure}

\begin{figure}[h]
\begin{centering}
\includegraphics[width=0.6\textwidth]{figures/nanopore.pdf}
\caption[The ATP molecule in the MspA nanopore]{...}
\label{fig:mspa_atp}
\end{centering}
\end{figure}

\subsection{Current noise of the open MspA nanopore}

\begin{figure}[h]
\begin{centering}
\includegraphics[width=0.6\textwidth]{figures/nanopore.pdf}
\caption[Current noise in the MspA nanopore]{...}
\label{fig:mspa_open_noise}
\end{centering}
\end{figure}