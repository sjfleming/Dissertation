\begin{savequote}[75mm]
At first sight, the essentially statistical character of atomic mechanics might even seem difficult to reconcile with an explanation of the marvellously refined organisation, which every living thing possesses.
\qauthor{Niels Bohr, ``Light and Life" \cite{Bohr1933a}}
\end{savequote}

\chapter{Lipid membranes and the MspA nanopore}
\label{lipids_mspa}

The fundamental unit of life in biology is the cell.  Though they are incredibly sophisticated and there is unbelievable diversity among species, cells all have in common that they must have a barrier between the ``inside" of the cell and the ``outside" world, and that they must have a means by which to transfer materials from the inside to the outside of the cell and vice versa.

\section{Membranes and ion channels in biology}

Cells have solved the problem of setting up a barrier that defines ``inside" and ``outside" with the \textit{cell membrane}.  The detailed structure of cell membranes was the subject of much debate up until 1966, when Dan Branton showed experimentally that cell membranes are composed of a lipid bilayer, studded by proteins, some of which span both leaflets of the membrane \citep{Branton1966,Branton2016}.

Many proteins span the entire membrane, but some also allow for the controlled passage of ions from one side of the membrane to the other.  These proteins are known as ``ion channels."  For a thorough treatment of ion channels, see Reference \citenum{Hille1978}.  Other membrane-spanning proteins have larger openings, on the order of nanometers.  These proteins allow for the uncontrolled passage of anything small enough to fit through them, and are known as ``porins."  \textit{Mycobacterium smegmatis} porin A [MspA] is one of these.

\begin{figure}[h]
\begin{centering}
\includegraphics[width=0.8\textwidth]{figures/cell.pdf}
\caption[Ion channels in cells]{Cartoon of a cell.  The cell membrane is studded with proteins of all kinds.  Some of these proteins span both leaflets of the lipid bilayer membrane.  Ion channels allow certain ions in and out of the cell in a controlled manner.  Porins are open holes in the cell membrane that allow small molecules in and out of the cell.}
\label{fig:ion_channels}
\end{centering}
\end{figure}

\begin{figure}[h]
\begin{centering}
\includegraphics[width=\textwidth]{figures/mspa.pdf}
\caption[The protein nanopore MspA]{The three dimensional structure of the protein nanopore MspA, viewed as a solvent-accessible surface.  MspA is a porin comprised of eight identical subunits.  On the left and right, one subunit is shown in blue.  The middle image depicts a cross-section of the nanopore, with the cut part in gold.  MspA has a funnel shape, and the narrowest constriction is approximately \SI{1.2}{\nm} in diameter.  This image was generated in PyMol using the crystal structure accessed from the Protein Data Bank (1UUN), obtained by Faller \textit{et al.} at \SI{2.5}{\angstrom} resolution using X-ray diffraction \citep{Faller2004a}.}
\label{fig:mspa}
\end{centering}
\end{figure}

\begin{figure}[h]
\begin{centering}
\includegraphics[width=0.8\textwidth]{figures/membrane.pdf}
\caption[Lipids and the lipid bilayer membrane]{The lipid referred to in this work as ``PC" is 1,2-diphytanoyl-\textit{sn}-glycero-3-phosphocholine.  The chemical structure of the lipid is shown, along with a cartoon version showing its amphiphilic character.  Self-assembly of this lipid can result in many three-dimensional structures, one of which is the lipid bilayer that comprises cell membranes.  Water occupies the space above and below the membrane, next to the hydrophilic head groups, and the hydrophobic tails hide in a central layer, through which ions cannot pass.  The lipid bilayer is a fluid structure.}
\label{fig:lipids}
\end{centering}
\end{figure}

\section{The engineering of MspA for DNA experiments}

Initial work cloning the mspA gene was done in 1999 by Niederweis \textit{et al.} \citep{Niederweis1999}.

\begin{figure}[h]
\begin{centering}
\includegraphics[width=\textwidth]{figures/mspa_mutations.pdf}
\caption[Genetic engineering of MspA]{The wild-type MspA protein has negatively-charged aspartic acids lining its narrowest constriction.  Much work was done by Butler \textit{et al.} \citep{Butler2008} in engineering MspA to allow negatively-charged ssDNA to translocate through it.  This work resulted in ``M2" MspA, shown on the left.  M2-MspA is the pore used in most of the work reported here, except for the work reported in Chapter \ref{dna_thermal_motion_mspa}, and where otherwise noted.  The MspA mutant used in Chapter \ref{dna_thermal_motion_mspa} has a few extra mutations, and is shown on the right.  In this work, this MspA mutant will be referred to as ``M3."}
\label{fig:mspa_engineering}
\end{centering}
\end{figure}

Note that ``M2" is the standard nomenclature used for a commonly used MspA mutant which can be found in the literature \citep{Butler2008,Manrao2012,Schreiber2013,Derrington2015}.  The designation of the second MspA mutant (Figure \ref{fig:mspa_engineering}, right) as ``M3" is so far unique to this dissertation, as this mutant does not appear in the literature, except in the work published by our group \citep{Lu2015,Fleming2017} which is described in Chapter \ref{dna_thermal_motion_mspa}.

\section{Electrical measurements of the MspA nanopore}

\subsection{Experimental setup}

\begin{figure}[h]
\begin{centering}
\includegraphics[width=0.9\textwidth]{figures/experiment_setup.pdf}
\caption[Experimental setup]{Schematic of the experimental setup.  Ag/AgCl electrodes make contact with a buffered solution of potassium chloride [KCl].  A lipid bilayer of PC blocks the passage of ions from one electrode to the other.  The only current which can cross the membrane is the ionic current through the single MspA nanopore.  The current is measured using a current amplifier.  The simplified concept of current amplification is sketched on the right.  A bias is applied across the membrane, and the picoampere current which flows through the nanopore also flows through the large feedback resistor, $R_f$.  The output voltage is then proportional to the current through the nanopore, but larger by a factor of $R_f$.}
\label{fig:setup}
\end{centering}
\end{figure}

\subsection{Electronics considerations}

Using the Axopatch 200B in ``whole cell" mode employs a feedback resistor, $R_f$, of \SI{500}{\mega\ohm}, which means that a \SI{1}{\pico\ampere} current shows up as an output voltage of \SI{0.5}{\milli\volt}.  Following this amplification by another 20-fold amplification (the headstage accounts for a factor of two, and setting the output gain $\alpha=10$ on the Axopatch 200B amplifies by a further factor of ten) results in a signal of \SI{10}{\milli\volt} per \SI{1}{\pico\ampere} of ionic current.  The analog-to-digital converter is a Digidata 1440a from Molecular Devices, Inc., with 16-bit conversion over a range of \SI{-10}{\volt} to $+$\SI{10}{\volt}, meaning that the signal is quantized in \SI{0.3}{\milli\volt} steps.  With a digitization noise of less than \SI{1}{\milli\volt}, this means that a \SI{10}{\milli\volt} signal, corresponding to \SI{1}{\pico\ampere} of ionic current, is readily measurable.  This is quite a feat, considering that \SI{1}{\pico\ampere} is only about \num{60} ions every \SI{10}{\micro\second}.

\subsection{Current as a function of applied voltage}

Mention gating at reverse bais.

\begin{figure}[h]
\begin{centering}
\includegraphics[width=\textwidth]{figures/mspa_iv.pdf}
\caption[MspA current versus voltage]{On the left is the current versus voltage relation measured for the M2-MspA nanopore in \SI{1}{\Molar} KCl, \SI{25}{\milli\Molar} phosphate buffer, pH \num{8.00}, at room temperature.  On the right, current is plotted as a function of time.  At \num{3} seconds, the applied voltage is switched from $+$\SI{100}{\milli\volt} to \SI{-100}{\milli\volt}.  It can be seen that the nanopore spontaneously closes periodically at \SI{-100}{\milli\volt} bias.  This is referred to as ``gating."}
\label{fig:iv_mspa}
\end{centering}
\end{figure}

\subsection{Effect of temperature}

Its relationship to water's viscosity as a function of temperature.

For ions with valence $z=1$, like potassium and chloride ions,

Diffusion, Stokes-Einstein relation:

\begin{equation}
D = \frac{k_B T}{6\pi\eta r}
\label{eqn:conductivity}
\end{equation}

Electrical mobility:

\begin{equation}
\frac{D}{\mu} = \frac{k_B T}{q}
\label{eqn:einstein_mobility}
\end{equation}

Ionic mobility:

\begin{equation}
\mu = \frac{e}{6 \pi \eta r}
\label{eqn:mobility}
\end{equation}

Conductivity, sum of both ions:

\begin{equation}
\lambda = F(\mu_1 + \mu_2)
\label{eqn:conductivity}
\end{equation}

For two ions, in terms of solvent viscosity:

\begin{equation}
\lambda = \frac{1}{\eta} \frac{Fe}{6\pi} \left( \frac{1}{r_1}+\frac{1}{r_2} \right)
\label{eqn:conductivity_eta}
\end{equation}

\begin{equation}
\lambda(T) = \lambda(T_0) \left[ \frac{\eta(T_0)}{\eta(T)} \right]
\label{eqn:conductivity_temp}
\end{equation}

and since conductance is proportional to this ionic conductivity, the conductance, $G$, is

\begin{equation}
G(T) = G(T_0) \left[ \frac{\eta(T_0)}{\eta(T)} \right]
\label{eqn:conductance_temp}
\end{equation}

This is what is plotted in Figure \ref{fig:mspa_current_temp_viscosity}.

\begin{figure}[h]
\begin{centering}
\includegraphics[width=\textwidth]{figures/mspa_temp.pdf}
\caption[MspA current versus temperature]{Measurements of the effect of temperature on the open pore current of M3-MspA.  The same data are displayed in the left and right panels.  On the left, the current is plotted as a function of voltage at temperatures from 5°C to 60°C in steps of 5°C.  On the right, the data are plotted as conductance versus voltage.  Note that for an ideal resistor, the conductance would be constant as a function of voltage.}
\label{fig:mspa_current_temp}
\end{centering}
\end{figure}

\begin{figure}[h]
\begin{centering}
\includegraphics[width=\textwidth]{figures/mspa_temp_viscosity.pdf}
\caption[MspA current versus temperature explained by viscosity]{The viscosity of water is responsible for the dependence of M3-MspA open pore current on temperature.  On the left, the black circles are measurements of the viscosity of water taken from Kestin \textit{et al.} \cite{Kestin1978}.  The black line is an exponential fit, the equation of which is shown.  The right panel shows data measured in this work (black circles).  Each circle is the average conductance measured for open pore M3-MspA at 50mV bias.  The black line is Equation \ref{eqn:conductance_temp}, and it is scaled to be equal to the conductance measured at 25°C.  Evidently, the temperature dependence of the open pore conductance is captured exactly by accounting for the change in the viscosity of water with temperature.}
\label{fig:mspa_current_temp_viscosity}
\end{centering}
\end{figure}

\subsection{Molecules block current through the MspA nanopore}

ssDNA 100mer and 5kb, along with maybe ATP alone

\begin{figure}[h]
\begin{centering}
\includegraphics[width=\textwidth]{figures/ATP_event_basics.pdf}
\caption[Current blockage events in the MspA nanopore]{Ionic current through M2-MspA is plotted as a function of time.  Data are low-pass filtered at 10kHz.  Blockages of the open pore current signify single molecules in the nanopore.  Here the blockages are caused by ATP molecules at 2mM concentration.  Current blockage ``events" can be characterized by a mean current blockage and a duration.}
\label{fig:mspa_event_basics}
\end{centering}
\end{figure}

\begin{figure}[h]
\begin{centering}
\includegraphics[width=\textwidth]{figures/5kb_ssDNA_scatter.pdf}
\caption[Scatter plot of 5.3kb ssDNA in MspA]{A scatter plot of several hundred events of 5.3kb ssDNA translocating through M3-MspA is shown in the center.  Small regions from the scatter plot have been selected, and the events contained in those regions are plotted, one on top of the other, in the four corner plots.  The applied voltage bias is 180mV.}
\label{fig:mspa_5kb_scatter}
\end{centering}
\end{figure}

\begin{figure}[h]
\begin{centering}
\includegraphics[width=0.7\textwidth]{figures/100mer_ssDNA_scatter.pdf}
\caption[Scatter plot of 100mer ssDNA in MspA]{Scatter plot of just over 5700 events of 100mer ssDNA translocating through M3-MspA.  The applied voltage bias is 180mV.  Events are generally shorter compared to the 5.3kb ssDNA events in Figure \ref{fig:mspa_5kb_scatter}.  There is an edge at around \SI{0.1}{ms} to the left of which no events occur.  This limit corresponds to the shortest events detectable when using a 10kHz low-pass filter in hardware.}
\label{fig:mspa_100mer_scatter}
\end{centering}
\end{figure}

\subsection{Current noise of the open MspA nanopore}

\begin{figure}[h]
\begin{centering}
\includegraphics[width=\textwidth]{figures/mspa_noise_intro.pdf}
\caption[Current noise in the MspA nanopore]{On the left is plotted a \SI{10}{\s} trace of ionic current through M2-MspA with no applied voltage bias.  The current is \SI{0}{\pA} on average, but there is some spread to the values, shown in the current histogram.  The thin gray line is a Gaussian fit, showing that the current noise is Gaussian distributed.  On the right, the current noise power spectral density is plotted as a function of frequency.  The current noise here is white at frequencies below \SI{1}{\kHz}, where there is a gradual increase due to capacitance, followed by a falloff due to the \SI{10}{\kHz} 4-pole Bessel hardware filter.  For reference, the calculated Johnson noise is shown as a dashed gray line.}
\label{fig:mspa_noise_intro}
\end{centering}
\end{figure}