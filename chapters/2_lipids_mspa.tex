\begin{savequote}[75mm]
At first sight, the essentially statistical character of atomic mechanics might even seem difficult to reconcile with an explanation of the marvellously refined organisation, which every living thing possesses.
\qauthor{Niels Bohr, ``Light and Life" \cite{Bohr1933a}}
\end{savequote}

\chapter{Lipid membranes and the MspA nanopore}
\label{lipids_mspa}

The fundamental unit of life in biology is the cell.  Though they are incredibly sophisticated and there is unbelievable diversity among species, cells all have in common that they must have a barrier between the ``inside" of the cell and the ``outside" world, and that they must have a means by which to transfer materials from the inside to the outside of the cell and vice versa.

\begin{figure}[h]
\begin{centering}
\includegraphics[width=0.8\textwidth]{figures/cell.pdf}
\caption[Ion channels in cells]{Cartoon of a cell.  The cell membrane is studded with proteins of all kinds.  Some of these proteins span both leaflets of the lipid bilayer membrane.  Ion channels allow certain ions in and out of the cell in a controlled manner.  Porins are open holes in the cell membrane that allow small molecules in and out of the cell.}
\label{fig:ion_channels}
\end{centering}
\end{figure}

\section{Membranes and ion channels in biology}

Cells have solved the problem of setting up a barrier that defines ``inside" and ``outside" with the \textit{cell membrane} (see Figure \ref{fig:ion_channels}).  The detailed structure of cell membranes was the subject of much debate up until 1966, when Dan Branton showed experimentally that cell membranes are composed of a lipid bilayer, studded by proteins, some of which span both leaflets of the membrane \citep{Branton1966,Branton2016}.

\begin{figure}[h]
\begin{centering}
\includegraphics[width=0.8\textwidth]{figures/membrane.pdf}
\caption[Lipids and the lipid bilayer membrane]{The lipid referred to in this work as ``PC" is 1,2-diphytanoyl-\textit{sn}-glycero-3-phosphocholine.  The chemical structure of the lipid is shown, along with a cartoon version showing its amphiphilic character.  Self-assembly of this lipid can result in many three-dimensional structures, one of which is the lipid bilayer that comprises cell membranes.  Water occupies the space above and below the membrane, next to the hydrophilic head groups, and the hydrophobic tails hide in a central layer, through which ions cannot pass.  The lipid bilayer is a fluid structure.}
\label{fig:lipids}
\end{centering}
\end{figure}

The structure of lipids and a bilayer lipid membrane is shown in Figure \ref{fig:lipids}.  The lipid used in this work is 1,2-diphytanoyl-\textit{sn}-glycero-3-phosphocholine, and will be referred to as ``PC."  All lipids are composed of a hydrophilic head group and hydrophobic tails, as shown.  Under the appropriate conditions, lipids can self-assemble into a fluid bilayer structure.  This is the structure of lipid membranes.  The hydrophilic heads face aqueous solution on both sides, while the interior is extremely hydrophobic, and prevents water (to some extent) and ions (almost completely) from passing through.

Of the proteins which span both leaflets of the bilayer membrane, some allow for the controlled passage of ions from one side of the membrane to the other.  These proteins are known as ``ion channels."  For a thorough treatment of ion channels, see Reference \citenum{Hille1978}.  Other membrane-spanning proteins have larger openings, on the order of nanometers.  These proteins allow for the uncontrolled passage of anything small enough to fit through them, and are known as ``porins."  \textit{Mycobacterium smegmatis} porin A [MspA] is one of these.

\begin{figure}[h]
\begin{centering}
\includegraphics[width=\textwidth]{figures/mspa.pdf}
\caption[The protein nanopore MspA]{The three-dimensional structure of the protein nanopore MspA, viewed as a solvent-accessible surface.  MspA is a porin comprised of eight identical subunits.  On the left and right, one subunit is shown in blue.  The middle image depicts a cross-section of the nanopore, with the cut part in gold.  MspA has a funnel shape, and the narrowest constriction is approximately \SI{1.2}{\nm} in diameter.  This image was generated in PyMol using the crystal structure accessed from the Protein Data Bank (1UUN), obtained by Faller \textit{et al.} at \SI{2.5}{\angstrom} resolution using X-ray diffraction \citep{Faller2004}.}
\label{fig:mspa}
\end{centering}
\end{figure}

\section{The engineering of MspA for DNA experiments}

The three-dimensional structure of the MspA protein is shown in Figure \ref{fig:mspa}.  In its biological context, MspA forms a hydrophilic, water-filled pore through the otherwise thick hydrophobic membrane of mycobacterium.  This pore allows the passage of hydrophilic molecules of appropriate size and charge into and out of the cell \citep{Niederweis1999}, and is the main pathway for hydrophilic molecules to enter and exit \citep{Stahl2001}.  MspA is an octameric protein composed of \SI{19406}{\dalton} subunits, forming a \SI{150}{\kilo\dalton} porin \citep{Niederweis1999}.

Initial work cloning the \textit{mspA} gene was done in 1999 by Niederweis \textit{et al.} \citep{Niederweis1999}.  This initial work  included single-channel recordings of ionic conductance through these wild-type MspA channels reconstituted in an artificial lipid bilayer.  The structure of the MspA protein was solved by Faller \textit{et al.} in 2004 \citep{Faller2004} (Figure \ref{fig:mspa}).  The size of the narrowest constriction of MspA, at approximately \SI{1.2}{\nm}, is just right to allow the passage of ssDNA, but too narrow to accommodate dsDNA.

\begin{figure}[h]
\begin{centering}
\includegraphics[width=\textwidth]{figures/mspa_mutations.pdf}
\caption[Genetic engineering of MspA]{The wild-type MspA protein has negatively-charged aspartic acids lining its narrowest constriction.  Much work was done by Butler \textit{et al.} \citep{Butler2008} in engineering MspA to allow negatively-charged ssDNA to translocate through it.  This work resulted in ``M2" MspA, shown on the left.  M2-MspA is the pore used in most of the work reported here, except for the work reported in Chapter \ref{dna_thermal_motion_mspa}, and where otherwise noted.  The MspA mutant used in Chapter \ref{dna_thermal_motion_mspa} has a few extra mutations, and is shown on the right.  In this work, this MspA mutant will be referred to as ``M3."}
\label{fig:mspa_engineering}
\end{centering}
\end{figure}

The first use of MspA in a nanopore experiment with the aim of characterizing DNA was by Butler \textit{et al.} in 2008 \citep{Butler2008}.  In that work, wild-type MspA was found to have a conductance of around \SI{5}{\nano\siemens} in \SI{1}{\Molar} KCl at room temperature.  However, the wild-type pore would spontaneously gate at applied bias voltages above \SI{60}{\mV}.  ssDNA was not detectable.  Butler \textit{et al.} surmised that the negative charges lining the narrowest constriction of wild-type MspA were preventing ssDNA from entering and translocating through the pore (since the phosphate backbone of DNA is also negatively charged in solution).  They made the mutations D90N/D91N/D93N, changing the negatively charged aspartic acid residues lining the constriction to neutral asparagine residues.  This allowed ssDNA to translocate through the pore, and also prevented spontaneous gating.  Three further mutations, E139K/D134R/D188R, added positively charged amino acids in the vestibule and around the top of MspA.  Butler \textit{et al.} found that this enhanced the capture rate of DNA at lower voltage biases.  This pore is called "M2-MspA," and is shown in Figure \ref{fig:mspa_engineering}.

\begin{figure}[h]
\begin{centering}
\includegraphics[width=\textwidth]{figures/mspa_surface_charge.png}
\caption[Hydrophobicity and surface charge of MspA]{Surface of MspA, colored by surface charge and hydrophobicity.  Hydrophobic regions are yellow, while negative regions are red, positive are blue, and neutral hydrophilic areas are white.  The outer surface shows a clear hydrophobic transmembrane region.  Cutaways of wild-type MspA and M3-MspA show the nature of the mutations made to accommodate negatively-charged DNA.  Coloring scheme is according to Hagemans \textit{et al.} \citep{Hagemans2015}.}
\label{fig:mspa_engineering_charge}
\end{centering}
\end{figure}

A closer look at the surface of MspA reveals why it is a tranmembrane porin.  Figure \ref{fig:mspa_engineering_charge} shows surface charges and hydrophobicity, colored according to the scheme developed by Hagemans \textit{et al.} \citep{Hagemans2015}.  Hydrophobic amino acids are in yellow, while negative charges are red and positive charges are blue.  White is neutral.  The figure shows that the outer surface of MspA has a $\approx$ \SI{4}{\nm} barrel of hydrophobic amino acids on the surface.  This sits in the lipid bilayer, which is also hydrophobic.  When MspA inserts into a lipid bilayer, this bottom half will insert, ensuring that pores introduced on a particular side of the membrane always have the same orientation.  The middle and right panels of Figure \ref{fig:mspa_engineering_charge} show a cutaway view of the interior of MspA.  Wild-type MspA has a negatively-charged constriction, while M3-MspA has a neutral constriction, and several added positive charges in the vestibule.

It is worth noting that ``M2" is the standard nomenclature used for a commonly used MspA mutant which can be found in the literature \citep{Butler2008,Manrao2012,Schreiber2013,Derrington2015}.  The designation of the second MspA mutant (Figure \ref{fig:mspa_engineering}, right) as ``M3" is so far unique to this dissertation, as this mutant does not appear in the literature, except in the work published by our group \citep{Lu2015,Fleming2017} which is described in Chapter \ref{dna_thermal_motion_mspa}.


\section{Electrical measurements of the MspA nanopore}

Nanopore experiments are measurements of a single molecule, and this is ensured by the experimental setup.  A lipid bilayer of PC is formed across a $\sim$ \SI{20}{\um} aperture in a Teflon partition that separates two volumes of buffered KCl solution, each contacted by an electrode.  A single nanopore is allowed to insert into this lipid bilayer, and forms the only conductive path through the membrane.

\subsection{Experimental setup}

A schematic of the experimental setup is shown in Figure \ref{fig:setup}.  Silver / silver chloride [Ag/AgCl] electrodes make contact with an aqueous electrolyte on either side of a lipid bilayer membrane.  A single MspA nanopore is inserted into this membrane.  The electrodes are connected to a sensitive current amplifier, in this work, the Axopatch 200B (Molecular Devices, Inc.).  A simplified version of the current amplifier is also shown schematically in Figure \ref{fig:setup}, where the current amplification comes from the fact that whatever current flows through the nanopore is the same current that flows through the feedback resistor, $R_f$.  Conceptually, this means that even a small current (\SI{1}{\pA}) can cause a measurable voltage (\SI{1}{\mV}) at the output if the value of $R_f$ is large enough (\SI{1}{\giga\ohm}).

\begin{figure}[h]
\begin{centering}
\includegraphics[width=0.9\textwidth]{figures/experiment_setup.pdf}
\caption[Experimental setup]{Schematic of the experimental setup.  Ag/AgCl electrodes make contact with a buffered solution of potassium chloride [KCl].  A lipid bilayer of PC blocks the passage of ions from one electrode to the other.  The only current which can cross the membrane is the ionic current through the single MspA nanopore.  The current is measured using a current amplifier.  The simplified concept of current amplification is sketched on the right.  A bias is applied across the membrane, and the picoampere current which flows through the nanopore also flows through the large feedback resistor, $R_f$.  The output voltage is then proportional to the current through the nanopore, but larger by a factor of $R_f$.}
\label{fig:setup}
\end{centering}
\end{figure}

Note that the electrodes are connected to the experiment in such a way that a positive applied voltage, $V_{bias}$, means that the ``trans" side of the membrane (Figure \ref{fig:setup}) is positive with respect to the ``cis" side, where analyte molecules are typically introduced.  This means that at positive voltage bias, negatively-charged DNA will be drawn through the nanopore, from the cis side to the trans side.

\subsection{Electronics considerations}

Using the Axopatch 200B in ``whole cell" mode employs a feedback resistor, $R_f$, of \SI{500}{\mega\ohm}, which means that a \SI{1}{\pA} current shows up as an output voltage of \SI{0.5}{\mV}.  Following this amplification by another 20-fold amplification (the headstage accounts for a factor of two, and setting the output gain $\alpha=10$ on the Axopatch 200B amplifies by a further factor of ten) results in a signal of \SI{10}{\mV} per \SI{1}{\pico\ampere} of ionic current.  The analog-to-digital converter is a Digidata 1440a from Molecular Devices, Inc., with 16-bit conversion over a range of \SI{-10}{\volt} to $+$\SI{10}{\volt}, meaning that the signal is quantized in \SI{0.3}{\mV} steps.  With a digitization noise of less than \SI{1}{\mV}, this means that a \SI{10}{\milli\volt} signal, corresponding to \SI{1}{\pico\ampere} of ionic current, is readily measurable.  This is quite a feat, considering that \SI{1}{\pA} is only about \num{60} ions every \SI{10}{\us}, and the current can be sampled as fast as once every \SI{4}{\us}.

Typically, ionic current is recorded at \SI{100}{\kHz} after hardware filtering at \SI{10}{\kHz}.  The Axopatch 200B applies a voltage bias across the membrane, and this is also recorded at \SI{100}{\kHz}.

\subsection{Current as a function of applied voltage}

The simplest experiment that can be performed is a measurement of ionic current through the open nanopore, without any analyte molecules present.  Open pore measurements of M2-MspA are shown in Figure \ref{fig:iv_mspa}.  At room temperature ($\sim$ \SI{23}{\degreeCelsius}) in \SI{1}{\Molar} KCl, \SI{25}{\milli\Molar} phosphate buffer, at pH \num{8.00}, the conductance of M2-MspA is roughly \SI{2}{\nano\siemens}.  However, as shown in Figure \ref{fig:iv_mspa}a, the current-versus-voltage curve is not linear.  The characteristic larger conductance at negative applied biases, as well as the slight nonlinear dip in conductance between $\sim$ \SI{100}{\mV} and $\sim$ \SI{180}{\mV}, reveals the pore's orientation in the membrane.  This current-versus-voltage curve is recorded as the initial step of each new experiment, in order to ensure that the pore orientation is as expected.

\begin{figure}[h]
\begin{centering}
\includegraphics[width=\textwidth]{figures/mspa_iv.pdf}
\caption[MspA current versus voltage]{Measurements of a single, upright, open M2-MspA nanpore without any analyte molecules.  (a) Current versus voltage relation measured in \SI{1}{\Molar} KCl, \SI{25}{\milli\Molar} phosphate buffer, pH \num{8.00}, at room temperature.  (b) Current is plotted as a function of time.  At \num{3} seconds, the applied voltage is switched from $+$\SI{100}{\milli\volt} to \SI{-100}{\milli\volt}.  It can be seen that the nanopore spontaneously closes periodically at \SI{-100}{\milli\volt} bias.  This is referred to as ``gating."}
\label{fig:iv_mspa}
\end{centering}
\end{figure}

Figure \ref{fig:iv_mspa}b shows \num{6} seconds of a current versus time trace.  The first \SI{3}{\s} are at $+$\SI{100}{\mV} applied bias across the membrane, while the last \SI{3}{\s} are at $-$\SI{100}{\mV}.  It can be seen that the current is very stable and quiet at $+$\SI{100}{\mV}.  At $-$\SI{100}{\mV}, the pore occasionally closes spontaneously.  This is referred to as ``gating," and it was also observed by Butler \textit{et al.} for M2-MspA \citep{Butler2008}.  Gating at large negative applied bias is another confirmation that the nanopore is inserted in the correct orientation; however, gating is not as reliable a factor as the current-versus-voltage curve.


\subsection{Effect of temperature}

The amount of ionic current passing through an MspA nanopore at constant voltage depends on temperature.  Measurements of current versus voltage at constant temperature are shown in Figure \ref{fig:mspa_current_temp}a, at temperatures from \SI{5}{\degreeCelsius} to \SI{60}{\degreeCelsius} in steps of \SI{5}{\degreeCelsius}.  Figure \ref{fig:mspa_current_temp}b shows the same data plotted as conductance versus voltage.  The non-linear relationship of current with voltage is clearly visible, and makes the nanopore different from a simple resistor.

\begin{figure}[h]
\begin{centering}
\includegraphics[width=\textwidth]{figures/mspa_temp.pdf}
\caption[MspA current versus temperature]{Measurements of the effect of temperature on the open pore current of M3-MspA.  (a) The current is plotted as a function of voltage at temperatures from \SI{5}{\degreeCelsius} to \SI{60}{\degreeCelsius} in steps of \SI{5}{\degreeCelsius}.  (b) The same data are plotted as conductance versus voltage.  Note that for an ideal resistor, the conductance would be constant as a function of voltage.}
\label{fig:mspa_current_temp}
\end{centering}
\end{figure}

What exactly is responsible for this dependence of open pore conductance on temperature?  An initial hypothesis might be that the change in the viscosity of the electrolyte solution with temperature should contribute.  For ions with valence $z=1$, like potassium and chloride ions,

Diffusion, Stokes-Einstein relation:

\begin{equation}
D = \frac{k_B T}{6\pi\eta r}
\label{eqn:conductivity}
\end{equation}

Electrical mobility:

\begin{equation}
\frac{D}{\mu} = \frac{k_B T}{q}
\label{eqn:einstein_mobility}
\end{equation}

Ionic mobility:

\begin{equation}
\mu = \frac{e}{6 \pi \eta r}
\label{eqn:mobility}
\end{equation}

Conductivity, sum of both ions:

\begin{equation}
\lambda = F(\mu_1 + \mu_2)
\label{eqn:conductivity}
\end{equation}

For two ions, in terms of solvent viscosity:

\begin{equation}
\lambda = \frac{1}{\eta} \frac{Fe}{6\pi} \left( \frac{1}{r_1}+\frac{1}{r_2} \right)
\label{eqn:conductivity_eta}
\end{equation}

\begin{equation}
\lambda(T) = \lambda(T_0) \left[ \frac{\eta(T_0)}{\eta(T)} \right]
\label{eqn:conductivity_temp}
\end{equation}

and since conductance is proportional to this ionic conductivity, the conductance, $G$, is

\begin{equation}
G(T) = G(T_0) \left[ \frac{\eta(T_0)}{\eta(T)} \right]
\label{eqn:conductance_temp}
\end{equation}

This is what is plotted in Figure \ref{fig:mspa_current_temp_viscosity}.

\begin{figure}[h]
\begin{centering}
\includegraphics[width=\textwidth]{figures/mspa_temp_viscosity.pdf}
\caption[MspA current versus temperature explained by viscosity]{The viscosity of water is responsible for the dependence of M3-MspA open pore current on temperature.  On the left, the black circles are measurements of the viscosity of water taken from Kestin \textit{et al.} \cite{Kestin1978}.  The black line is an exponential fit, the equation of which is shown.  The right panel shows data measured in this work (black circles).  Each circle is the average conductance measured for open pore M3-MspA at 50mV bias.  The black line is Equation \ref{eqn:conductance_temp}, and it is scaled to be equal to the conductance measured at 25°C.  Evidently, the temperature dependence of the open pore conductance is captured exactly by accounting for the change in the viscosity of water with temperature.}
\label{fig:mspa_current_temp_viscosity}
\end{centering}
\end{figure}

\subsection{Molecules block current through the MspA nanopore}

ssDNA 100mer and 5kb, along with maybe ATP alone

\begin{figure}[h]
\begin{centering}
\includegraphics[width=\textwidth]{figures/ATP_event_basics.pdf}
\caption[Current blockage events in the MspA nanopore]{Ionic current through M2-MspA is plotted as a function of time.  Data are low-pass filtered at 10kHz.  Blockages of the open pore current signify single molecules in the nanopore.  Here the blockages are caused by ATP molecules at 2mM concentration.  Current blockage ``events" can be characterized by a mean current blockage and a duration.}
\label{fig:mspa_event_basics}
\end{centering}
\end{figure}

\begin{figure}[h]
\begin{centering}
\includegraphics[width=\textwidth]{figures/5kb_ssDNA_scatter.pdf}
\caption[Scatter plot of 5.3kb ssDNA in MspA]{A scatter plot of several hundred events of 5.3kb ssDNA translocating through M3-MspA is shown in the center.  Small regions from the scatter plot have been selected, and the events contained in those regions are plotted, one on top of the other, in the four corner plots.  The applied voltage bias is 180mV.}
\label{fig:mspa_5kb_scatter}
\end{centering}
\end{figure}

\begin{figure}[h]
\begin{centering}
\includegraphics[width=0.7\textwidth]{figures/100mer_ssDNA_scatter.pdf}
\caption[Scatter plot of 100mer ssDNA in MspA]{Scatter plot of just over 5700 events of 100mer ssDNA translocating through M3-MspA.  The applied voltage bias is 180mV.  Events are generally shorter compared to the 5.3kb ssDNA events in Figure \ref{fig:mspa_5kb_scatter}.  There is an edge at around \SI{0.1}{ms} to the left of which no events occur.  This limit corresponds to the shortest events detectable when using a 10kHz low-pass filter in hardware.}
\label{fig:mspa_100mer_scatter}
\end{centering}
\end{figure}

\subsection{Current noise of the open MspA nanopore}

\begin{figure}[h]
\begin{centering}
\includegraphics[width=\textwidth]{figures/mspa_noise_intro.pdf}
\caption[Current noise in the MspA nanopore]{On the left is plotted a \SI{10}{\s} trace of ionic current through M2-MspA with no applied voltage bias.  The current is \SI{0}{\pA} on average, but there is some spread to the values, shown in the current histogram.  The thin gray line is a Gaussian fit, showing that the current noise is Gaussian distributed.  On the right, the current noise power spectral density is plotted as a function of frequency.  The current noise here is white at frequencies below \SI{1}{\kHz}, where there is a gradual increase due to capacitance, followed by a falloff due to the \SI{10}{\kHz} 4-pole Bessel hardware filter.  For reference, the calculated Johnson noise is shown as a dashed gray line.}
\label{fig:mspa_noise_intro}
\end{centering}
\end{figure}