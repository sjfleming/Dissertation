\begin{savequote}[75mm]
Apply your scrutiny whenever the sun's rays are let in and pour their light through a dark room: you will see many minute particles mingling in many ways throughout the void ... such turmoil indicates that there are secret and unseen motions also hidden in matter.
\qauthor{Lucretius, \textit{On the Nature of Things}}
\end{savequote}

\chapter{Introduction}
\label{introduction}

The work described in this dissertation is focused on single-molecule measurement.  The measuring instrument of choice is a biological molecule, a protein called a \textit{nanopore}, and it is used here to measure individual molecules of DNA.  This dissertation is concerned with characterizing physical properties of the nanopore system as a measurement device, and understanding its behavior with an eye toward precise control over the motion of single molecules under study.

\section{Motivation and context}

Individual molecules have been a subject of inquiry in modern physics from the time of early work on gases by John Dalton, Amedeo Avogadro, and others at the turn of the 19\textsuperscript{th} century.  Before that, the idea that matter might be made of small, indivisible units was a subject of philosophical discussion in ancient India and Greece.  In the latter part of the 19\textsuperscript{th} century, Boltzmann's work on statistical mechanics hinged on the existence of molecules, but his work was not fully appreciated during his lifetime for that very reason.  It was finally the work of Albert Einstein and Marian Smoluchowski in 1905-1906, and subsequent experiments by several others, notably Jean Baptiste Perrin in 1908, that confirmed ``the discontinuous structure of matter," in the words of Perrin's 1926 Nobel Prize in Physics \cite{Nobel1926}.

Statistical mechanics and the molecular kinetic theory of gases allowed physicists to measure properties of individual molecules indirectly, by making measurements of equilibrium properties of macroscopic systems.  In 1911, Charles Thomson Rees Wilson's invention of the cloud chamber enabled the direct visualization of individual ionized particles.

Single molecule measurements
Other techniques for single-molecule measurements

\section{The ``Nanopore"}

\begin{enumerate}
\item Nanopores as single-molecule measurement instruments
\item Measuring ionic current through a nanopore
\item Spatial localization of the resistance
\item Electrophoresis and force on molecules near nanopore
\item Physical blockage of nanopore causes change in resistance
\item Brief mention of signal-to-noise, comparison with other techniques
\end{enumerate}

\section{Using nanopores for DNA sequencing}



\section{Organization of thesis}

In this thesis... \cite{Bryant1994} \cite{Vercoutere2001}