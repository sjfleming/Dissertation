\begin{savequote}[75mm]
blank
\qauthor{blank}
\end{savequote}

\chapter{Introduction}
\label{introduction}

The work described in this dissertation is focused on single-molecule measurement.  The measuring instrument of choice is a biological protein called a \textit{nanopore}, and it is used here to measure individual molecules of DNA.  This dissertation is concerned with characterizing physical properties of the nanopore system as a measurement device, and understanding its behavior with an eye toward precise control over the motion of single molecules under study.

\section{Motivation and context}


Single molecule measurements
Other techniques for single-molecule measurements

\section{The ``Nanopore"}

\begin{enumerate}
\item Nanopores as single-molecule measurement instruments
\item Measuring ionic current through a nanopore
\item Spatial localization of the resistance
\item Electrophoresis and force on molecules near nanopore
\item Physical blockage of nanopore causes change in resistance
\item Brief mention of signal-to-noise, comparison with other techniques
\end{enumerate}

\section{Using nanopores for DNA sequencing}



\section{Organization of thesis}

In this thesis... \cite{Bryant1994} \cite{Vercoutere2001}