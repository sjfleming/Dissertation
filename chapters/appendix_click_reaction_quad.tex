\chapter{Stopping helicase movement}
\label{click_reaction_quad}

How to stop the motion of a helicase, even when ATP is around.

\section{Click reaction}

Reversing the direction of DNA mid-strand.

\section{G-quadruplex}

Using a G-quadruplex knot to stop the helicase.

\subsection{General scheme}

\subsection{Details of assay protocol}

The protocol is summarized in Table \ref{table:quad_gel_assay_summary}.  The total Mg\textsuperscript{2+} concentration is 2.7mM.  The total ATP concentration (when added) is 5mM.  The total volume is 10$\upmu$L, which facilitates gel loading.

\begin{table}[h]
\begin{center}
\captionsetup{justification=centering}
\caption[Stopping helicase: G-quadruplex assay summary]{Final amounts used in G-quadruplex assay.} 
\label{table:quad_gel_assay_summary}
\begin{tabularx}{0.5\textwidth}{|X|X|}
\hline
DNA & 1 pmol \\
Streptavidin & 30 pmol \\
Biotin & 1 nmol \\
Helicase & 30 pmol \\
\hline
\end{tabularx}
\end{center}
\end{table}

\begin{figure}[h]
\begin{centering}
\includegraphics[width=0.85\textwidth]{figures/quad_gel_assay_dna.pdf}
\caption[Stopping helicase: G-quadruplex and control sequences]{The so-called ``thrombin aptamer" G-quadruplex is the sequence \texttt{GGTTGGTGTGGTTGG}.  Shown are the DNA sequences of molecules used in the H1 helicase assay, with and without the thrombin aptamer sequence.  Both strands end in a biotinylated dT nucleotide.  Duplexes were synthesized by Integrated DNA Technologies, Inc.}
\label{fig:quad_gel_assay_dna}
\end{centering}
\end{figure}

\begin{figure}[h]
\begin{centering}
\includegraphics[width=0.6\textwidth]{figures/quad_gel_assay.pdf}
\caption[Stopping helicase: G-quadruplex DNA gel assay]{Gel electrophoresis results of an assay which demonstrates that helicase H1, in the presence of ATP, can move past the thrombin G-quadruplex sequence.  Lane 4 is the positive control, where the very faint band between 25bp and 50bp that matches the position of the band in Lane 1, demonstrates that the H1 helicase can displace streptavidin from a biotinylated oligo.  Lane 8 is the experiment, and the band near the 50bp marker that matches the position of the band in Lane 5, demonstrates that the H1 helicase can displace streptavidin from a biotinylated oligo, even when a G-quadruplex blocks its path.}
\label{fig:quad_gel_assay}
\end{centering}
\end{figure}

Assay, and results.
