\begin{savequote}[75mm]
When asked if his painting was inspired by nature, Jackson Pollock replied, ``I am nature."
\end{savequote}

\chapter{The nanopore as a DNA sequencing technology}
\label{dna_sequencing}

Some data obtained from nanopore DNA sequencing experiments has already been presented to provide evidence that thermal motion of ssDNA contributes to the ionic current signal.  This chapter briefly presents recent developments which enabled nanopores to be used as a single-molecule DNA sequencing technology, in order to provide a background for the experiments covered in the rest of this work.

\section{Inspired by nature}

The initial idea to use nanopores as single-molecule sensors for sequencing nucleic acids was independently proposed by Dave Deamer and George Church around 1989 \citep{Deamer2016,Branton2008}.  The inspiration was that the opening and closing of ion channels in biology is often dependent on small peptide domains blocking the channel.  If small peptide domains can block the channel, so could nucleic acids, and the different sizes of the different nucleic acids could potentially cause different current blockages, enabling a direct readout of the sequence \citep{Branton2008}.

\subsection{Developments from 1990 - 2010}



\section{Measurements of DNA sequence and its effect on ionic current}



\section{Observations from the data}



\section{A theory to explain the effect}

M2-MspA introduced the key mutations D90N/D91N/D93N, which changed the amino acids lining the narrowest constriction from negatively-charged aspartic acid residues to neutral asparagine residues.

Amino acids can interact with DNA bases.  By searching through 129 structures of protein-DNA complexes in the Protein Data Bank, Luscombe \textit{et al.} \citep{Luscombe2001} found that asparagine interacts favorably with adenine, while arginine strongly interacts with guanine.