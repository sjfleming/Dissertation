\begin{savequote}[75mm]
When asked if his painting was inspired by nature, Jackson Pollock replied, ``I am nature."
\end{savequote}

Some data obtained in nanopore sequencing experiments has already been presented to provide evidence that thermal motion of ssDNA contributes to the ionic current signal measured in nanopore experiments.  Here we briefly cover the historical development of nanopores as a single-molecule DNA sequencing technology, in order to provide a background for the experiments covered in the rest of this work.

\chapter{The nanopore as a DNA sequencing technology}
\label{dna_sequencing}

\subsection{Initial inspiration}

\section{Slowing down DNA}
\begin{enumerate}
\item Signal and noise
\item Simple attempts at remedies and why they fail
\begin{enumerate}
\item Measuring faster
\item Slowing down DNA with viscosity
\end{enumerate}
\end{enumerate}

\subsection{Developments from 1990 - 2010}

\section{Measurements of DNA sequence and its effect on ionic current}

\section{Observations from the data}

\section{A theory to explain the effect}

M2-MspA introduced the key mutations D90N/D91N/D93N, which changed the amino acids lining the narrowest constriction from negatively-charged aspartic acid residues to neutral asparagine residues.

Amino acids can interact with DNA bases.  By searching through 129 structures of protein-DNA complexes in the Protein Data Bank, Luscombe \textit{et al.} \citep{Luscombe2001} found that asparagine interacts favorably with adenine, while arginine strongly interacts with guanine.