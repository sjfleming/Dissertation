\begin{savequote}[75mm]
Blank.
\qauthor{Blank.}
\end{savequote}

\chapter{Thermal motion of ssDNA in the MspA nanopore}
\label{dna_thermal_motion_mspa}

Intro and motivation

\section{The capture and escape experiment}

\begin{figure}[h]
\begin{centering}
\includegraphics[width=\textwidth]{figures/NA-ssDNA_fluctuation_in_MspA.pdf}
\caption[The idea of capture and escape]{On the left hand side, a NeutrAvidin/ssDNA complex is shown trapped in MspA.  The membrane is shown in gray, and a voltage bias across the membrane results in an electrophoretic force pulling the negatively-charged ssDNA into the MspA pore.  Thermal motion causes non-equilibrium position fluctuations.  On the right, the potential energy profile is shown schematically.  Thermal fluctuations of a distance greater than $L$ leads to escape of the complex.}
\label{fig:capture_escape_idea}
\end{centering}
\end{figure}

The NeutrAvidin protein is too large to enter the vestibule of MspA, and so the complex cannot translocate.

\subsection{Theory of drift-diffusion and first-passage}

Equations used and model being fit.

\section{Experimental setup}



\section{Results}

\begin{figure}[h]
\begin{centering}
\includegraphics[width=0.9\textwidth]{figures/NA-ssDNA_capture_and_escape_event.pdf}
\caption[Capture and escape experiment schematically]{Implementation of the capture and escape experiment.}
\label{fig:capture_escape_event}
\end{centering}
\end{figure}

\begin{figure}[h]
\begin{centering}
\includegraphics[width=0.8\textwidth]{figures/polyA_escape_histograms.pdf}
\caption[Measured escape times of poly(dA) ssDNA from MspA]{The histogram of poly(dA) escape data at \SI{60}{\mV}, \SI{25}{\celsius}, plotted with linear and logarithmic x- and y-axis.  Data points are shown in black.  An exponential fit with the same time constant of \SI{57.8}{\ms} is shown as the green line in each plot.  The exponential fits well to the shortest $\sim 90\%$ of events, denoted by the green shaded region.}
\label{fig:escape_times}
\end{centering}
\end{figure}

\begin{figure}[h]
\begin{centering}
\includegraphics[width=0.8\textwidth]{figures/cell.pdf}
\caption[Measured escape times of homopolymer ssDNA from MspA]{.}
\label{fig:escape_times}
\end{centering}
\end{figure}

\begin{figure}[h]
\begin{centering}
\includegraphics[width=0.8\textwidth]{figures/cell.pdf}
\caption[Current blockages for homopolymer ssDNA]{.}
\label{fig:homopolymer_blockages}
\end{centering}
\end{figure}

\begin{figure}[h]
\begin{centering}
\includegraphics[width=0.8\textwidth]{figures/cell.pdf}
\caption[Current fluctuations for poly(dC) ssDNA]{.}
\label{fig:polyC_fluctuations}
\end{centering}
\end{figure}

\begin{figure}[h]
\begin{centering}
\includegraphics[width=0.8\textwidth]{figures/cell.pdf}
\caption[Current fluctuations for all homopolymer ssDNA]{.}
\label{fig:homopolymer_fluctuations}
\end{centering}
\end{figure}

\section{Discussion}

\subsection{Thermal motion of the ssDNA in MspA}

\begin{figure}[h]
\begin{centering}
\includegraphics[width=0.8\textwidth]{figures/cell.pdf}
\caption[Position fluctuations of ssDNA in MspA]{.}
\label{fig:position_fluctuations}
\end{centering}
\end{figure}

\begin{figure}[h]
\begin{centering}
\includegraphics[width=0.8\textwidth]{figures/cell.pdf}
\caption[Thermal motion averaging and sequencing data]{.}
\label{fig:thermal_motion}
\end{centering}
\end{figure}

\subsection{Correlations}

\begin{figure}[h]
\begin{centering}
\includegraphics[width=0.8\textwidth]{figures/cell.pdf}
\caption[Correlation between force and current]{.}
\label{fig:force_current_correlation}
\end{centering}
\end{figure}

\begin{figure}[h]
\begin{centering}
\includegraphics[width=0.8\textwidth]{figures/cell.pdf}
\caption[Correlation between diffusion and noise]{.}
\label{fig:diffusion_noise_correlation}
\end{centering}
\end{figure}

\subsection{Implications}

\begin{figure}[h]
\begin{centering}
\includegraphics[width=0.8\textwidth]{figures/cell.pdf}
\caption[Thermal motion averaging with voltage]{Model of position fluctuations of ssDNA in MspA, and its effects on signal averaging, as a function of applied bias voltage.}
\label{fig:thermal_motion_projections}
\end{centering}
\end{figure}
