\begin{savequote}[75mm]
Imagine living in a world where a Richter 9 earthquake raged continuously ... At the scale of proteins, Brownian motion is even more furious.
\qauthor{George Oster and Hongyun Wang \citep{Oster2003}}
\end{savequote}

\chapter{Thermal motion of ssDNA in the MspA nanopore}
\label{dna_thermal_motion_mspa}

Several experiments have been carried out in order to quantify the thermal motion of analyte molecules in a nanopore.  ssDNA's passage through a nanopore can be modeled as a drift-diffusion process, where the electrophoretic drift is caused by the applied voltage bias, and diffusion is caused by thermal motion.  The drift can be quantified in terms of an effective charge, while the diffusion can be quantified in terms of an effective diffusion constant.  Therefore, measurement of the effective charge and diffusion constant of ssDNA in MspA provides a complete description of the ssDNA motion within the context of a drift-diffusion model.  Due to the extreme speed with which ssDNA translocates through an MspA nanopore, it is difficult to quantify thermal motion using measurements of free translocation.

\section{The capture and escape experiment}

The experimental technique used here involves tethering a large protein to the ssDNA to prevent its translocation through the nanopore.  The work in this chapter is the subject of two papers \citep{Lu2015,Fleming2017}, and the experimental technique is based on preceding work with DNA in $\alpha$-hemolysin nanopores \citep{Wiggin2008}, where the protein Avidin was used as a molecular stop to arrest ssDNA translocation.  Other methods used to measure the effective charge of DNA in $\alpha$-hemolysin nanopores have included unzipping dsDNA hairpins \citep{Sauer-Budge2003,Mathe2004,Lakatos2005} and monitoring the escape of DNA hairpins \citep{Wanunu2008,Lathrop2010}.

\begin{figure}[h]
\begin{centering}
\includegraphics[width=\textwidth]{figures/NA-ssDNA_fluctuation_in_MspA.pdf}
\caption[The idea of capture and escape]{The left panel shows a NeutrAvidin/ssDNA complex trapped in MspA.  The membrane is in gray, and a voltage bias across the membrane results in an electrophoretic force pulling the negatively-charged ssDNA into the MspA pore.  The large NeutrAvidin protein cannot fit through MspA.  Thermal motion causes non-equilibrium position fluctuations.  On the right, the potential energy profile is shown schematically.  Thermal fluctuations of a distance greater than $L$ lead to escape of the complex.}
\label{fig:capture_escape_idea}
\end{centering}
\end{figure}

The idea is depicted in Figures \ref{fig:capture_escape_idea} and \ref{fig:capture_escape_event}.  Here, the protein NeutrAvidin is used as a molecular stop that binds to the 3' end of ssDNA via a 3'-biotin (see Appendix \ref{idt_dna} for the chemical structure).  The NeutrAvidin protein is too large to enter the vestibule of MspA, and so the complex cannot translocate.  However, an electrophoretic force proportional to the effective charge and applied voltage bias pulls the ssDNA through the nanopore, effectively trapping the complex in the pore, as shown in the potential energy diagram on the right.  Thermal motion leads to position fluctuations, depicted in Figure \ref{fig:capture_escape_idea} as movement of the ssDNA within the pore.  Eventually, if the bias voltage is appropriately low, thermal motion will lead to escape of the complex from the nanopore.  Using $x$ to denote the position of the ssDNA within the nanopore's constriction (see Figure \ref{fig:capture_escape_idea}), escape happens for $x \geq L$, $L$ being the length of ssDNA that sticks through the nanopore at equilibrium at \SI{160}{\mV} bias.

\subsection{Drift-diffusion model and first-passage}

The stochastic dynamics of the NeutrAvidin-ssDNA complex in MspA can be modeled as a one dimensional drift-diffusion process that obeys the Smoluchowski equation.  The probability that MspA's narrowest constriction is at location $x$ along the ssDNA at time $t$, having started at position $x_0$, written $p(x,t|x_0)$, obeys

\begin{equation}
\pard{p(x,t|x_0)}{t} = - \frac{\partial}{\partial x} \left[ \frac{F(x)}{\gamma} p(x,t|x_0) \right] + D \pardd{p(x,t|x_0)}{x}
\label{eqn:drift_diffusion}
\end{equation}

\noindent
where $F(x)$ is the electrophoretic force on the ssDNA as a function of position, and $\gamma$ is a drag coefficient, the inverse of the mobility.  $\gamma$ can be rewritten in terms of the diffusion constant using the Einstein relation: $\gamma = \frac{k_B T}{D}$.  Making this substitution,

\begin{equation}
\pard{p(x,t|x_0)}{t} = - \frac{D}{k_B T} \frac{\partial}{\partial x} \left[ F(x) p(x,t|x_0) \right] + D \pardd{p(x,t|x_0)}{x}
\label{eqn:drift_diffusion2}
\end{equation}

The force $F(x)$ depends on the applied bias voltage and the effective charge of the ssDNA.  In the simplest case, we can neglect entropic forces (see Appendix \ref{first_passage} for a discussion).  The remaining forces are the electric force on the DNA due to its charge and the applied potential, and an opposing force of viscous drag caused by electroosmotic flow.  The total force measured is the sum of these forces, both of which are proportional to the applied bias voltage:

\begin{equation}
F(x) = F_{\text{bare}} + F_{\text{drag}} = (\sigma_{\text{bare}} + \sigma_{\text{drag}}) V = \sigma V
\label{eqn:simple_force}
\end{equation}

\noindent
where $V$ is the bias voltage and $\sigma$ is the effective charge per unit length.  In this case,

\begin{equation}
\frac{1}{D} \pard{p(x,t|x_0)}{t} = - \frac{\sigma V}{k_B T} \pard{p(x,t|x_0)}{x} + \pardd{p(x,t|x_0)}{x}
\label{eqn:drift_diffusion3}
\end{equation}

The only free parameters in Equation \ref{eqn:drift_diffusion3} are the effective charge per unit length, $\sigma$, and the diffusion constant, $D$.  Even in the case where entropic effects are considered in the force term, these can be estimated without introducing additional free parameters.  For a discussion, see Appendix \ref{first_passage}.  Using a reflecting boundary condition at $x=0$ and an absorbing boundary condition at $x=L$, signifying escape of the ssDNA from MspA, Equation \ref{eqn:drift_diffusion2} can be solved numerically.  In the special case where the force is adequately described by Equation \ref{eqn:simple_force}, then Equation \ref{eqn:drift_diffusion3} can be solved analytically, as in Appendix \ref{first_passage}.

The constriction of MspA initially starts on the domain $x \in (0,L)$ along the ssDNA, but the absorbing boundary means that, as time $t$ increases, the probability that the ssDNA is still in the nanopore, $\int_0^L p(x,t|x_0) \, dx < 1$.  The probability density for escape of ssDNA from MspA at a given time is given by the probability flux from the boundary (at $x=L$):

\begin{equation}
f(t,x_0) = -\frac{d}{dt} \int_0^L p(x,t|x_0) \, dx
\label{eqn:escape_time_dist}
\end{equation}

Finally, the average escape time, $\langle \tau \, \rangle$, can be calculated as

\begin{equation}
\langle \tau \, \rangle = \int_0^L \left( w(x_0) \, \int_0^{\infty} t \, f(t|x_0) \, dt \right) \, dx_0
\label{eqn:mean_escape_time}
\end{equation}

\noindent
where $w(x_0)$ is the probability distribution function for the initial position.  The analytical solution for the simple case where $F(x) = \sigma V$ is discussed in Appendix \ref{first_passage}.

The mean escape time, along with the escape time distribution, can be directly compared to experiment, and can be used to obtain an estimate for the values of $D$ and $\sigma$.

\section{Experimental methods}

\begin{figure}[h]
\begin{centering}
\includegraphics[width=0.9\textwidth]{figures/experiment_setup_capture_escape.pdf}
\caption[Experimental setup for capture and escape]{Schematic diagram of the experimental setup for the capture and escape experiment.  Active voltage control is achieved by running a computer program which reads voltage and current data from a National Instruments DAQ and sets the command voltage by supplying an analog voltage output signal from the DAQ that goes to the external command input of the Axopatch 200B.  The Axopatch directly controls all voltages in the experiment.  Meanwhile, current and voltage data are recorded separately on the computer via the Digidata 1440A.}
\label{fig:setup_capture}
\end{centering}
\end{figure}

\subsection{Setup}

The experimental setup is similar to what was described in Section \ref{experimental_setup}.  However, the capture and escape experiment requires active voltage control.  The idea is to capture many individual molecules, and measure the time it takes them to escape from the nanopore at various low bias voltages.  The setup which makes active voltage control possible is shown in Figure \ref{fig:setup_capture}.  The computer records voltage and current outputs from the Axopatch 200B (Molecular Devices, Sunnyvale, CA) which are digitized by the Digidata 1440A (Molecular Devices, Sunnyvale, CA).  The computer simultaneously runs a program written in Python that monitors current and voltage from the Axopatch which are digitized by the NI-DAQ USB 6003 (National Instruments, Inc.), and sends commands back to the DAQ, which sets an analog voltage output that is fed into the external command input of the Axopatch.  The Axopatch then clamps the membrane voltage to the value specified by the external command.

\subsection{Sample preparation}

ssDNA was purchased from Integrated DNA Technologies, Inc. [IDT] with HPLC purification.  Sequences used in the work covered in this chapter are all 27 bases long, and include (dA)$_{27}$ [poly(dA)], (dC)$_{27}$ [poly(dC)], (dT)$_{27}$ [poly(dT)], and (dGdA)$_{13}$dG [poly(dGdA)], each with a 3' biotin (chemical details of biotin attachment in Appendix \ref{idt_dna}).  Sequences are listed in Appendix \ref{sequences}.

NeutrAvidin-ssDNA complexes are created by the slow, dropwise addition of a low concentration of biotinylated ssDNA (\SI{1}{\micro\Molar} in TE, pH 7.6) to a high concentration of NeutrAvidin (\SI{10}{\micro\Molar} in \num{1}x PBS), with rapid mixing.  Since each NeutrAvidin has four binding sites for biotin, this method ensures that most NeutrAvidins have either zero or one ssDNA molecule bound.  Free ssDNA is removed using a \SI{20}{\kilo\dalton} molecular weight cutoff size exclusion spin column.  This step also serves to buffer exchange the complexes into the buffer used in the experiment, which is \SI{1}{\Molar} KCl, \SI{10}{\m\Molar} HEPES, pH \num{8.00}.

\SI{10}{\pico\mole} of NeutrAvidin-ssDNA complexes are loaded into the cis side of the experiment, and the cis side is mixed well.  Active voltage control and current monitoring are implemented using a custom Python program to control the NI-DAQ.  The experiment is enclosed by a copper lid, and maintained at \SI{25}{\celsius} by a copper block enclosure heated and cooled by a Peltier device, controlled by a Newport temperature controller Model 3040.

\begin{figure}[h]
\begin{centering}
\includegraphics[width=0.8\textwidth]{figures/NA-ssDNA_capture_and_escape_event.pdf}
\caption[Capture and escape experiment schematically]{A typical capture and escape event showing measured current and applied voltage. The cartoon depicts snapshots of the complex in the pore that correspond to the data below.  The complex is captured at \SI{160}{\mV} and held for nearly \SI{200}{\ms}.  The applied voltage is then reduced to \SI{65}{\mV}.  After approximately \SI{670}{\ms}, the complex escapes, as indicated by the current's increase to its open pore value at \SI{65}{\mV}.  The period of time between voltage reduction and escape of the complex is called the ``escape time."  After escape, the voltage is increased to \SI{160}{\mV} to await the next molecule.}
\label{fig:capture_escape_event}
\end{centering}
\end{figure}

\section{Results}

\subsection{Escape time measurements}

In order to measure many individual molecules and gather statistics about their escape times, it is necessary to use a relatively high voltage to ensure a reasonable capture rate.  Here, \SI{160}{\mV} is used for this purpose.  Upon capturing a molecule, the voltage is decreased to a constant, low voltage until the molecule escapes.  This process is shown in Figure \ref{fig:capture_escape_event}, along with data recorded for one molecule of poly(dT).

\begin{figure}[!h]
\begin{centering}
\includegraphics[width=0.7\textwidth]{figures/polyA_escape_histograms.pdf}
\caption[Measured escape times of poly(dA) ssDNA from MspA]{The histogram of poly(dA) escape data at \SI{60}{\mV}, \SI{25}{\celsius}, plotted with linear and logarithmic x- and y-axis.  Data points are shown in black.  Error bars in each bin are $\sqrt{N}$.  An exponential fit with the same time constant of \SI{57.8}{\ms} is shown as the green line in each plot.  The exponential fits well to the shortest $\sim 90\%$ of events, denoted by the green shaded region.}
\label{fig:polyA_escape_times}
\end{centering}
\end{figure}

In order to measure short escape times, it is necessary to use the ``pipette capacitance compensation" on the Axopatch 200B, and to use a small lipid bilayer area in order that the capacitance can be fully compensated.  Without full capacitance compensation, large capacitive current spikes will obscure short escape events ($<$ \SI{2}{\ms}).

Measuring the escape times for many individual molecules provides a distribution of escape times.  The distribution measured for poly(dA) escaping against \SI{60}{\mV} applied bias is shown in Figure \ref{fig:polyA_escape_times}.  Shown in the figure are three ways of plotting the same data with linear and logarithmic axes.

The fit to a single exponential, which is the prediction for a first-passage process (see Appendix \ref{first_passage}), is shown in green.  An exponential is a good fit to at least 90\% of the events, while the number of events with long escape times slightly exceeds the predictions of a single exponential.  A chi-squared fit to an exponential ($\sim e^{-t/\tau}$) is performed by binning the data as shown and using error bars of $\sqrt{N}$ from Poisson statistics, where $N$ is the number of molecules in a given bin.  The ``average escape time" is defined as the best fit value of $\tau$, and error bars on $\tau$ represent one standard deviation obtained from the chi-squared fit.

\begin{figure}[h]
\begin{centering}
\includegraphics[width=\textwidth]{figures/homopolymer_escape_data.pdf}
\caption[Measured escape times of homopolymer ssDNA from MspA]{(a) Histograms of the escape times measured for the given NeutrAvidin-ssDNA complexes with \SI{60}{\mV} bias applied.  Error bars in each bin are $\sqrt{N}$.  At a given clamping voltage, the distribution of escape times is approximately exponential (solid lines) for all four ssDNA molecules, but the time constant differs.  This time constant is called the ``average escape time."  (b) Average escape time plotted as a function of bias voltage.  Error bars reflect parameter estimation uncertainty at one standard deviation.  Fits to a first-passage model are shown as dashed lines.}
\label{fig:escape_times}
\end{centering}
\end{figure}

Homopolymer ssDNA of each of the four nucleotides were chosen as analytes in order to measure the effective charge and diffusion constant of each, and to see if there were noticeable differences.  Figure \ref{fig:escape_times} shows the results of the same measurements described above for each of the oligonucleotides [oligos].  Homopolymer dG was not used in order to prevent G-quadruplex formation, and a repeat of (dGdA) was chosen instead.  As demonstrated in Figure \ref{fig:escape_times}a, the average escape times at \SI{60}{\mV} clamping voltage are different for each oligo.  In particular, poly(dT) shows substantially longer escape times than the others.  This trend holds for all clamping voltages used.

Figure \ref{fig:escape_times}b shows a summary of all the escape time data measured.  Each data point is the average escape time at a given voltage, measured by fitting an exponential to a histogram of escape times measured at that voltage.  Each data point represents between several hundred and over a thousand individual capture and escape events.  The vertical (time) axis is logarithmic, so differences between the different homopolymers 
span more than an order of magnitude at some voltages.  Changes of \SI{40}{\mV} in the bias voltage during escape lead to changes in the average escape time that span four orders of magnitude.

Fits to the first-passage model are obtained by performing a chi-squared fit of the numerical solution of Equation \ref{eqn:drift_diffusion2} to the data, allowing the parameters $\sigma$ and $D$ to vary.  The fitting procedure yields values for $\sigma$ and $D$ for each oligo, and these are tabulated in Table \ref{table:fit_values}.  The ``effective charge" is calculated as $Q_{eff} = \sigma L_b$, where the length per base for ssDNA, $L_b = $ \SI{0.5}{\nm} \citep{Smith1996,Chi2013}.

\begin{table}[h]
\caption{Effective charges, electrophoretic driving forces, and diffusion constants extracted from the data.  Parameters are estimated from fits to first-passage model.  Uncertainties reflect one standard deviation and are obtained from chi-squared fitting.}
\label{table:fit_values}
\sisetup{
table-number-alignment = center,
table-figures-integer = 2
}
\centering
\begin{tabular}{
l
S[separate-uncertainty,table-figures-uncertainty = 1]
S[separate-uncertainty,table-figures-uncertainty = 1]
S[separate-uncertainty,table-figures-uncertainty = 1]
}
\hline
{ssDNA Sequence}
& {Effective charge}
& {Driving force}
& {Diffusion constant} \\
{}
& {$Q_{eff} \,$ ($e^{-}/$base)}
& {$F$, at \SI{100}{\mV} (\si{\pico\newton})}
& {$D$ (\si{\micro\meter^2/\s})} \\ [0.5ex]
\hline \hline
\rule{0pt}{3ex}(dT)$_{27}$ & 0.62(2) & 19.7(4) & 53(10) \\ 
(dC)$_{27}$ & 0.56(1) & 17.9(3) & 83(17) \\
(dA)$_{27}$ & 0.53(2) & 17.1(5) & 56(17) \\
(dGdA)$_{13}$(dG) & 0.51(1) & 16.2(5) & 43(13) \\ [1ex]
\hline
\end{tabular}
\end{table}

\subsection{Current blockage measurements}

The capture and escape experiment yields much more information than escape times.  The current blockage caused by the presence of the homopolymer ssDNA is also recorded.  Figure \ref{fig:homopolymer_blockages} shows a composite recording of current versus time for four individual NeutrAvidin-ssDNA complexes held in M3-MspA at \SI{160}{\mV} bias, measured in separate experiments, along with the open pore current for M3-MspA.

\begin{figure}[h]
\begin{centering}
\includegraphics[width=0.55\textwidth]{figures/homopolymer_block_traces.pdf}
\caption[Current blockage traces for homopolymer ssDNA]{Recordings of current versus time for NeutrAvidin-ssDNA complexes, where the ssDNA is a 3'-biotinylated 27mer of either poly(dA), poly(dT), poly(dC), or poly(dGdA), as indicated.  Data are compiled from four separate experiments, each filtered at \SI{10}{\kHz}.  The scale bar for the time axis indicates that each segment lasts about \SI{25}{\ms}.  For reference, the open pore M3-MspA current is shown at the upper left.  Buffer is \SI{1}{\Molar} KCl, \SI{10}{\milli\Molar} HEPES, pH \num{8.00}, and temperature is controlled at \SI{25.00}{\celsius}.}
\label{fig:homopolymer_blockages}
\end{centering}
\end{figure}

Perhaps most immediately apparent is that the blockage levels are different depending on the identity of the nucleotides.  The homopolymers used in this work each block different amounts of current in M3-MspA.  Due to our ability to probe single molecules individually, we can create a histogram of these current blockage levels for many molecules.  Figure \ref{fig:homopolymer_blockage_hists} shows histograms of the mean blockage currents measured for all four oligos at two different bias voltages: \SI{60}{\mV} and \SI{160}{\mV}.  Current blockages are reported in terms of mean $I/I_0$, the ratio of the measured current with DNA blocking the pore to the measured current for the open pore.

\begin{figure}[!h]
\begin{centering}
\includegraphics[width=0.6\textwidth]{figures/homopolymer_block_hists.pdf}
\caption[Current blockage histograms for homopolymer ssDNA]{Histograms of mean blockage currents when different NeutrAvidin-tethered oligonucleotides are captured in the M3-MspA pore.  $I/I_0$ is the ratio of mean measured current, $I$, to open-pore current, $I_0$.  Each element in the histogram is an individual molecule.  (a) At \SI{60}{\mV} bias, the blockages are well separated, while at \SI{160}{\mV} bias (b), the distributions are broader, and show significant overlap.}
\label{fig:homopolymer_blockage_hists}
\end{centering}
\end{figure}

The spread in the measured current blockages greatly increases at higher voltages, leading to significant overlap of poly(dA), poly(dC), and poly(dT).  A careful look reveals that the histograms even become multimodal, with indications of a peak emerging at lower values of $I/I_0$.

Figure \ref{fig:polyC_hists} shows a detailed look at the emergence of this deeper-blockage peak as a function of applied voltage.  While the peak is not so apparent in Figure \ref{fig:homopolymer_blockage_hists} due to the fact that the histograms there are mean blockages, the histograms plotted in Figure \ref{fig:polyC_hists} are histograms of the full current trace, sampled at \SI{100}{\kHz} after hardware filtering at \SI{10}{\kHz}, and median filtering with a 100-point window post-acquisition.  Every data point contributes one point to the histogram.  Histograms are plotted at each voltage horizontally, with the current ($I/I_0$) on the vertical axis.

\begin{figure}[h]
\begin{centering}
\includegraphics[width=0.7\textwidth]{figures/polyC_current_voltage.pdf}
\caption[Current histograms for poly(dC) ssDNA versus voltage]{Current blockage histograms for \num{30} molecules of poly(dC) are plotted at \SI{5}{\mV} increments from \SI{45}{\mV} to \SI{170}{\mV}.  The histograms show that the mean $I/I_0$ increases with applied voltage, and that the distribution is Gaussian at voltages below about \SI{100}{\mV}.  At higher bias voltages, peaks appear at lower values of $I/I_0$, and the current distribution ceases to be Gaussian.  Two distributions are singled out, and short segments of $I/I_0$ versus time are plotted in insets.}
\label{fig:polyC_hists}
\end{centering}
\end{figure}

At voltages below about \SI{100}{\mV}, the current is Gaussian distributed around a single value.  However, as the voltage increases, the distribution becomes asymmetrical, with a distinct peak or peaks emerging at lower values of $I/I_0$.  Two voltages are singled out in the insets: \SI{90}{\mV} and \SI{170}{\mV}.  The currents plotted in the insets correspond to the y-axis values.  Short traces of current versus time are plotted in order to show the dynamic nature of these current fluctuations.  The \SI{170}{\mV} data seems consistent with the picture of transient deviations from a resting state to deeper-blockage states.  These transient deeper-blockage states last on the order of \SI{1}{\ms} or less, and they get farther away from the main peak as voltage increases.  Poly(dC) is the oligo shown in Figure \ref{fig:polyC_hists}, but all four of the oligos measured in this work show the emergence of a similar deeper-blockage peak.

We speculate that the emergence of this deeper-blockage peak at high biases could be caused by transient interactions between the NeutrAvidin protein and MspA.  Such interactions have been speculated about before by others \citep{Manrao2011} in different contexts.  We propose that at high bias voltages, above \SI{100}{\mV}, the electrophoretic force is strong enough to pull NeutrAvidin into a tightly clamped state where it interacts with MspA and blocks more current.  At low bias voltages, the NeutrAvidin tents to be farther away from MspA more often due to reduced tension in the ssDNA.  The likelihood of NeutrAvidin being in the tightly-clamped state is low at low bias voltages.

An alternative explanation for the emergence of these discrete peaks at high bias could be that the ssDNA interacts in a specific manner with the constriction of MspA, and that these specific interactions are stabilized by the ssDNA being under high tension.  Recent molecular dynamics simulations with ssDNA and MspA have indicated that there are specific interactions between the phosphate backbone of the ssDNA and the pore's constriction that can stabilize high-tension states of the ssDNA \citep{Bhattacharya2016a}.  However, those simulations show that the lifetimes of these states are on the order of \SI{1}{\micro\s} or less, while the fluctuations observed in experiments here are on the order of milliseconds (\SI{170}{\mV} inset of Figure \ref{fig:polyC_hists}).

A final observation about the data in Figure \ref{fig:polyC_hists} is that even though the current at each voltage is normalized to the open pore current at that voltage (and open-pore current is itself non-linear with voltage, see Figure \ref{fig:iv_mspa}a), the fractional blockage $I/I_0$ increases significantly as the bias voltage is increased.  This holds true for all four oligos studied, and it indicates that the ionic conductivity of the blocked pore increases as voltage is increased.  We attribute this to a reduction in the number of nucleotides between the NeutrAvidin and the constriction of MspA as voltage increases, due to a larger force on the ssDNA and increased tension.  Stretching of ssDNA leading to elongation was reported by Stoddart \textit{et al.} in $\alpha$-hemolysin \citep{Stoddart2015}.

\subsection{Current noise measurements}

It is also instructive to examine the noise power spectra of current blockages recorded when ssDNA is trapped in the MspA nanopore.  In light of the current fluctuations to a deeper-blocked state at high voltages, Figure \ref{fig:polyC_fluctuations} shows the current noise power spectral density, $S(f)$, for two individual molecules of poly(dC): one held in the pore at \SI{60}{\mV} bias, and the other at \SI{160}{\mV}, corresponding to the voltages shown in Figure \ref{fig:homopolymer_blockage_hists}.  The power spectrum of the molecule at \SI{60}{\mV} is largely ``white," meaning independent of frequency, from \SI{10}{\Hz} to the filer cutoff near \SI{10}{\kHz}.  The molecule trapped at \SI{160}{\mV} shows an entirely different power spectrum.  The large increase in noise at low frequencies, with a shoulder near \SI{200}{\Hz}, is consistent with a telegraph process where the current fluctuates between two (or more) states \citep{Dutta1981} with a characteristic frequency of about \SI{200}{\Hz} (see Appendix \ref{noise}).  For reference, the open pore noise is plotted in Figure \ref{fig:mspa_noise_intro}.

\begin{figure}[h]
\begin{centering}
\includegraphics[width=0.6\textwidth]{figures/polyC_current_noise.pdf}
\caption[Current fluctuations for single molecule poly(dC) ssDNA]{Current noise power spectral density, $S$, plotted as a function of frequency, for two individual molecules of NeutrAvidin-poly(dC) ssDNA.  One molecule is held at \SI{160}{\mV} bias, and the other at \SI{60}{\mV} bias.  The current fluctuations at high bias show an increase in noise power spectral density similar to telegraph noise.}
\label{fig:polyC_fluctuations}
\end{centering}
\end{figure}

The power spectrum obtained at \SI{160}{\mV} contains valuable information about the process involved in fluctuations to the lower-current state, which we speculate has to do with the NeutrAvidin protein, but the extra noise obscures the baseline level of noise associated with the ssDNA itself.  That baseline level of noise is revealed by the measurement at \SI{60}{\mV}.  The noise power spectra at \SI{60}{\mV} can be compared for all four homopolymers, and they are plotted in Figure \ref{fig:homopolymer_fluctuations}.

\begin{figure}[h]
\begin{centering}
\includegraphics[width=0.7\textwidth]{figures/homopolymer_noise.pdf}
\caption[Current fluctuations for all homopolymer ssDNA]{Current noise power spectral density for the four oligonucleotides tethered to NeutrAvidin and trapped in M3-MspA for at least \SI{5}{\ms} by a \SI{60}{\mV} bias.  The noise power spectral density is averaged over many individual molecules ($n_A=322, n_C=716, n_{GA}=108, n_T=1197$).  The mean white noise, $\bar{S}$, is calculated by averaging the noise from \SI{50}{\Hz} to \SI{500}{\Hz}, labeled the white noise domain.  Capacitive noise begins to dominate above \SI{1}{\kHz}, and produces the visible bump before the filter at \SI{10}{\kHz}.}
\label{fig:homopolymer_fluctuations}
\end{centering}
\end{figure}

Evidently, the noise associated with the ssDNA in MspA is different for each of the oligos in this study.  The noise plotted in Figure \ref{fig:homopolymer_fluctuations} agrees with a qualitative observation of the noise visible on the traces shown in Figure \ref{fig:homopolymer_blockages}, where poly(dGdA) shows visibly more noise than the others, while poly(dT) is the quietest.  The frequency domain of $f \in [50, 500]$\si{\Hz} is appropriate for quantifying the white noise level associated with each oligo, since the molecules are only trapped on the order of \SI{50}{\ms} at an applied bias of \SI{60}{\mV}, and capacitive noise sources begin to dominate at frequencies above \SI{1}{\kHz}.  The mean white noise level between \SI{50}{\Hz} and \SI{500}{\Hz} will be called $\bar{S}$ in the discussion that follows.

\section{Discussion}

\subsection{Thermal motion of the ssDNA in MspA}

Equation \ref{eqn:drift_diffusion2} can be solved to find the probability that the ssDNA will be at a given position at a given time.  This can be averaged over time, giving a quasi-equilibrium position probability, by finding the time-integrated position probability and normalizing.  The solution, with the appropriate reflecting boundary condition at $x=0$ and absorbing boundary condition at $x=L$ is (see Appendix \ref{first_passage}, Equation \ref{eqn:biased_diffusion_solution}):

\begin{equation}
p_{eq}(x) \propto 1 - \exp{\left( -\frac{| \sigma V | (x-L)}{k_B T} \right)}
\label{eqn:biased_diffusion_solution2}
\end{equation}

\begin{figure}[h]
\begin{centering}
\includegraphics[width=\textwidth]{figures/calc_position_fluctuations.pdf}
\caption[Position fluctuations of ssDNA in MspA]{(a) Drift and diffusion lead to a process whereby non-equilibrium position fluctuations of ssDNA in a nanopore relax back to equilibrium.  At a bias voltage of \SI{140}{\mV}, a position fluctuation which puts base number 4 in the pore's constriction (red dashed line) will relax back to nearly the equilibrium distribution (solid blue line) in under \SI{10}{\ns}.  The diffusion constant is \SI{5.9e-11}{\m^2/s}, and the force is \SI{17}{\pico\N} at \SI{140}{\mV}, corresponding to an effective charge density of \num{0.54}$e/\text{base}$ on the ssDNA.  These values come from measurements of poly(dA).  (b) At equilibrium, the distribution of ssDNA position depends on the applied force, or equivalently, the applied bias voltage.  Higher voltages lead to tighter clamping of the ssDNA and less thermal motion.}
\label{fig:position_fluctuations}
\end{centering}
\end{figure}

If there were no absorbing (escape) boundary at $x=L$, this would reduce to $p_{eq}(x) \propto \exp (- \frac{|\sigma V|}{k_B T} x)$, which is what would be expected from the standpoint of a Boltzmann factor.   Plots of Equation \ref{eqn:biased_diffusion_solution2} are shown in Figure \ref{fig:position_fluctuations}b for a few applied voltage biases.  The blue curve shows that at \SI{100}{\mV} bias, the ssDNA spends about 14\% of its time greater than one base away from the fully-clamped position.  However, if the voltage is increased to \SI{300}{\mV}, the ssDNA spends less than 0.3\% of its time greater than one base away from fully-clamped.  This suggests that signal averaging due to thermal motion could be minimized by applying larger bias voltages.

Figure \ref{fig:position_fluctuations}a shows a calculation of the timescale over which non-equilibrium position fluctuations relax back to equilibrium.  This is obtained by numerically solving Equation \ref{eqn:drift_diffusion3}.  At an applied bias of \SI{140}{\mV}, the ssDNA relaxes back a distance of 4 bases in less than \SI{10}{\ns}.  This is in agreement with a rough estimate, since the average velocity is $\bar{v} \approx \sigma V D / k_B T$, and the time it would take to move four bases, $4 L_b = $ \SI{2}{\nm}, would be $t \approx 4 L_b / \bar{v} \approx $ \SI{8}{\ns}.  The picture that emerges is one of constant thermal agitation of the position of the ssDNA, with rapid relaxation to equilibrium.  Over the timescales we can measure (order of microseconds), the system has had time to average over thousands of position configurations.  Thus the equilibrium position probability is sampled thoroughly over even the shortest timescales measured in experiments.

\begin{figure}[h] % this may be too aggressive...
\begin{centering}
\includegraphics[width=\textwidth]{figures/thermal_motion_in_sequence_data.pdf}
\caption[Thermal motion averaging and sequencing data]{Thermal motion of ssDNA is consistent with nanopore sequencing data.  The cartoons show how a few example sequences sit in the constriction of MspA.  The plots show measured current as a function of the position of a single base substitution in a homopolymeric strand.  The single base substitution results in a different current than the homopolymer, but only in some locations.  The fact that the width of this peak is never less than two or three bases is consistent with the thermal motion we report.  Sequence-specific currents come from measurements on M2-MspA made available online by Laszlo \textit{et al.} \citep{Laszlo2014}.}
\label{fig:thermal_motion_sequencing}
\end{centering}
\end{figure}

This measurement of thermal position fluctuations of the ssDNA can be compared to sequencing experiments carried out using M2-MspA by Laszlo \textit{et al.} \citep{Laszlo2014}.  The method used to obtain this data will be explained in Chapter \ref{dna_enzymes}.  Specific sequences of four nucleotides, called ``4-mers," were correlated with their current blockages and reported.  We can use this data to learn about position fluctuations as well.  Take, for example, the 4-mer \texttt{TTTT}.  If we introduce a single \texttt{A}, the current may change from the \texttt{TTTT} level, depending on the position of the \texttt{A}.  In some positions, the single \texttt{A} changes the current significantly, while in other positions, the presence of the \texttt{A} makes no difference.

Figure \ref{fig:thermal_motion_sequencing} shows selected 4-mers and their associated current levels.  The 4-mers are grouped so that single-letter changes in a homopolymer are shown together.  The presence of the single letter change alters the current in some positions and not others.  The region over which the current is altered from the homopolymer is pointed out by the gray arrows.  The data were recorded at \SI{180}{\mV}.  The widths of the peaks are never narrower than two or three bases, and this is consistent with the thermal motion expected from the calculations above.

\subsection{Correlations}

Interesting correlations exist between the measured effective charge and current blockage $I/I_0$, as well as between the diffusion constant and current noise.  In this section, we restrict our discussion to measurements at \SI{60}{\mV} in order to be able to safely ignore extra current fluctuations due to the NeutrAvidin.  The correlation between the effective charge and the mean blockage current is shown in Figure \ref{fig:charge_current_correlation}.  Evidently, smaller residual currents $I/I_0$ correlate with larger effective charges.

\begin{figure}[h]
\begin{centering}
\includegraphics[width=0.6\textwidth]{figures/charge_blockage_correlation.pdf}
\caption[Correlation between effective charge and current]{Correlation between effective charge and mean blockage current.  The larger the blockage current, $I/I_0$, the smaller the effective charge.  Mean values of $I/I_0$ are measured at \SI{60}{\mV}, as shown in Figure \ref{fig:homopolymer_blockage_hists}, and the associated small error bars are the standard error of the mean.  Error bars on effective charge are one standard deviation uncertainty from the first-passage model parameter estimation.}
\label{fig:charge_current_correlation}
\end{centering}
\end{figure}

We hypothesize that the relationship between effective charge and mean blockage current could be caused by the fact that the residual current itself is associated with an osmotic flow.  The force measured in the capture and escape experiment is the sum of the direct electric force and an opposing electroosmotic flow, both of which are proportional to the applied voltage.  The effective charge reflects both of these forces.  DNA has one negative charge per phosphate, and no charge on the nucleobases at pH \num{8}.  With the ssDNA blocking the MspA nanopore, the majority of the ions which can squeeze past the ssDNA are potassium cations.  This has been shown both in molecular dynamics simulations \citep{Bhattacharya2012}, and in experiments \citep{Nova2017}.  The values of effective charge measured here are less than one electronic charge, resulting from the sum of these two opposing forces.

\begin{figure}[h]
\begin{centering}
\includegraphics[width=\textwidth]{figures/diffusion_noise_correlation.pdf}
\caption[Correlation between diffusion and noise]{Correlation between diffusion constant, $D$, and the mean normalized excess white noise, $\bar{N}$, defined in the text.  (a) The four oligonucleotides in this study plausibly fall along the gray curve, denoting $D \propto 1/\bar{N}$.  Large error bars in $D$ reflect parameter estimation uncertainty at one standard deviation.  (b) Further exploration of this correlation, using measurements of poly(dT) at various temperatures.  The gray curve is the same curve plotted in (a).}
\label{fig:diffusion_noise_correlation}
\end{centering}
\end{figure}

Figure \ref{fig:diffusion_noise_correlation} shows the correlation between the measured diffusion constants and the noise.  Each of the four oligos are shown in panel (a).  The diffusion constant is written in terms of $D_0$, the diffusion constant of a \SI{1.5}{\nm} diameter sphere, where \SI{1.5}{\nm} is the approximate Kuhn length of ssDNA \citep{Smith1996}, and a Kuhn length is the length scale over which ssDNA behaves as if it were a freely-jointed segment, independent of the rest of the strand.

The noise is plotted in terms of a quantity we call ``mean normalized excess white noise."  The notion of ``excess white noise" is explained in Figure \ref{fig:homopolymer_fluctuations}.  The dashed green line shows the mean noise, $\bar{S}$, for poly(dGdA) in the frequency range of \num{50} to \SI{500}{\Hz}.  The Johnson noise, $S_J$, for poly(dGdA) is shown in the figure as the dotted black line.  The ``excess white noise" is the difference between the measured noise and the expected Johnson noise for the resistance of the pore blocked by a given oligo.

The measured current noise, $\bar{S}$, includes three contributions: (1) the Johnson noise, $S_J$, (2) noise caused by ion number fluctuations, $S_{ion}$, as discussed in Section \ref{sec:ion_number_fluctuations}, and (3) excess current fluctuations caused by the ssDNA itself, $S_{DNA}$, which may be nucleotide-specific.  The noise can then be written $\bar{S} = S_J + S_{ion} + S_{DNA}$.  We define ``mean normalied excess white noise" as

\begin{equation}
\bar{N} \equiv (\bar{S} - S_J) / \bar{I}^2
\label{eqn:newn}
\end{equation}
 
\noindent
where $I$ is the mean ionic current measured for a given oligo.  It then follows that $\bar{N} = (S_{ion}+S_{DNA})/\bar{I}^2 = C + S_{DNA}/\bar{I}^2$, where $C$ is approximately constant and independent of the nucleotide identity of the oligo.  The reason for division by current squared is so that $S_{DNA}$ can be separated from $S_{ion}$ and compared for different oligos.

We hypothesize that the current fluctuations caused by ssDNA in MspA should have the form $S_{DNA} \propto (\Delta I)^2/D$, where $D$ is the diffusion constant, and $\Delta I$ is the difference in current between minimal and maximal blockages caused by different positions of the ssDNA.  A relation of this type was derived by Rostovtseva \textit{et al.} \citep{Rostovtseva1998, Berezhkovskii2002, Rostovtseva2002} for particle number fluctuations in a membrane channel caused by diffusion.

This sort of relation can also arise in the context of a simple model of telegraph noise.  Assume there are two states: (1) a lower current state when the nucleobase is in the narrow pore constriction, and (2) a higher current state when the space between two nucleobases is in the narrow pore constriction.  Assume these states have a difference in current $\Delta I$.  A telegraph process of flipping between these states with a characteristic time scale $\tau$ gives rise to the power spectral density \citep{Dutta1981} $S(f) = (\Delta I)^2 \tau \, / \left[ 1 + (2 \pi \tau \, f)^2 \right]$.  For fluctuations that happen on very fast timescales, like those predicted here, the frequency $2\pi/\tau$ is much higher than frequencies we can measure (above \SI{10}{\kHz}).  If the fluctuation timescale is set by diffusion, then we expect $\tau \approx L_b^2 / D$, where $L_b$ is the length per base, which is \SI{0.5}{\nm}.  This timescale, as mentioned above, is on the order of nanoseconds.  Within the frequency range of interest for our measurements, from \num{50} to \SI{500}{\Hz}, the power spectral density from a telegraph process appears to be $S(f) \propto (\Delta I)^2/D$, independent of frequency.

Figure \ref{fig:diffusion_noise_correlation}a shows the relation $\bar{N} \propto 1/D$ as a gray line.  The relation appears plausible; however, the limited number of data points from the four homopolymers and the large error bars in measurement of the diffusion constants of the oligos at \SI{25}{\celsius} prevents a stronger degree of certainty.  Panel (b) shows the same correlation between $D$ and $\bar{N}$ for poly(dT) measured at several temperatures.  Changing the temperature changes the diffusion constant and the noise simultaneously.  The gray line in panel (b) is the same gray line as in panel (a), and shows a convincing $D \propto 1 / \bar{N}$ relationship.

\subsection{Implications}

It was demonstrated in Figure \ref{fig:thermal_motion_sequencing} that thermal motion of the ssDNA should make its effects felt when a single base substitution, such as \texttt{A}, is in different locations in a homopolymeric strand, such as poly(dT).  The substitution affects the measured current in some locations, but not others.  The current is noticeably different when the substitution is in several neighboring locations.  The number of these neighboring locations is never less than 2-3 bases, consistent with the thermal motion measured by capture and escape experiments.

Another independent measure of thermal motion in sequencing experiments can be obtained by looking at a transition between poly(dT) and poly(X), where ``X" stands for an 18-carbon abasic spacer from Integrated DNA Technologies (see Appendix \ref{idt_dna} for chemical structure).  A region of poly(dT) is held in the MspA nanopore by a helicase enzyme and stepped through the nanopore one base at a time as the sequence goes from poly(dT) to poly(X).  The data from such an experiment is shown in Figure \ref{fig:thermal_motion_sequencing2}.

\begin{sidewaysfigure}[h]
\begin{centering}
\includegraphics[width=\textwidth]{figures/TTTTT_to_XXXXX.pdf}
\caption[Thermal motion averaging with voltage]{Traces of current recorded as ssDNA steps through MspA nanopore base by base from poly(dT) to poly(X), where X is an 18-carbon abasic spacer.  The number of current levels in this transition reveals the number of bases that contribute to the current signal.  This number depends on voltage, as expected from thermal motion.  Initial current levels are poly(dT), and the last current levels are poly(X).  Current shows a sharp transition at each step of the ssDNA, by one base, through the MspA nanopore.}
\label{fig:thermal_motion_sequencing2}
\end{centering}
\end{sidewaysfigure}

If MspA were averaging over 2 nucleotides, the transition would show 3 distinct current levels: those corresponding to \texttt{TT}, \texttt{TX}, and \texttt{XX}.  If MspA were averaging over 4 nucleotides, the transition would show 5 distinct current levels: those corresponding to \texttt{TTTT}, \texttt{TTTX}, and \texttt{TTXX}, \texttt{TXXX}, and \texttt{XXXX}.  Figure \ref{fig:thermal_motion_sequencing2} shows that the number of nucleotides averaged over does in fact depend on voltage, consistent with the notion of thermal motion contributing to this averaging.  There are 12 levels measured in the transition at \SI{80}{\mV} applied bias, while there are 8 levels at \SI{160}{\mV}.  The difference of 4 bases at the different voltages is the product of thermal motion of the ssDNA in the MspA nanopore.
