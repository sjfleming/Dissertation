\begin{savequote}[75mm]
Less is more.
\end{savequote}

\chapter{Detailed protocol for nanopore experiments}
\label{protocol}

\section{General setup and operation}

Measurements of ionic current through a nanopore in a lipid bilayer are obtained by applying a voltage across the lipid bilayer in an aqueous solution of potassium chloride [KCl] using silver-silver chloride [Ag/AgCl] electrodes.

Our setup uses a Teflon “patch stand” to hold two volumes of buffer solution which are separated by a lipid bilayer.  A separate Ag/AgCl electrode makes contact to each of these two volumes of buffer.  The buffer chamber closest to the operator is electrically grounded, and is termed the “cis side.”  The buffer chamber attached to the headstage of the current amplifier is termed the “trans side.”  During the experiment, all manipulations by the experimenter take place on the cis side.

The cis side and the trans side are connected by a small, U-shaped tube, also made of Teflon.  At the trans side, the tube is wide open, and the opening of the tube in the cis side is terminated with a cone-shaped void in a Teflon piece with a very small (~40 micron) aperture.  This small aperture forms the only opening for an electrical connection between the cis and the trans side.  Thus, spreading a lipid bilayer across this aperture entirely blocks the flow of ionic current from one electrode to the other.

\section{Preparation of the electrodes}

The electrodes are Ag/AgCl pellets connected to a copper connector.  The pellets are housed in a Teflon sheath, made to fit into the patch stand.  Electrodes should be cleaned with ethanol and DI water, and prepared in the following way before each experiment.

Take a very small amount of bleach (just enough to cover Ag/AgCl but not much of the Teflon, about 70uL) in a tiny (1.5mL) eppendorf tube.  Put the electrode into the bleach so that only the Ag/AgCl is submerged.  Do this for both electrodes.  Wait for 5 minutes while surface Ag is converted to AgCl.  The color of fresh Ag electrodes should change from light silver to a dark brownish color.  If something looks wrong, sand down the surface of the electrodes gently with fine grit sand paper and retry.

\begin{figure}[h]
\begin{centering}
\includegraphics[width=0.2\textwidth]{figures/protocol_electrodes.pdf}
\caption[Preparing silver chloride electrodes]{Cartoon diagram showing the process for bleaching the electrodes in order to convert surface Ag to AgCl.  Electrodes should be dipped in bleach for about 5 minutes.}
\label{fig:protocol_electrodes}
\end{centering}
\end{figure}

\section{Patch stand preparation}

Sensitive measurements of single molecules require the patch stand to be thoroughly cleaned before and after each use to avoid any problems with contamination.  Preparation of the patch stand and patch tube is described here.  The patch stand is normally stored in water after cleaning.  Remove the patch stand from the water and shake off to dry slightly.

\begin{itemize}

\item Water

\begin{itemize}
\item Insert the three Teflon plugs into the cis side.  Fill the cis side with deionized [DI] water (\SI{18}{\mega\ohm\cm}).  Attach the water syringe to the U-tube in the trans side and gently suck the water through the U-tube from the cis side until all the water is gone.  Refill the cis side with water and repeat.
\end{itemize}

\item Dry

\begin{itemize}
\item Remove the buffer syringe and the Teflon plugs and dry the patch stand with nitrogen.  Attach the air syringe to the U-tube in the trans side and pull air through the U-tube / pull a vacuum inside the U-tube.  Repeat until the U-tube is dry inside.
\end{itemize}

\item Ethanol

\begin{itemize}
\item Dip the entire patch stand into a beaker of \num{200} proof ethanol and swirl.  Remove patch stand from the ethanol and shake off to dry slightly.  Insert the three Teflon plugs into the cis side.  Fill the cis side with \num{200} proof ethanol.  Attach the solvent syringe to the U-tube in the trans side and gently suck the ethanol through the U-tube from the cis side until all the ethanol is gone.  Refill the cis side with ethanol and repeat.
\end{itemize}

\item Dry

\begin{itemize}
\item Same as above.
\end{itemize}

\item Hexane

\begin{itemize}
\item Insert the three Teflon plugs into the cis side.  Fill the cis side with clean hexane.  Attach the solvent syringe to the U-tube in the trans side and gently suck the hexane through the U-tube from the cis side until all the hexane is gone.  Refill the cis side with hexane and repeat.
\end{itemize}

\item Dry

\begin{itemize}
\item Same as above.
\end{itemize}

\end{itemize}

\section{Patch stand preparation: lipid ``pre-paint”}

Without pre-coating the Teflon patch tube's aperture with lipid, the membranes will be too fragile.  In order to enhance membrane stability, a lipid pre-paint step is required.  Start with a patch stand that has been prepared according to the preparation protocol.  The patch stand should be completely dry.

\subsection{Preparing lipid in hexane}

Take a fresh \SI{0.4}{\m\g} tube of PC lipid (Avanti Polar Lipids, Inc., dessicated and stored at \SI{-20}{\celsius}).  Add \SI{200}{\micro\L} hexane and gently mix by flicking the tube.  Store in the refrigerator in a ``humidor” of hexane.

\subsection{Pre-paint}

Set up the patch stand under a dissecting microscope and look at the end of the patch tube where the aperture is (the cis side).  Using a long pipette tip, withdraw \SI{5}{\micro\L} of PC/hexane.  Drop onto the Teflon aperture, one drop at a time, allowing the solution to wick into the patch tube.  As each drop is applied, give several seconds for some of the hexane to evaporate.  After applying all \SI{5}{\micro\L} of PC/hexane to the aperture, attach the air syringe to the trans side of the patch tube and push air through the tube, from trans to cis, expelling excess PC/hexane from the patch tube and drying it.

Allow at least \num{20} minutes for the patch stand to dry fully.

\section{Setting up the patch stand to start an experiment}

After the patch stand has been pre-painted with PC lipid, set up the patch stand in its copper block holder, with the Ag/AgCl electrodes and the buffer flow lines.  Choose a buffer to use.  Typically, we use \SI{1}{\Molar} KCl, \SI{10}{\m\Molar} HEPES, at pH \num{8.0}.  Pull some buffer into the buffer syringe, and use it to fill the U-tube from trans side to cis.  Once the patch tube is filled with buffer and buffer has started to come through the cis side aperture, stop and remove the buffer syringe.  Fill the cis buffer flow lines with buffer using the syringes attached to each (one inlet, one outlet, see Figure \ref{fig:protocol_perfusion}).  Fill the cis and trans sides of the patch stand with buffer, ensuring there are no bubbles and that there is good contact between the solution and the electrodes.

At this point, the current amplifier should probably be overloaded.  If even a small voltage is applied by the amplifier, the current should certainly overload.  This indicates that ionic current is flowing freely, and the patch stand has been set up correctly.  If you see zero current, there is likely a bubble present somewhere, either over one electrode, or on/under the aperture, or deep in the patch tube.  Remove the bubble to achieve ionic current flow.  A bubble in the patch tube can be removed by pushing fresh buffer trans to cis using the buffer syringe.

\section{Creation of a lipid membrane}

The nanopore experiment is a single-molecule experiment.  A large amount of lipid, oil, protein, etc. is not necessary.  Less is more.  The goal of the experiment is to establish a single nanopore in a bilayer membrane using as little material as possible.

Based on the kind instruction I received from Andrew Smith and Robin Abu-Shumays in the lab of Mark Akeson at the University of California at Santa Cruz, I learned of the ``lipid ball" technique.  This technique for membrane formation allowed experiments to last for days, where before they lasted for hours at most.  This technique is an art form, but one well worth learning for these single-channel nanopore experiments.

\subsection{Lipid ball technique}

First, dry ``lipid drops" are needed, which are dried down lipid on a glass coverslip, each ``drop" being maybe \SI{2}{\mm} in diameter.  To make these, add \SI{40}{\micro\L} of chloroform to a \SI{0.4}{\m\g} tube of dessicated PC from the freezer.  Take \num{4} clean glass coverslips.  At each of the four corners of each coverslip, drop a small \SI{1}{\micro\L} drop of PC/chloroform and let the chloroform evaporate.  Repeat this, dropping drops on top of each dried PC spot in turn, over and over until the PC/chloroform is all gone.  Store these coverslips of ``lipid drops" in vacuum in a bell jar.

Using one of these glass coverslips, pick one of the four lipid drops and use it to create a ``lipid ball."  This is accomplished by re-hydrating the dried lipid very slowly with 1-hexadecene (Sigma Aldrich, Inc.) using a paintbrush with only a single bristle.  The bristle should be pliable.  Dip the bristle into hexadecene.  Under a microscope, use the bristle as a squeegee to gently scrape the dry lipid up from one side.  Push the bristle against the dry lipid.  Dip the bristle in hexadecene again.  Repeat this dipping and pushing from the side hundreds of times.  The lipid will reluctantly begin to re-hydrate.  The process should not be rushed, and will take about \num{10} minutes.  Eventually, the entire spot of dried lipid will be hydrated with hexadecene, and will be optically transparent and very uniform.  If it is not uniform, use the bristle to ``knead" the lipid ball, folding it over on itself to mix it.  The ideal lipid ball is viscous but not solid, sticky but not like rubber cement, and will form a tiny tendril of lipid/hexadecene attached to the bristle when the bristle is pulled away from the ball.  The ball should be quite wet, just barely solid enough to not disintegrate into a liquid.

Using another brush with a single thick, less flexible bristle, pick up the lipid ball.  The ball is now stuck to the tip of the thick bristle.  Use this bristle to dunk the lipid ball into the cis side buffer and touch the lipid ball to the surface of the patch tube where the aperture is.  Use the bristle to roll the lipid ball around the surface of the patch tube.  As it rolls, the lipid ball will leave a slight trail of PC/hexadecene.  This trail will be the reservoir which maintains the stability of the membrane.  Once the lipid ball has been rolled around and over the aperture, use the bristle to remove the lipid ball from the cis side, and place the ball back on its glass coverslip, where it can be stored for immediate re-use the same day, if necessary.

\subsection{Using an air bubble to establish a membrane}

Dip a clean, long (gel-loading) pipette tip into the cis side buffer.  Looking into the microscope, blow an air bubble out of the pipette tip (holding the tip under water), and while the air bubble is still attached to the pipette tip, rub the bubble across the aperture.  Remove the bubble by suckin git back up.  This process is depicted in Figure \ref{fig:protocol_bubble}.  Watch the ionic current during this process (or, preferably, listen to the current using a voltage-to-audio speaker box connected to the output of the Axopatch).  A seal can be established that blocks ionic current, and the membrane can be destroyed and recreated in this way many times.  The membrane is shown from a side-view in Figure \ref{fig:protocol_membrane}.

\begin{figure}[h]
\begin{centering}
\includegraphics[width=0.9\textwidth]{figures/protocol_bubble.pdf}
\caption[Establishing a lipid membrane with a bubble]{These four cartoon images depict the formation of a lipid membrane using an air bubble expressed from the end of a long pipette tip.  When a bubble touches down and spans the aperture, current ceases flowing.  When the bubble is removed, most of the time, a membrane remains across the aperture.}
\label{fig:protocol_bubble}
\end{centering}
\end{figure}

\begin{figure}[h]
\begin{centering}
\includegraphics[width=0.8\textwidth]{figures/protocol_pic_membrane.png}
\caption[Diagram of membrane on Teflon patch tube]{Membrane forms across the open aperture in the Teflon patch tube.  Lipid and hexadecene oil are adhered to the Teflon, and after applying and removing an air bubble, a lipid membrane remains.  If everything is right, the lipid membrane has an area of thin bilayer.}
\label{fig:protocol_membrane}
\end{centering}
\end{figure}

Note: too much of the hexadecene oil can clog the aperture entirely.  If the lipid ball has been smeared over the aperture and too much remains, it can also block the aperture.  This excess can be cleared by gently using the thick bristle, or by blasting buffer through the patch tube.

Break the membrane and reform it.  The membrane should be stable at \SI{300}{\m\V}.  Use the ``seal test” feature on the AxoPatch to look at the capacitive current spike that results when a voltage step is applied.  A diagram of the current spikes is shown in Figure \ref{fig:protocol_capacitance}.  The size of the current spike is proportional to the capacitance.  More bilayer area means higher capacitance, so break and reform new membranes and look for the size of the spike to increase.  Keep forming membranes and try to get a high capacitance membrane.

\begin{figure}[h]
\begin{centering}
\includegraphics[width=0.9\textwidth]{figures/protocol_mem_capacitance.pdf}
\caption[Monitoring membrane capacitance]{The transient current spike that results from the application of a voltage step is proportional to the capacitance of the membrane.  A larger bilayer area results in a larger capacitance.  Therefore, bilayer membrane formation can be inferred by monitoring the capacitive current spikes that result from abrupt steps in the applied voltage.}
\label{fig:protocol_capacitance}
\end{centering}
\end{figure}



\begin{figure}[h]
\begin{centering}
\includegraphics[width=0.8\textwidth]{figures/protocol_pic_membrane_pores.png}
\caption[Diagram of nanopore introduction]{...}
\label{fig:protocol_pores}
\end{centering}
\end{figure}

\begin{figure}[h]
\begin{centering}
\includegraphics[width=0.8\textwidth]{figures/protocol_pic_membrane_single_pore.png}
\caption[Diagram of single nanopore capture]{...}
\label{fig:protocol_single_pore}
\end{centering}
\end{figure}

\begin{figure}[h]
\begin{centering}
\includegraphics[width=\textwidth]{figures/protocol_perfusion.pdf}
\caption[Setup for removing nanopores by buffer perfusion]{...}
\label{fig:protocol_perfusion}
\end{centering}
\end{figure}

\section{Sample Preparation}
\label{sample_prep}

\subsection{Handling of reagents}



\subsection{Preparation of NeutrAvidin/ssDNA complexes}



\subsection{Preparation of Dda/DNA complexes}

