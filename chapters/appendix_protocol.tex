\chapter{Detailed protocol for nanopore experiments}
\label{protocol}

\section{General setup and operation}

Measurements of ionic current through a nanopore in a lipid bilayer are obtained by applying a voltage across the lipid bilayer in an aqueous solution of potassium chloride [KCl] using silver-silver chloride [Ag/AgCl] electrodes.

Our setup uses a Teflon “patch stand” to hold two volumes of buffer solution which are separated by a lipid bilayer.  A separate Ag/AgCl electrode makes contact to each of these two volumes of buffer.  The buffer chamber closest to the operator is electrically grounded, and is termed the “cis side.”  The buffer chamber attached to the headstage of the current amplifier is termed the “trans side.”  During the experiment, all manipulations by the experimenter take place on the cis side.

The cis side and the trans side are connected by a small, U-shaped tube, also made of Teflon.  At the trans side, the tube is wide open, and the opening of the tube in the cis side is terminated with a cone-shaped void in a Teflon piece with a very small (~40 micron) aperture.  This small aperture forms the only opening for an electrical connection between the cis and the trans side.  Thus, spreading a lipid bilayer across this aperture entirely blocks the flow of ionic current from one electrode to the other.

\begin{figure}[h]
\begin{centering}
\includegraphics[width=0.2\textwidth]{figures/protocol_electrodes.pdf}
\caption[Preparing silver chloride electrodes]{...}
\label{fig:protocol_electrodes}
\end{centering}
\end{figure}

\begin{figure}[h]
\begin{centering}
\includegraphics[width=0.9\textwidth]{figures/protocol_bubble.pdf}
\caption[Establishing a lipid membrane with a bubble]{...}
\label{fig:protocol_bubble}
\end{centering}
\end{figure}

\begin{figure}[h]
\begin{centering}
\includegraphics[width=0.8\textwidth]{figures/protocol_pic_membrane.png}
\caption[Diagram of membrane on Teflon patch tube]{...}
\label{fig:protocol_membrane}
\end{centering}
\end{figure}

\begin{figure}[h]
\begin{centering}
\includegraphics[width=0.9\textwidth]{figures/protocol_mem_capacitance.pdf}
\caption[Monitoring membrane capacitance]{...}
\label{fig:protocol_capacitance}
\end{centering}
\end{figure}

\begin{figure}[h]
\begin{centering}
\includegraphics[width=0.8\textwidth]{figures/protocol_pic_membrane_pores.png}
\caption[Diagram of nanopore introduction]{...}
\label{fig:protocol_pores}
\end{centering}
\end{figure}

\begin{figure}[h]
\begin{centering}
\includegraphics[width=0.8\textwidth]{figures/protocol_pic_membrane_single_pore.png}
\caption[Diagram of single nanopore capture]{...}
\label{fig:protocol_single_pore}
\end{centering}
\end{figure}

\begin{figure}[h]
\begin{centering}
\includegraphics[width=\textwidth]{figures/protocol_perfusion.pdf}
\caption[Setup for removing nanopores by buffer perfusion]{...}
\label{fig:protocol_perfusion}
\end{centering}
\end{figure}