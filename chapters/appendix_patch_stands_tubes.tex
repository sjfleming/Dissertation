\chapter{Patch stands and patch tubes}
\label{patch_stands_tubes}

\section{Physical setup for a nanopore experiment}

A schematic diagram of the physical setup for a nanopore experiment is shown in Figure \ref{fig:physical_setup}.  The lipid bilayer membrane is spread over a tiny aperture in one end of a Teflon tube.  This tube, called the ``patch tube," is filled with buffer, and connects two reservoirs (about \SI{200}{\uL} volume each) of buffer, called ``cis" and ``trans," which are housed in a machined piece of Teflon, called the ``patch stand."  Each reservoir of buffer is contacted by an electrode of Ag/AgCl, sheathed in Teflon, which can be removed.  The patch stand and patch tube are all Teflon.

The patch stand is snugly press-fit into a copper stand, also shown in Figure \ref{fig:physical_setup}, which is mounted on a thermoelectric heater/cooler.  The temperature of the copper stand is monitored by a thermistor in contact with the outside of the stand, and is controlled by a Newport 3040 temperature controller (Newport Co., Irvine, CA).  A water flow heat exchanger is mounted below, and water can be flowed through using a Haake DC10-P5 water circulator.

\begin{figure}[h]
\begin{centering}
\includegraphics[width=0.6\textwidth]{figures/setup.pdf}
\caption[Diagram of physical experimental setup]{...}
\label{fig:physical_setup}
\end{centering}
\end{figure}



\section{The ``patch stand"}

\begin{figure}[h]
\begin{centering}
\includegraphics[width=0.7\textwidth]{figures/protocol_diagram.pdf}
\caption[Schematic of patch stand]{...}
\label{fig:patch_stand}
\end{centering}
\end{figure}

\section{The ``patch tube"}

\subsection{Fabrication using the traditional method}

\subsection{Fabrication using a focused ion beam}