\begin{savequote}[75mm]
blank
\qauthor{blank}
\end{savequote}

\chapter{Measuring helicase movement along DNA}
\label{helicase_on_dna}

\section{Experimental setup}

\begin{figure}[h]
\begin{centering}
\includegraphics[width=0.95\textwidth]{figures/helicase_scheme.pdf}
\caption[Helicase can control DNA translocation]{(a) Helicase is bound to ssDNA at the junction where the ssDNA meets an unzipping duplex.  (b) The scheme for using a helicase in a nanopore experiment.  The helicase, perched atop the nanopore, is too large to fit into the pore.  Because it is bound to the ssDNA strand, the helicase prevents the DNA from translocating.}
\label{fig:helicase_scheme}
\end{centering}
\end{figure}

Figure \ref{fig:helicase_scheme} depicts the experimental scheme for using a helicase enzyme to slow the translocation of DNA through the nanopore, from the cis to the trans side of the lipid membrane.  The helicase, bound to the ssDNA strand at the ssDNA – dsDNA junction, unwinds the dsDNA duplex by stepping forward along the ssDNA, deriving its motion from the hydrolysis of ATP.  When the free 5’ end of the ssDNA strand is pulled into the MspA nanopore, the helicase ends up perched on top of MspA.  The helicase is too large to fit into the vestibule of MspA, and so it is clamped in place by the electrophoretic force which pulls the DNA through the nanopore.  The extent of this clamping, and the thermal motion of the helicase in this clamped state, was explored in Chapter \ref{dna_thermal_motion_mspa}, where now the helicase is playing the role previously played by the NeutrAvidin protein.

Of course, unlike the NeutrAvidin protein, the helicase is able to step along the ssDNA.  This helicase is always oriented with its front facing the 3’ end of the ssDNA strand to which it is bound (see the crystal structure in Figures \ref{fig:helicase_stereo} and \ref{fig:helicase_ssDNA_interactions}).  It will be shown that the helicase is capable of moving forward (5’ to 3’) and backward (3’ to 5’) along ssDNA; however, the steps associated with hydrolysis of ATP are thought to be exclusively in the forward (5’ to 3’) direction.

\section{Experimental data for helicase stepping along DNA}

\begin{figure}[h]
\begin{centering}
\includegraphics[width=\textwidth]{figures/helicase_data.pdf}
\caption[Data for helicase stepping along DNA]{(a) Current recorded when one particular Dda helicase / ssDNA complex is held in the M2-MspA nanopore at \SI{120}{\mV} bias using the experimental setup in Figure \ref{fig:helicase_scheme}b, with ATP powering the helicase.  (b) Same data, but with the y-axis zoomed in and the data discretized into stable levels (shown in red) with sudden jumps between them.  Buffer is \SI{1}{\Molar} KCl, \SI{25}{\m\Molar} K-phosphate, pH \num{8.00}, with \SI{2}{\m\Molar} MgCl$_2$ and \SI{2}{\m\Molar} ATP.}
\label{fig:helicase_data}
\end{centering}
\end{figure}

\begin{figure}[h]
\begin{centering}
\includegraphics[width=0.8\textwidth]{figures/helicase_stepping_three_molecules.pdf}
\caption[Helicase stepping along three DNA molecules]{Data from three individual molecules.  The current levels shown are those that correspond to the same region of the DNA molecule as the current levels from about \SI{2.3}{\s} to about \SI{4.5}{\s} in the data shown in Figure \ref{fig:helicase_data}.  The discrete levels have been labeled with the same numbers in each dataset.  The length of time each stable current level lasts, as well as the presence or absence of particular levels, gives a feel for the variability between individual molecules.}
\label{fig:helicase_data_3}
\end{centering}
\end{figure}

\section{Data analysis}

\begin{figure}[h]
\begin{centering}
\includegraphics[width=\textwidth]{figures/helicase_levels_several_mols.pdf}
\caption[Current levels are reproducible molecule to molecule]{ ...}
\label{fig:helicase_data_aligned}
\end{centering}
\end{figure}

\section{Movement of helicase along DNA is revealed by current levels}

\begin{figure}[h]
\begin{centering}
\includegraphics[width=0.95\textwidth]{figures/helicase_sequencing.pdf}
\caption[Sequencing DNA with the help of a helicase]{(a) Initial position of helicase on DNA.  The five ssDNA bases in the nanopore constriction are \texttt{ATTCG}.  (b) After the helicase takes a single, ATP-driven step, the five bases in the nanopore constriction are \texttt{TTCGC}.  (c) Schematic of measured current that would correspond to the situations in panels (a) and (b) shows a sharp jump between two stable mean current values, which occurs precisely when the helicase takes a step.}
\label{fig:helicase_sequencing}
\end{centering}
\end{figure}

In the configuration shown in Figure \ref{fig:helicase_sequencing}a, an ATP-driven step of the helicase will result in the helicase being positioned one base closer to the 3’ end of the DNA, shown in panel (b), and a single base of the duplex will have been unzipped in the process.  As soon as the helicase moves, the electrophoretic force on the DNA causes it to be pulled through the MspA nanopore by one base.  Before the ATP-driven step, the 5 base sequence near the narrow constriction of MspA was \texttt{ATTCG}.  The 5-base sequence near the nanopore’s constriction is referred to as the “5-mer.”  After the ATP-driven step, the 5-mer is \texttt{TTCGC}.  As shown in Figure \ref{fig:helicase_sequencing}c, the measured current depends on the 5-mer, and there is an abrupt step in the current when the helicase takes a step.

\begin{figure}[h]
\begin{centering}
\includegraphics[width=\textwidth]{figures/helicase_position_time.pdf}
\caption[Current levels reveal helicase movement]{(a) ...}
\label{fig:helicase_movement}
\end{centering}
\end{figure}

\section{Discussion}